\documentstyle[titlepage,longtable]{article}
% NetHack 3.7  Guidebook.tex $NHDT-Date: 1431192762 2015/12/16 17:32:42 $  $NHDT-Branch: master $:$NHDT-Revision: 1.60 $ */
%+% we're still limping along in LaTeX 2.09 compatibility mode
%-%\documentclass{article}
%-%\usepackage{hyperref} % before longtable
%-%% if hyperref isn't available, we can get by with this instead
%-%%\RequirePackage[errorshow]{tracefnt} \DeclareSymbolFont{typewriter}{OT1}{cmtt}{m}{n}
%-%\usepackage{longtable}
\textheight 220mm
\textwidth 160mm
\oddsidemargin 0mm
\evensidemargin 0mm
\topmargin 0mm

\newcommand{\nd}{\noindent}

\newcommand{\tb}[1]{\tt #1 \hfill}
\newcommand{\bb}[1]{\bf #1 \hfill}
\newcommand{\ib}[1]{\it #1 \hfill}

\newcommand{\blist}[1]
{\begin{list}{$\bullet$}
    {\leftmargin 30mm \topsep 2mm \partopsep 0mm \parsep 0mm \itemsep 1mm
     \labelwidth 28mm \labelsep 2mm
     #1}}

\newcommand{\elist}{\end{list}}

% this will make \tt underscores look better, but requires that
% math subscripts will never be used in this document
\catcode`\_=12

\begin{document}
%
% input file: guidebook.mn
%
%.ds h0 "
%.ds h1 %.ds h2 \%
%.ds f0 "

%.mt
\title{\LARGE A Guide to the Mazes of Menace:\\
\Large Guidebook for {\it NetHack\/}}

%.au
\author{Original version - Eric S. Raymond\\
(Edited and expanded for 3.7.0 by Mike Stephenson and others)}
%DO NOT REMOVE NH_DATESUB \date{DATE(%B %-d, %Y)}
\date{May 7, 2023}

\maketitle

%.pg
%.hn 1
\section{Introduction}

%.pg

Recently, you have begun to find yourself unfulfilled and distant
in your daily occupation.  Strange dreams of prospecting, stealing,
crusading, and combat have haunted you in your sleep for many months,
but you aren't sure of the reason.  You wonder whether you have in
fact been having those dreams all your life, and somehow managed to
forget about them until now.  Some nights you awaken suddenly
and cry out, terrified at the vivid recollection of the strange and
powerful creatures that seem to be lurking behind every corner of the
dungeon in your dream.  Could these details haunting your dreams be real?
As each night passes, you feel the desire to enter the mysterious caverns
near the ruins grow stronger.  Each morning, however, you quickly put
the idea out of your head as you recall the tales of those who entered
the caverns before you and did not return.  Eventually you can resist
the yearning to seek out the fantastic place in your dreams no longer.
After all, when other adventurers came back this way after spending time
in the caverns, they usually seemed better off than when they passed
through the first time.  And who was to say that all of those who did
not return had not just kept going?
%.pg

Asking around, you hear about a bauble, called the Amulet of Yendor by some,
which, if you can find it, will bring you great wealth.  One legend you were
told even mentioned that the one who finds the amulet will be granted
immortality by the gods.  The amulet is rumored to be somewhere beyond the
Valley of Gehennom, deep within the Mazes of Menace.  Upon hearing the
legends, you immediately realize that there is some profound and
undiscovered reason that you are to descend into the caverns and seek
out that amulet of which they spoke.  Even if the rumors of the amulet's
powers are untrue, you decide that you should at least be able to sell the
tales of your adventures to the local minstrels for a tidy sum, especially
if you encounter any of the terrifying and magical creatures of
your dreams along the way.  You spend one last night fortifying yourself
at the local inn, becoming more and more depressed as you watch the odds
of your success being posted on the inn's walls getting lower and lower.

%.pg
\nd In the morning you awake, collect your belongings, and
set off for the dungeon.  After several days of uneventful
travel, you see the ancient ruins that mark the entrance to the
Mazes of Menace.  It is late at night, so you make camp at the entrance
and spend the night sleeping under the open skies.  In the morning, you
gather your gear, eat what may be your last meal outside, and enter the
dungeon\ldots

%.hn 1
\section{What is going on here?}

%.pg
You have just begun a game of {\it NetHack}.  Your goal is to grab as much
treasure as you can, retrieve the Amulet of Yendor, and escape the
Mazes of Menace alive.

%.pg
Your abilities and strengths for dealing with the hazards of adventure
will vary with your background and training:

%.pg
%
\blist{}
\item[\bb{Archeologists}]%
understand dungeons pretty well; this enables them
to move quickly and sneak up on the local nasties.  They start equipped
with the tools for a proper scientific expedition.
%.pg
%
\item[\bb{Barbarians}]%
are warriors out of the hinterland, hardened to battle.
They begin their quests with naught but uncommon strength, a trusty hauberk,
and a great two-handed sword.
%.pg
%
\item[\bb{Cavemen {\rm and} Cavewomen}]
start with exceptional strength, but unfortunately, neolithic weapons.
%.pg
%
\item[\bb{Healers}]%
are wise in medicine and apothecary.  They know the
herbs and simples that can restore vitality, ease pain, anesthetize,
and neutralize
poisons; and with their instruments, they can divine a being's state
of health or sickness.  Their medical practice earns them quite reasonable
amounts of money, with which they enter the dungeon.
%.pg
%
\item[\bb{Knights}]%
are distinguished from the common skirmisher by their
devotion to the ideals of chivalry and by the surpassing excellence of
their armor.
%.pg
%
\item[\bb{Monks}]%
are ascetics, who by rigorous practice of physical and mental
disciplines have become capable of fighting as effectively without weapons
as with.  They wear no armor but make up for it with increased mobility.
%.pg
%
\item[\bb{Priests {\rm and} Priestesses}]%
are clerics militant, crusaders
advancing the cause of righteousness with arms, armor, and arts
thaumaturgic.  Their ability to commune with deities via prayer
occasionally extricates them from peril, but can also put them in it.
%.pg
%
\item[\bb{Rangers}]%
are most at home in the woods, and some say slightly out
of place in a dungeon.  They are, however, experts in archery as well
as tracking and stealthy movement.
%.pg
%
\item[\bb{Rogues}]%
are agile and stealthy thieves, with knowledge of locks,
traps, and poisons.  Their advantage lies in surprise, which they employ
to great advantage.
%.pg
%
\item[\bb{Samurai}]%
are the elite warriors of feudal Nippon.  They are lightly
armored and quick, and wear the %
{\it dai-sho}, two swords of the deadliest
keenness.
%.pg
%
\item[\bb{Tourists}]%
start out with lots of gold (suitable for shopping with),
a credit card, lots of food, some maps, and an expensive camera.  Most
monsters don't like being photographed.
%.pg
%
\item[\bb{Valkyries}]%
are hardy warrior women.  Their upbringing in the harsh
Northlands makes them strong, inures them to extremes of cold, and instills
in them stealth and cunning.
%.pg
%
\item[\bb{Wizards}]%
start out with a knowledge of magic, a selection of magical
items, and a particular affinity for dweomercraft.  Although seemingly weak
and easy to overcome at first sight, an experienced Wizard is a deadly foe.
\elist

%.pg
You may also choose the race of your character (within limits; most
roles have restrictions on which races are eligible for them):

%.pg
%
\blist{}
\item[\bb{Dwarves}]%
are smaller than humans or elves, but are stocky and solid
individuals.  Dwarves' most notable trait is their great expertise in mining
and metalwork.  Dwarvish armor is said to be second in quality not even to the
mithril armor of the Elves.
%.pg
%
\item[\bb{Elves}]%
are agile, quick, and perceptive; very little of what goes
on will escape an Elf.  The quality of Elven craftsmanship often gives
them an advantage in arms and armor.
%.pg
%
\item[\bb{Gnomes}]%
are smaller than but generally similar to dwarves.  Gnomes are
known to be expert miners, and it is known that a secret underground mine
complex built by this race exists within the Mazes of Menace, filled with
both riches and danger.
%.pg
%
\item[\bb{Humans}]%
are by far the most common race of the surface world, and
are thus the norm to which other races are often compared.  Although
they have no special abilities, they can succeed in any role.
%.pg
%
\item[\bb{Orcs}]%
are a cruel and barbaric race that hate every living thing
(including other orcs).  Above all others, Orcs hate Elves with a passion
unequalled, and will go out of their way to kill one at any opportunity.
The armor and weapons fashioned by the Orcs are typically of inferior quality.
\elist

%.hn 1
\section{What do all those things on the screen mean?}
%.pg
On the screen is kept a map of where you have been and what you have
seen on the current dungeon level; as you explore more of the level,
it appears on the screen in front of you.

%.pg
When {\it NetHack\/}'s ancestor {\it rogue\/} first appeared, its screen
orientation was almost unique among computer fantasy games.  Since
then, screen orientation has become the norm rather than the
exception; {\it NetHack\/} continues this fine tradition.  Unlike text
adventure games that accept commands in pseudo-English sentences and
explain the results in words, {\it NetHack\/} commands are all one or two
keystrokes and the results are displayed graphically on the screen.  A
minimum screen size of 24 lines by 80 columns is recommended; if the
screen is larger, only a $21\times80$ section will be used for the map.

%.pg
{\it NetHack\/} can even be played by blind players, with the assistance of
Braille readers or speech synthesisers.  Instructions for configuring
{\it NetHack\/} for the blind are included later in this document.

%.pg
{\it NetHack\/} generates a new dungeon every time you play it; even the
authors still find it an entertaining and exciting game despite
having won several times.

%.pg
{\it NetHack\/} offers a variety of display options.  The options available to
you will vary from port to port, depending on the capabilities of your
hardware and software, and whether various compile-time options were
enabled when your executable was created.  The three possible display
options are: a monochrome character interface, a color character interface,
and a graphical interface using small pictures called tiles.  The two
character interfaces allow fonts with other characters to be substituted,
but the default assignments use standard ASCII characters to represent
everything.  There is no difference between the various display options
with respect to game play.  Because we cannot reproduce the tiles or
colors in the Guidebook, and because it is common to all ports, we will
use the default ASCII characters from the monochrome character display
when referring to things you might see on the screen during your game.
%.pg
In order to understand what is going on in {\it NetHack}, first you must
understand what {\it NetHack\/} is doing with the screen.  The {\it NetHack\/}
screen replaces the ``You see \ldots'' descriptions of text adventure games.
Figure 1 is a sample of what a {\it NetHack\/} screen might look like.
The way the screen looks for you depends on your platform.
%.BR 2

% (Either generated by hand or else the composite of two different
% situations.  Originally the character had only reached a second room
% (unchanged here) by turn 257 (now changed to 752) and was already
% Weak from hunger (now changed to just Hungry) and also lacked any of
% Tourist's starting gold.  Confusion is added to include a condition.)
%
% Width is forced to match similar figure in Guidebook.mn where it is
% constrained by the margins of plain text output (Guidebook.txt).
\vbox{
\begin{verbatim}
        The bat bites!

                ------
                |....|    ----------
                |.<..|####...@...$.|
                |....-#   |...B....+
                |....|    |.d......|
                ------    -------|--



        Player the Rambler   St:12 Dx:7 Co:18 In:11 Wi:9 Ch:15 Neutral
        Dlvl:1 $:993 HP:9(12) Pw:3(3) AC:10 Exp:1/19 T:752 Hungry Conf
\end{verbatim}
\begin{center}
Figure 1
\end{center}
}

% 3-line status includes trailing spaces to force the width to match the
% 2-line data above; unlike Guidebook.pm, we can't add a trailing comment
% to make them visible
\vbox{
\begin{verbatim}
        Player the Rambler   St:12 Dx:7 Co:18 In:11 Wi:9 Ch:15        
        Neutral $:993 HP:9(12) Pw:3(3) AC:10 Exp:1/19 Hungry          
        Dlvl:1 T:752                                  Conf            
\end{verbatim}
\begin{center}
Figure 2
\end{center}
}

%.hn 2
\subsection*{The status lines (bottom)}

The bottom two (or three) lines of the screen contain several cryptic
pieces of information describing your current status.
Figure 1 shows the traditional two-line status area below the map.
Figure 2 shows just the status area, when the {\it statuslines:3\/}
option has been set (not all interfaces support this option).
If any status line becomes wider than the screen, you might not see all
of it due to truncation.
When the numbers grow bigger and multiple {\it conditions\/} are present,
the two-line format will run out of room on the second line, but
{\it statuslines:2\/}
is the default because a basic 24-line terminal isn't tall enough for
the third line.

%.pg
Here are explanations of what the various status items mean:

%.lp
\blist{}
\item[\bb{Title}]
Your character's name and professional ranking (based on role and
{\it experience level\/}, see below).
%.lp
\item[\bb{Strength}]
A measure of your character's strength; one of your six basic
attributes.  A human character's attributes can range from 3 to 18 inclusive;
non-humans may exceed these limits
(occasionally you may get super-strengths of the form 18/xx, and magic can
also cause attributes to exceed the normal limits).  The
higher your strength, the stronger you are.  Strength affects how
successfully you perform physical tasks, how much damage you do in
combat, and how much loot you can carry.
%.lp
\item[\bb{Dexterity}]
Dexterity affects your chances to hit in combat, to avoid traps, and
do other tasks requiring agility or manipulation of objects.
%.lp
\item[\bb{Constitution}]
Constitution affects your ability to recover from injuries and other
strains on your stamina.
When strength is low or modest, constitution also affects how much you
can carry.  With sufficiently high strength, the contribution to
carrying capacity from your constitution no longer matters.
%.lp
\item[\bb{Intelligence}]
Intelligence affects your ability to cast spells and read spellbooks.
%.lp
\item[\bb{Wisdom}]
Wisdom comes from your practical experience (especially when dealing with
magic).  It affects your magical energy.
%.lp
\item[\bb{Charisma}]
Charisma affects how certain creatures react toward you.  In
particular, it can affect the prices shopkeepers offer you.
%.lp
\item[\bb{Alignment}]
%
{\it Lawful}, {\it Neutral\/} or {\it Chaotic}.  Often, Lawful is
taken as good and Chaotic as evil, but legal and ethical do not always
coincide.  Your alignment influences how other
monsters react toward you.  Monsters of a like alignment are more likely
to be non-aggressive, while those of an opposing alignment are more likely
to be seriously offended at your presence.
%.lp
\item[\bb{Dungeon Level}]
How deep you are in the dungeon.  You start at level one and the number
increases as you go deeper into the dungeon.  Some levels are special,
and are identified by a name and not a number.  The Amulet of Yendor is
reputed to be somewhere beneath the twentieth level.
%.lp
\item[\bb{Gold}]
The number of gold pieces you are openly carrying.  Gold which you have
concealed in containers is not counted.
%.lp
\item[\bb{Hit Points}]
Your current and maximum hit points.  Hit points indicate how much
damage you can take before you die.  The more you get hit in a fight,
the lower they get.  You can regain hit points by resting, or by using
certain magical items or spells.  The number in parentheses is the maximum
number your hit points can reach.
%.lp
\item[\bb{Power}]
Spell points.  This tells you how much mystic energy ({\it mana\/})
you have available for spell casting.  Again, resting will regenerate the
amount available.
%.lp
\item[\bb{Armor Class}]
A measure of how effectively your armor stops blows from unfriendly
creatures.  The lower this number is, the more effective the armor; it
is quite possible to have negative armor class.
See the {\it Armor\/} subsection of {\it Objects\/} for more information.
%.lp
\item[\bb{Experience}]
Your current experience level.
If the {\it showexp\/}
option is set, it will be followed by a slash and experience points.
As you adventure, you gain experience points.
At certain experience point totals, you gain an experience level.
The more experienced you are, the better you fight and withstand magical
attacks.
(By the time your level reaches double digits, the usefulness of showing
the points with it has dropped significantly.
You can use the `{\tt O}' command to turn {\it showexp\/}
off to avoid using up the limited status line space.)
%.lp
\item[\bb{Time}]
The number of turns elapsed so far, displayed if you have the
{\it time\/} option set.
%.lp
\item[\bb{Status}]
Hunger:
your current hunger status.
Values are {\it Satiated}, {\it Not~Hungry\/} (or {\it Normal\/}),
{\it Hungry}, {\it Weak}, and {\it Fainting}.
%.\" not mentioned: Fainted
Not shown when {\it Normal}.

%.lp ""
Encumbrance:
an indication of how what you are carrying affects your ability to move.
Values are {\it Unencumbered}, {\it Encumbered}, {\it Stressed},
{\it Strained}, {\it Overtaxed}, and {\it Overloaded}.
Not shown when {\it Unencumbered}.

%.lp ""
Fatal~conditions:
{\it Stone\/} (aka {\it Petrifying}, turning to stone),
{\it Slime\/} (turning into green slime),
{\it Strngl\/} (being strangled),
{\it FoodPois\/} (suffering from acute food poisoning),
{\it TermIll\/} (suffering from a terminal illness).

%.lp ""
Non-fatal~conditions:
{\it Blind\/} (can't see), {\it Deaf\/} (can't hear),
{\it Stun\/} (stunned), {\it Conf\/} (confused), {\it Hallu\/} (hallucinating).

%.lp ""
Movement~modifiers:
{\it Lev\/} (levitating), {\it Fly\/} (flying), {\it Ride\/} (riding).

%.lp ""
Other conditions and modifiers exist, but there isn't enough room to
display them with the other status fields.
\\
% unindented paragraph
The {\tt \#attributes} command (default key {\tt \^{}X}) will show
all current status information in unabbreviated format.
It also shows other information which might be included on the status
lines if those had more room.

\elist

%.hn 2
\subsection*{The message line (top)}

%.pg
The top line of the screen is reserved for messages that describe
things that are impossible to represent visually.  If you see a
``{\tt --More--}'' on the top line, this means that {\it NetHack\/} has
another message to display on the screen, but it wants to make certain
that you've read the one that is there first.  To read the next message,
just press the space bar.

%.pg
To change how and what messages are shown on the message line,
see ``{\it Configuring Message Types\/}`` and the {\it verbose\/}
option.

%.hn 2
\subsection*{The map (rest of the screen)}

%.pg
The rest of the screen is the map of the level as you have explored it
so far.  Each symbol on the screen represents something.  You can set
various graphics
options to change some of the symbols the game uses; otherwise, the
game will use default symbols.  Here is a list of what the default
symbols mean:

\blist{}
%.lp
\item[\tb{- {\rm and} |}]
The walls of a room, or an open door.  Or a grave ({\tt |}).
%.lp
\item[\tb{.}]
The floor of a room, ice, or a doorless doorway.
%.lp
\item[\tb{\#}]
A corridor, or iron bars, or a tree, or possibly a kitchen sink (if
your dungeon has sinks), or a drawbridge.
%.lp
\item[\tb{>}]
Stairs down: a way to the next level.
%.lp
\item[\tb{<}]
Stairs up: a way to the previous level.
%.lp
\item[\tb{+}]
A closed door, or a spellbook containing a spell you may be able to learn.
%.lp
\item[\tb{@}]
Your character or a human.
%.lp
\item[\tb{\$}]
A pile of gold.
%.lp
\item[\tb{\^}]
A trap (once you have detected it).
%.lp
\item[\tb{)}]
A weapon.
%.lp
\item[\tb{[}]
A suit or piece of armor.
%.lp
\item[\tb{\%}]
Something edible (not necessarily healthy).
%.lp
\item[\tb{?}]
A scroll.
%.lp
\item[\tb{/}]
A wand.
%.lp
\item[\tb{=}]
A ring.
%.lp
\item[\tb{!}]
A potion.
%.lp
\item[\tb{(}]
A useful item (pick-axe, key, lamp \ldots).
%.lp
\item[\tb{"}]
An amulet or a spider web.
%.lp
\item[\tb{*}]
A gem or rock (possibly valuable, possibly worthless).
%.lp
\item[\tb{\`}]
A boulder or statue.
%.lp
\item[\tb{0}]
An iron ball.
%.lp
\item[\tb{\verb+_+}]
An altar, or an iron chain.
%.lp
\item[\tb{\{}]
A fountain.
%.lp
\item[\tb{\}}]
A pool of water or moat or a pool of lava.
%.lp
\item[\tb{$\backslash$}]
An opulent throne.
%.lp
\item[\tb{a-zA-Z {\rm \& other symbols}}]
Letters and certain other symbols represent the various inhabitants
of the Mazes of Menace.  Watch out, they can be nasty and vicious.
Sometimes, however, they can be helpful.
%.lp
\item[\tb{I}]
This marks the last known location of an invisible or otherwise unseen
monster.  Note that the monster could have moved.
The `{\tt F}' and `{\tt m}' commands may be useful here.

\elist
%.pg
You need not memorize all these symbols; you can ask the game what any
symbol represents with the `{\tt /}' command (see the next section for
more info).

%.hn 1
\section{Commands}

%.pg
Commands can be initiated by typing one or two characters to which
the command is bound to, or typing the command name in the extended
commands entry.  Some commands,
like ``{\tt search}'', do not require that any more information be collected
by {\it NetHack\/}.  Other commands might require additional information, for
example a direction, or an object to be used.  For those commands that
require additional information, {\it NetHack\/} will present you with either
a menu of choices, or with a command line prompt requesting information.
Which you are presented with will depend chiefly on how you have set the
`{\it menustyle\/}'
option.

%.pg
For example, a common question in the form ``{\tt What do you want to
use? [a-zA-Z\ ?*]}'', asks you to choose an object you are carrying.
Here, ``{\tt a-zA-Z}'' are the inventory letters of your possible choices.
Typing `{\tt ?}' gives you an inventory list of these items, so you can see
what each letter refers to.  In this example, there is also a `{\tt *}'
indicating that you may choose an object not on the list, if you
wanted to use something unexpected.  Typing a `{\tt *}' lists your entire
inventory, so you can see the inventory letters of every object you're
carrying.  Finally, if you change your mind and decide you don't want
to do this command after all, you can press the `ESC' key to abort the
command.

%.pg
You can put a number before some commands to repeat them that many
times; for example, ``{\tt 10s}'' will search ten times.  If you have the
{\it number\verb+_+pad\/}
option set, you must type `{\tt n}' to prefix a count, so the example above
would be typed ``{\tt n10s}'' instead.  Commands for which counts make no
sense ignore them.  In addition, movement commands can be prefixed for
greater control (see below).  To cancel a count or a prefix, press the
`ESC' key.

%.pg
The list of commands is rather long, but it can be read at any time
during the game through the `{\tt ?}' command, which accesses a menu of
helpful texts.  Here are the default key bindings for your reference:

\blist{}
%.lp
\item[\tb{?}]
Help menu:  display one of several help texts available.
%.lp
\item[\tb{/}]
The {\tt whatis} command, to
tell what a symbol represents.  You may choose to specify a location
or type a symbol (or even a whole word) to explain.
Specifying a location is done by moving the cursor to a particular spot
on the map and then pressing one of `{\tt .}', `{\tt ,}', `{\tt ;}',
or `{\tt :}'.  `{\tt .}' will explain the symbol at the chosen location,
conditionally check for ``{\tt More info?}'' depending upon whether the
`{\it help\/}'
option is on, and then you will be asked to pick another location;
`{\tt ,}' will explain the symbol but skip any additional
information, then let you pick another location;
`{\tt ;}' will skip additional info and also not bother asking
you to choose another location to examine; `{\tt :}' will show additional
info, if any, without asking for confirmation.  When picking a location,
pressing the {\tt ESC} key will terminate this command, or pressing `{\tt ?}'
will give a brief reminder about how it works.

%.lp ""
If the
{\it autodescribe\/}
option is on, a short description of what you see at each location is
shown as you move the cursor.  Typing `{\tt \#}' while picking a location will
toggle that option on or off.
The
{\it whatis\verb+_+coord\/}
option controls whether the short description includes map coordinates.

%.lp ""
Specifying a name rather than a location
always gives any additional information available about that name.

%.lp ""
You may also request a description of nearby monsters,
all monsters currently displayed, nearby objects, or all objects.
The
{\it whatis\verb+_+coord\/}
option controls which format of map coordinate is included with their
descriptions.
%.lp
\item[\tb{\&}]
Tell what a command does.
%.lp
\item[\tb{<}]
Go up to the previous level (if you are on a staircase or ladder).
%.lp
\item[\tb{>}]
Go down to the next level (if you are on a staircase or ladder).
%.lp
\item[\tb{[yuhjklbn]}]
Go one step in the direction indicated (see Figure 3).  If you sense
or remember
a monster there, you will fight the monster instead.  Only these
one-step movement commands cause you to fight monsters; the others
(below) are ``safe.''
%.sd
\begin{center}
\begin{tabular}{cc}
\verb+   y  k  u   + & \verb+   7  8  9   +\\
\verb+    \ | /    + & \verb+    \ | /    +\\
\verb+   h- . -l   + & \verb+   4- . -6   +\\
\verb+    / | \    + & \verb+    / | \    +\\
\verb+   b  j  n   + & \verb+   1  2  3   +\\
                     & (if {\it number\verb+_+pad\/} set)
\end{tabular}
\end{center}
%.ed
\begin{center}
Figure 3
\end{center}
%.lp
\item[\tb{[YUHJKLBN]}]
Go in that direction until you hit a wall or run into something.
%.lp
\item[\tb{m[yuhjklbn]}]
Prefix:  move without picking up objects or fighting (even if you remember
a monster there).\\
%.lp ""
A few non-movement commands use the `{\tt m}' prefix to request
operating via menu (to temporarily override the
{\it menustyle:Traditional\/}
option).
Primarily useful for `{\tt ,}' (pickup) when there is only one class of
objects present (where there won't be any ``what kinds of objects?'' prompt,
so no opportunity to answer `{\tt m}' at that prompt).
\\
%.lp ""
The prefix will
make ``{\tt \#travel}'' command show a menu of interesting targets in sight.
It can also be used with the `{\tt $\backslash$}' (known, show a
list of all discovered objects) and the `{\tt \`{}}' (knownclass,
show a list of discovered objects in a particular class) commands to offer
a menu of several sorting alternatives (which sets a new value for the
{\it sortdiscoveries\/}
option); also for ``{\tt \#vanquished}'' and ``{\tt \#genocided}'' commands
to offer a sorting menu.
\\
%.lp ""
A few other commands (eat food, offer sacrifice, apply tinning-kit,
drink/quaff, dip, tip container) use
the `{\tt m}' prefix to skip checking for applicable objects on
the floor and go straight to checking inventory,
or (for ``{\tt \#loot}'' to remove a saddle),
skip containers and go straight to adjacent monsters.
\\
%.lp ""
In debug mode (aka ``wizard mode''), the `{\tt m}' prefix may also be
used with the ``{\tt \#teleport}'' and ``{\tt \#wizlevelport}'' commands.
%.lp
\item[\tb{F[yuhjklbn]}]
Prefix:  fight a monster (even if you only guess one is there).
%.lp
\item[\tb{g[yuhjklbn]}]
Prefix:  Move until something interesting is found.
%.lp
\item[\tb{G[yuhjklbn] {\rm or} <Control>+[yuhjklbn]}]
Prefix:  Similar to `{\tt g}', but forking of corridors is not considered
interesting.
\\
Note:  {\tt <Control>+<key>} means holding the {\tt <Control>} or
{\tt <Ctrl>} key down like {\tt <Shift>} while typing and releasing
{\tt <key>}, then releasing {\tt <Control>}.  {\tt \^{}<key>} is used as
shorthand elsewhere in the Guidebook to mean the same thing.  Control
characters are case-insensitive so {\tt \^{}x} and {\tt \^{}X} are the same.
%.lp
\item[\tb{M[yuhjklbn]}]
Old versions supported `{\tt M}' as a movement prefix which
combined the effect of `{\tt m}' with {\tt <Control>+<direction>}.
That is no longer supported as a prefix but similar effect can be achieved
by using {\tt m} and {\tt G<direction>} in combination.
{\tt m} can also be used in combination with {\tt g<direction>},
{\tt <Control>+<direction>}, or {\tt <Shift>+<direction>}.
%.lp
\item[\tb{\tt \verb+_+}]
Travel to a map location via a shortest-path algorithm.\\
%.lp ""
The shortest path
is computed over map locations the hero knows about (e.g. seen or
previously traversed).
If there is no known path, a guess is made instead.
Stops on most of
the same conditions as the `{\tt G}' command, but without picking up
objects, so implicitly forces the `{\tt m}' prefix.
For ports with mouse
support, the command is also invoked when a mouse-click takes place on a
location other than the current position.
%.lp
\item[\tb{.}]
Wait or rest, do nothing for one turn.
Precede with the `{\tt m}' prefix
to wait for a turn even next to a hostile monster, if {\it safe\verb+_+wait\/}
is on.
%.lp
\item[\tb{a}]
Apply (use) a tool (pick-axe, key, lamp \ldots).\\
%.lp ""
If used on a wand, that wand will be broken, releasing its magic in the
process.
Confirmation is required.
%.lp
\item[\tb{A}]
Remove one or more worn items, such as armor.\\
%.lp ""
Use `{\tt T}' (take off) to take off only one piece of armor
or `{\tt R}' (remove) to take off only one accessory.
%.lp
\item[\tb{\^{}A}]
Repeat the previous command.
%.lp
\item[\tb{c}]
Close a door.
%.lp
\item[\tb{C}]
Call (name) a monster, an individual object, or a type of object.\\
%.lp ""
Same as extended command ``{\tt \#name}''.
%.lp
\item[\tb{\^{}C}]
Panic button.  Quit the game.
%.lp
\item[\tb{d}]
Drop something.\\
For example {\tt d7a} --- drop seven items of object
{\it a}.
%.lp
\item[\tb{D}]
Drop several things.\\
%.lp ""
In answer to the question\\
``{\tt What kinds of things do you want to drop? [!\%= BUCXPaium]}''\\
you should type zero or more object symbols possibly followed by
`{\tt a}' and/or `{\tt i}' and/or `{\tt u}' and/or `{\tt m}'.
In addition, one or more of
the bless\-ed/\-un\-curs\-ed/\-curs\-ed groups may be typed.\\
%.sd
%.si
{\tt DB}  --- drop all objects known to be blessed.\\
{\tt DU}  --- drop all objects known to be uncursed.\\
{\tt DC}  --- drop all objects known to be cursed.\\
{\tt DX}  --- drop all objects of unknown B/U/C status.\\
{\tt DP}  --- drop objects picked up last.\\
{\tt Da}  --- drop all objects, without asking for confirmation.\\
{\tt Di}  --- examine your inventory before dropping anything.\\
{\tt Du}  --- drop only unpaid objects (when in a shop).\\
{\tt Dm}  --- use a menu to pick which object(s) to drop.\\
{\tt D\%u} --- drop only unpaid food.
%.ei
%.ed
The last example shows a combination.
There are four categories of object filtering: class (`{\tt !}' for
potions, `{\tt ?}' for scrolls, and so on), shop status (`{\tt u}' for
unpaid, in other words, owned by the shop), bless/curse state
(`{\tt B}', `{\tt U}', `{\tt C}', and `{\tt X}' as shown above),
and novelty (`{\tt P}', recently picked up items; controlled by picking
up or dropping things rather than by any time factor).
%.lp ""
\\
If you specify more than one value in a category (such as ``{\tt !?}'' for
potions and scrolls or ``{\tt BU}'' for blessed and uncursed), an inventory
object will meet the criteria if it matches any of the specified
values (so ``{\tt !?}'' means `{\tt !}' or `{\tt ?}').
If you specify more than one category, an inventory object must meet
each of the category criteria (so ``{\tt \%u}'' means class `{\tt \%}' and
unpaid `{\tt u}').
Lastly, you may specify multiple values within multiple categories:
``{\tt !?BU}'' will select all potions and scrolls which are known to be
blessed or uncursed.
(In versions prior to 3.6, filter combinations behaved differently.)
%.lp
\item[\tb{\^{}D}]
Kick something (usually a door).
%.lp
\item[\tb{e}]
Eat food.\\
%.lp ""
Normally checks for edible item(s) on the floor, then if none are found
or none are chosen, checks for edible item(s) in inventory.
Precede `{\tt e}' with the `{\tt m}' prefix to bypass attempting to eat
anything off the floor.\\
%.lp ""
If you attempt to eat while already satiated, you might choke to death.
If you risk it, you will be asked whether
to ``continue eating?'' {\it if you survive the first bite\/}.
You can set the
{\it paranoid\verb+_+confirmation:eating\/}
option to require a response of ``{\tt yes}'' instead of just `{\tt y}'.
%.lp
% Make sure Elbereth is not hyphenated below, the exact spelling matters.
% (Only specified here to parallel Guidebook.mn; use of \tt font implicitly
% prevents automatic hyphenation in TeX and LaTeX.)
\hyphenation{Elbereth}		%override the deduced syllable breaks
\item[\tb{E}]
Engrave a message on the floor.\\
%.sd
%.si
{\tt E-} --- write in the dust with your fingers.\\
%.ei
%.ed
%.lp ""
Engraving the word ``{\tt Elbereth}'' will cause most monsters to not attack
you hand-to-hand (but if you attack, you will rub it out); this is
often useful to give yourself a breather.
%.lp
\item[\tb{f}]
Fire (shoot or throw) one of the objects placed in your quiver (or
quiver sack, or that you have at the ready).
You may select ammunition with a previous `{\tt Q}' command, or let the
computer pick something appropriate if {\it autoquiver\/} is true.
If your wielded weapon has the throw-and-return property, your quiver
is empty, and {\it autoquiver\/}
is false, you will throw that wielded weapon instead of filling the quiver.
This will also automatically use a polearm if wielded.
If {\it fireassist\/} is true, firing will automatically try to wield a launcher
(for example, a bow or a sling) matching the ammo in the quiver; this might
take multiple turns, and get interrupted by a monster.
Remember to swap back to your main melee weapon afterwards.
%.lp ""
\\
See also `{\tt t}' (throw) for more general throwing and shooting.
%.lp
\item[\tb{i}]
List your inventory (everything you're carrying).
%.lp
\item[\tb{I}]
List selected parts of your inventory, usually be specifying the character
for a particular set of objects, like `{\tt [}' for armor or `{\tt !}'
for potions.\\
%.sd
%.si
{\tt I*} --- list all gems in inventory;\\
{\tt Iu} --- list all unpaid items;\\
{\tt Ix} --- list all used up items that are on your shopping bill;\\
{\tt IB} --- list all items known to be blessed;\\
{\tt IU} --- list all items known to be uncursed;\\
{\tt IC} --- list all items known to be cursed;\\
{\tt IX} --- list all items whose bless/curse status is unknown;\\
{\tt IP} --- list items picked up last;\\
{\tt I\$} --- count your money.
%.ei
%.ed
%.lp
\item[\tb{o}]
Open a door.
%.lp
\item[\tb{O}]
Set options.\\
%.lp ""
A menu showing the current option values will be
displayed.  You can change most values simply by selecting the menu
entry for the given option (ie, by typing its letter or clicking upon
it, depending on your user interface).  For the non-boolean choices,
a further menu or prompt will appear once you've closed this menu.
The available options
are listed later in this Guidebook.  Options are usually set before the
game rather than with the `{\tt O}' command; see the section on options below.
Precede {\tt O} with the {\tt m} prefix to show advanced options.
%.lp
\item[\tb{\^{}O}]
Show overview.\\
%.lp ""
Shortcut for ``{\tt \#overview}'':
list interesting dungeon levels visited.\\
%.lp ""
(Prior to 3.6.0, `{\tt \^{}O}' was a debug mode command which listed
the placement of all special levels.
Use ``{\tt \#wizwhere}'' to run that command.)
%.lp
\item[\tb{p}]
Pay your shopping bill.
%.lp
\item[\tb{P}]
Put on an accessory (ring, amulet, or blindfold).\\
%.lp ""
This command may also be used to wear armor.  The prompt for
which inventory item to use will only list accessories, but choosing
an unlisted item of armor will attempt to wear it.
(See the `{\tt W}' command below.  It lists armor as the inventory
choices but will accept an accessory and attempt to put that on.)
%.lp
\item[\tb{\^{}P}]
Repeat previous message.\\
%.lp ""
Subsequent {\tt \^{}P}'s repeat earlier messages.
For some interfaces, the behavior can be varied via the
{\it msg\verb+_+window\/} option.
%.lp
\item[\tb{q}]
Quaff (drink) something (potion, water, etc).\\
%.lp ""
When there is a fountain or sink present, it asks whether to drink
from that.
If that is declined, then it offers a chance to choose a potion from
inventory.
Precede {\tt q} with the {\tt m} prefix to skip asking about
drinking from a fountain or sink.
%.lp
\item[\tb{Q}]
Select an object for your quiver, quiver sack, or just generally at
the ready (only one of these is available at a time).  You can then throw
this (or one of these) using the `{\tt f}' command.
%.lp
\item[\tb{r}]
Read a scroll or spellbook.
%.lp
\item[\tb{R}]
Remove a worn accessory (ring, amulet, or blindfold).\\
%.lp ""
If you're wearing more than one, you'll be prompted for which one to
remove.  When you're only wearing one, then by default it will be removed
without asking, but you can set the
{\it paranoid\verb+_+confirmation\/}
option to require a prompt.\\
%.lp ""
This command may also be used to take off armor.  The prompt for which
inventory item to remove only lists worn accessories, but an item of
worn armor can be chosen.
(See the `{\tt T}' command below.  It lists armor as the inventory
choices but will accept an accessory and attempt to remove it.)
%.lp
\item[\tb{\^{}R}]
Redraw the screen.
%.lp
\item[\tb{s}]
Search for secret doors and traps around you.
It usually takes several tries to find something.
Precede with the `{\tt m}' prefix to wait for a turn
even next to a hostile monster, if {\it safe\verb+_+wait\/}
is on.\\
%.lp ""
Can also be used to figure out whether there is still a monster at
an adjacent ``remembered, unseen monster'' marker.
%.lp
\item[\tb{S}]
Save the game (which suspends play and exits the program).
The saved game will be restored automatically the next time you play
using the same character name.\\
%.lp ""
In normal play, once a saved game is restored the file used to hold
the saved data is deleted.
In explore mode, once restoration is accomplished you are asked whether
to keep or delete the file.
Keeping the file makes it feasible to play for a while then quit
without saving and later restore again.\\
%.lp ""
There is no ``save current game state and keep playing'' command, not
even in explore mode where saved game files can be kept and re-used.
%.lp
\item[\tb{t}]
Throw an object or shoot a projectile.\\
%.lp ""
There's no separate ``shoot'' command.
If you ``throw'' an arrow while wielding a bow, you are shooting
that arrow and any weapon skill bonus or penalty for bow applies.
If you ``throw'' an arrow while not wielding a bow, you are throwing
it by hand and it will generally be less effective than when shot.\\
%.lp ""
See also `{\tt f}' (fire) for throwing or shooting an item pre-selected
via the `{\tt Q}' (quiver) command, with some extra assistance.
%.lp
\item[\tb{T}]
Take off armor.\\
%.lp ""
If you're wearing more than one piece, you'll be prompted for which
one to take off.  (Note that this treats a cloak covering a suit
and/or a shirt, or a suit covering a shirt, as if the underlying items
weren't there.)
When you're only wearing one, then by default it will
be taken off without asking, but you can set the
{\it paranoid\verb+_+confirmation\/}
option to require a prompt.\\
%.lp ""
This command may also be used to remove accessories.  The prompt
for which inventory item to take off only lists worn armor, but a worn
accessory can be chosen.
(See the `{\tt R}' command above.  It lists accessories as the inventory
choices but will accept an item of armor and attempt to take it off.)
%.lp
\item[\tb{\^{}T}]
Teleport, if you have the ability.
%.lp
\item[\tb{v}]
Display version number.
%.lp
\item[\tb{V}]
Display the game history.
%.lp
\item[\tb{w}]
Wield weapon.\\
%.sd
%.si
{\tt w-} --- wield nothing, use your bare (or gloved) hands.\\
%.ei
%.ed
Some characters can wield two weapons at once; use the `{\tt X}' command
(or the ``{\tt \#twoweapon}'' extended command) to do so.
%.lp
\item[\tb{W}]
Wear armor.\\
%.lp ""
This command may also be used to put on an accessory (ring, amulet, or
blindfold).  The prompt for which inventory item to use will only list
armor, but choosing an unlisted accessory will attempt to put it on.
(See the `{\tt P}' command above.  It lists accessories as the inventory
choices but will accept an item of armor and attempt to wear it.)
%.lp
\item[\tb{x}]
Exchange your wielded weapon with the item in your alternate weapon slot.\\
%.lp ""
The latter is used as your secondary weapon when engaging in
two-weapon combat.  Note that if one of these slots is empty,
the exchange still takes place.
%.lp
\item[\tb{X}]
Toggle two-weapon combat, if your character can do it.  Also available
via the ``{\tt \#twoweapon}'' extended command.\\
%.lp ""
(In versions prior to 3.6 this keystroke ran the command to switch from normal
play to ``explore mode'', also known as ``discovery mode'', which has now
been moved to ``{\tt \#exploremode}'' and {\tt M-X}.)
%.lp
\item[\tb{\^{}X}]
Display basic information about your character.\\
%.lp ""
Displays name, role, race, gender (unless role name makes that
redundant, such as {\tt Caveman} or {\tt Priestess}), and alignment,
along with your patron deity and his or her opposition.  It also
shows most of the various items of information from the status line(s)
in a less terse form, including several additional things which don't
appear in the normal status display due to space considerations.\\
%.lp ""
In normal play, that's all that `{\tt \^{}X}' displays.
In explore mode, the role and status feedback is augmented by the
information provided by {\it enlightenment\/} magic.
%.lp
\item[\tb{z}]
Zap a wand.\\
%.sd
%.si
{\tt z.} --- to aim at yourself, use `{\tt .}' for the direction.
%.ei
%.ed
%.lp
\item[\tb{Z}]
Zap (cast) a spell.\\
%.sd
%.si
{\tt Z.} --- to cast at yourself, use `{\tt .}' for the direction.
%.ei
%.ed
%.lp
\item[\tb{\^{}Z}]
Suspend the game (UNIX versions with job control only).
See ``\#suspend'' below for more details.
%.lp
\item[\tb{:}]
Look at what is here.
%.lp
\item[\tb{;}]
Show what type of thing a visible symbol corresponds to.
%.lp
\item[\tb{,}]
Pick up some things from the floor beneath you.\\
%.lp ""
May be preceded by `{\tt m}' to force a selection menu.
%.lp
\item[\tb{@}]
Toggle the {\it autopickup\/} option on and off.
%.lp
\item[\tb{\^{}}]
Ask for the type of an adjacent trap you found earlier.
%.lp
\item[\tb{)}]
Tell what weapon you are wielding.
%.lp
\item[\tb{[}]
Tell what armor you are wearing.
%.lp
\item[\tb{=}]
Tell what rings you are wearing.
%.lp
\item[\tb{"}]
Tell what amulet you are wearing.
%.lp
\item[\tb{(}]
Tell what tools you are using.
%.lp
\item[\tb{*}]
Tell what equipment you are using.\\
%.lp ""
Combines the preceding five type-specific
commands into one.
%.lp
\item[\tb{\$}]
Report the gold you're carrying, possibly shop credit and/or debt too.
%.lp
\item[\tb{+}]
List the spells you know.\\
%.lp ""
Using this command, you can also rearrange
the order in which your spells are listed, either by sorting the entire
list or by picking one spell from the menu then picking another to swap
places with it.  Swapping pairs of spells changes their casting letters,
so the change lasts after the current `{\tt +}' command finishes.  Sorting
the whole list is temporary.  To make the most recent sort order persist
beyond the current `{\tt +}' command, choose the sort option again and then
pick ``reassign casting letters''.  (Any spells learned after that will
be added to the end of the list rather than be inserted into the sorted
ordering.)
%.lp
\item[\tb{$\backslash$}]
Show what types of objects have been discovered.
\\
%.lp ""
May be preceded by `{\tt m}' to select preferred display order.
%.lp
\item[\tb{\`}]
Show discovered types for one class of objects.
\\
.lp ""
May be preceded by `{\tt m}' to select preferred display order.

%.lp
\item[\tb{|}]
If persistent inventory display is supported and enabled (with the
{\it perm\verb+_+invent\/}
option), interact with it instead of with the map.
\\
%.lp ""
Allows scrolling with the
menu\verb+_+first\verb+_+page, menu\verb+_+previous\verb+_+page,
menu\verb+_+next\verb+_+page, and menu\verb+_+last\verb+_+page
keys (`{\tt \^{}}', `{\tt <}', `{\tt >}', `{\tt \verb+|+}' by default).
Some interfaces also support menu\verb+_+shift\verb+_+left and menu\verb+_+shift\verb+_+right
keys (`{\tt \verb+{+}' and `{\tt \verb+}+}' by default).
Use the {\it Return\/} (aka {\it Enter\/}) or {\it Escape\/} key to
resume play.

%.lp
\item[\tb{!}]
Escape to a shell.
See ``\#shell'' below for more details.
%.lp
\item[\tb{Del}]
Show map without obstructions.
You can view the explored portion of the current level's map without
monsters; without monsters and objects; or without monsters, objects,
and traps.\\
%.lp ""
The {\tt <del>} key is also shown as {\tt <delete>} on some keyboards or
{\tt <rubout>} on others.
It is sometimes displayed as {\tt \^{}?} even though that is not an actual
control character.\\
%.lp ""
Many terminals have an option to swap the {\tt <delete>} and {\tt <backspace>}
keys, so typing the {\tt <del>} key might not execute this command.
If that happens, you can use the extended command ``{\tt \#terrain}'' instead.
%.lp
\item[\tb{\#}]
Perform an extended command.\\
%.lp ""
As you can see, the authors of {\it NetHack\/}
used up all the letters, so this is a way to introduce the less frequently
used commands.
What extended commands are available depends on what features
the game was compiled with.
%.lp
\item[\tb{\#adjust}]
Adjust inventory letters (most useful when the
{\it fixinv\/}
option is ``on''). Autocompletes. Default key is `{\tt M-a}'.\\
%.lp ""
This command allows you to move an item from one particular inventory
slot to another so that it has a letter which is more meaningful for you
or that it will appear in a particular location when inventory listings
are displayed.
You can move to a currently empty slot, or if the destination is
occupied---and won't merge---the
item there will swap slots with the one being moved.
``{\tt \#adjust}'' can also be used to split a stack of objects; when
choosing the item to adjust, enter a count prior to its letter.\\
%.lp ""
Adjusting without a count used to collect all compatible stacks when
moving to the destination.  That behavior has been changed; to gather
compatible stacks, ``{\tt \#adjust}'' a stack into its own inventory slot.
If it has a name assigned, other stacks with the same name or with
no name will merge provided that all their other attributes match.
If it does not have a name, only other stacks with no name are eligible.
In either case, otherwise compatible stacks with a different name
will not be merged.  This contrasts with using ``{\tt \#adjust}'' to move
from one slot to a different slot.  In that situation, moving (no
count given) a compatible stack will merge if either stack has a
name when the other doesn't and give that name to the result, while
splitting (count given) will ignore the source stack's name when
deciding whether to merge with the destination stack.
%.lp
\item[\tb{\#annotate}]
Allows you to specify one line of text to associate with the current
dungeon level.  All levels with annotations are displayed by the
``{\tt \#overview}'' command. Autocompletes.
Default key is `{\tt M-A}',
and also `{\tt \^{}N}' if {\it number\verb+_+pad\/} is on.
%.lp
\item[\tb{\#apply}]
Apply (use) a tool such as a pick-axe, a key, or a lamp.
Default key is `{\tt a}'.\\
%.lp ""
If the tool used acts on items on the floor, using the `{\tt m}' prefix
skips those items.\\
%.lp ""
If used on a wand, that wand will be broken, releasing its magic in the
process.  Confirmation is required.
%.lp
\item[\tb{\#attributes}]
Show your attributes. Default key is `{\tt \^{}X}'.
%.lp
\item[\tb{\#autopickup}]
Toggle the {\it autopickup\/} option. Default key is `{\tt @}'.
%.lp
\item[\tb{\#call}]
Call (name) a monster, or an object in inventory, on the floor,
or in the discoveries list, or add an annotation for the
current level (same as ``{\tt \#annotate}''). Default key is `{\tt C}'.
%.lp
\item[\tb{\#cast}]
Cast a spell. Default key is `{\tt Z}'.
%.lp
\item[\tb{\#chat}]
Talk to someone. Default key is `{\tt M-c}'.
%.lp
\item[\tb{\#chronicle}]
Show a list of important game events.
%.lp
\item[\tb{\#close}]
Close a door. Default key is `{\tt c}'.
%.lp
\item[\tb{\#conduct}]
List voluntary challenges you have maintained. Autocompletes.
Default key is `{\tt M-C}'.\\
%.lp ""
See the section below entitled ``Conduct'' for details.
%.lp
\item[\tb{\#debugfuzzer}]
Start the fuzz tester.
Debug mode only.
%.lp
\item[\tb{\#dip}]
Dip an object into something. Autocompletes. Default key is `{\tt M-d}'.\\
%.lp ""
The {\tt m} prefix skips dipping into a fountain or pool if there
is one at your location.
%.lp
\item[\tb{\#down}]
Go down a staircase. Default key is `{\tt >}'.
%.lp
\item[\tb{\#drop}]
Drop an item. Default key is `{\tt d}'.
%.lp
\item[\tb{\#droptype}]
Drop specific item types. Default key is `{\tt D}'.
%.lp
\item[\tb{\#eat}]
Eat something. Default key is `{\tt e}'.
The `{\tt m}' prefix skips eating items on the floor.
%.lp
\item[\tb{\#engrave}]
Engrave writing on the floor. Default key is `{\tt E}'.
%.lp
\item[\tb{\#enhance}]
Advance or check weapon and spell skills. Autocompletes.
Default key is `{\tt M-e}'.
%.lp
\item[\tb{\#exploremode}]
Switch from normal play to non-scoring explore mode.
Default key is `{\tt M-X}'.\\
%.lp ""
Requires confirmation; default response is `{\tt n}' (no).
To really switch to explore mode, respond with `{\tt y}'.
You can set the
{\it paranoid\verb+_+confirmation:quit\/}
option to require a response of ``{\tt yes}'' instead.
%.lp
\item[\tb{\#fight}]
Prefix key to force fight a direction, even if you see nothing
to fight there.
Default key is `{\tt F}', or `{\tt -}' with
{\it number\verb+_+pad\/}
%.lp
\item[\tb{\#fire}]
Fire ammunition from quiver, possibly autowielding a launcher,
or hit with a wielded polearm.
Default key is `{\tt f}'.
%.lp
\item[\tb{\#force}]
Force a lock. Autocompletes. Default key is `{\tt M-f}'.
%.lp
\item[\tb{\#genocided}]
List any monster types which have been genocided.
In explore mode and debug mode it also shows types which have become
extinct.
\\
%.lp ""
The display order is the same as is used by {\\tt \#vanquished}.
The `{\tt m}' prefix brings up a menu of available sorting orders, and
doing that for either {\\tt \#genocided} or {\\tt \#vanquished} changes the order for both.
\\
%.lp ""
If the sorting order is ``count high to low'' or ``count low to high''
(which are applicable for {\tt \#vanquished}), that will be ignored
for {\tt \#genocided} and alphabetical will be used instead.
The menu omits those two choices when used for {\tt \#genocide}.
\\
%.lp ""
Autocompletes.
Default key is `{\tt M-g}'.
%.lp
\item[\tb{\#glance}]
Show what type of thing a map symbol corresponds to. Default key is `{\tt ;}'.
%.lp
\item[\tb{\#help}]
Show the help menu.
Default key is `{\tt ?}',
and also `{\tt h}' if {\it number\verb+_+pad\/} is on.
%.lp
\item[\tb{\#herecmdmenu}]
Show a menu of possible actions directed at your current location.
The menu is limited to a subset of the likeliest actions, not an
exhaustive set of all possibilities.
Autocompletes.\\
%.lp ""
If mouse support is enabled and the {\it herecmd\verb+_+menu\/}
option is On, clicking on the hero (or steed when mounted) will
execute this command.
%.lp
\item[\tb{\#history}]
Show long version and game history. Default key is `{\tt V}'.
%.lp
\item[\tb{\#inventory}]
Show your inventory. Default key is `{\tt i}'.
%.lp
\item[\tb{\#inventtype}]
Inventory specific item types. Default key is `{\tt I}'.
%.lp
\item[\tb{\#invoke}]
Invoke an object's special powers. Autocompletes. Default key is `{\tt M-i}'.
%.lp
\item[\tb{\#jump}]
Jump to another location. Autocompletes.
Default key is `{\tt M-j}',
and also `{\tt j}' if {\it number\verb+_+pad\/} is on.
%.lp
\item[\tb{\#kick}]
Kick something.
Default key is `{\tt \^{}D}',
and also `{\tt k}' if {\it number\verb+_+pad\/} is on.
%.lp
\item[\tb{\#known}]
Show what object types have been discovered.
Default key is `{\tt $\backslash$}'.
\\
%.lp ""
The `{\tt m}' prefix allows assigning a new value to the
{\it sortdiscoveries\/}
option to control the order in which the discoveries are displayed.
%.lp
\item[\tb{\#knownclass}]
Show discovered types for one class of objects.
Default key is `{\tt `}'.
\\
%.lp ""
The `{\tt m}' prefix operates the same as for {\tt \#known}.
%.lp
\item[\tb{\#levelchange}]
Change your experience level.
Autocompletes.
Debug mode only.
%.lp
\item[\tb{\#lightsources}]
Show mobile light sources.
Autocompletes.
Debug mode only.
%.lp
\item[\tb{\#look}]
Look at what is here, under you. Default key is `{\tt :}'.
%.lp
\item[\tb{\#loot}]
Loot a box or bag on the floor beneath you, or the saddle
from a steed standing next to you. Autocompletes.
Precede with the `{\tt m}' prefix to skip containers at your location
and go directly to removing a saddle.
Default key is `{\tt M-l}',
and also `{\tt l}' if {\it number\verb+_+pad\/} is on.
%.lp
\item[\tb{\#monster}]
Use a monster's special ability (when polymorphed into monster form).
Autocompletes. Default key is `{\tt M-m}'.
%.lp
\item[\tb{\#name}]
Name a monster, an individual object, or a type of object.
Same as ``{\tt \#call}''.
Autocompletes.
Default keys are `{\tt N}', `{\tt M-n}', and `{\tt M-N}'.
%.lp
\item[\tb{\#offer}]
Offer a sacrifice to the gods. Autocompletes. Default key is `{\tt M-o}'.\\
%.lp ""
You'll need to find an altar to have any chance at success.
Corpses of recently killed monsters are the fodder of choice.
%.lp ""
The `{\tt m}' prefix skips offering any items which are on the altar.\\
%.lp
\item[\tb{\#open}]
Open a door. Default key is `{\tt o}'.
%.lp
\item[\tb{\#options}]
Show and change option settings. Default key is `{\tt O}'.
Precede with the {\tt m} prefix to show advanced options.
%.lp
\item[\tb{\#optionsfull}]
Show advanced game option settings.
No default key.
Precede with the `{\tt m}' prefix to execute the simpler options command.
(Mainly useful if you use {\tt BINDING=O:optionsfull} to switch
`{\tt O}' from simple options back to traditional advanced options.)
%.lp
\item[\tb{\#overview}]
Display information you've discovered about the dungeon.
Any visited level
% [note: amnesia no longer causes levels to be forgotten so exclude this]
% (unless forgotten due to amnesia)
with an annotation is included,
and many things (altars, thrones, fountains, and so on; extra stairs
leading to another dungeon branch) trigger an automatic annotation.
If dungeon overview is chosen during end-of-game disclosure, every visited
level will be included regardless of annotations.
\\
%.lp ""
Precede \#overview with the `{\tt m}' prefix to display the dungeon
overview as a menu where you can select any visited level to add or
remove an annotation without needing to return to that level.
This will also force all visited levels to be displayed rather than just
the ``interesting'' subset.
\\
%.lp ""
Autocompletes.
Default keys are `{\tt \^{}O}', and `{\tt M-O}'.
% DON'T PANIC!
%.lp
\item[\tb{\#panic}]
Test the panic routine.
Terminates the current game.
Autocompletes.
Debug mode only.\\
%.lp ""
Asks for confirmation; default is `{\tt n}' (no); continue playing.
To really panic, respond with `{\tt y}'.
You can set the
{\it paranoid\verb+_+confirmation:quit\/}
option to require a response of ``{\tt yes}'' instead.
%.lp
\item[\tb{\#pay}]
Pay your shopping bill. Default key is `{\tt p}'.
%.lp
\item[\tb{\#perminv}]
If persistent inventory display is supported and enabled (with the
{\it perm\verb+_+invent\/} option), interact with it instead of with the map.
You'll be prompted for menu scrolling keystrokes such
as `{\tt \verb+>+}' and `{\tt \verb+<+}'.
Press {\tt Return} or {\tt Escape} to resume normal play.
Default key is {\tt \verb+|+}.
%.lp
\item[\tb{\#pickup}]
Pick up things at the current location. Default key is `{\tt ,}'.
The `{\tt m}' prefix forces use of a menu.
%.lp
\item[\tb{\#polyself}]
Polymorph self.
Autocompletes.
Debug mode only.
%.lp
\item[\tb{\#pray}]
Pray to the gods for help. Autocompletes. Default key is `{\tt M-p}'.\\
%.lp ""
Praying too soon after receiving prior help is a bad idea.
(Hint: entering the dungeon alive is treated as having received help.
You probably shouldn't start off a new game by praying right away.)
Since using this command by accident can cause trouble, there is an
option to make you confirm your intent before praying.  It is enabled
by default, and you can reset the
{\it paranoid\verb+_+confirmation\/}
option to disable it.
%.lp
\item[\tb{\#prevmsg}]
Show previously displayed game messages. Default key is `{\tt \^{}P}'.
%.lp
\item[\tb{\#puton}]
Put on an accessory (ring, amulet, etc). Default key is `{\tt P}'.
%.lp
\item[\tb{\#quaff}]
Quaff (drink) something. Default key is `{\tt q}'.\\
%.lp ""
The {\tt m} prefix skips drinking from a fountain or sink if there
is one at your location.
%.lp
\item[\tb{\#quit}]
Quit the program without saving your game. Autocompletes.\\
%.lp ""
Since using this command by accident would throw away the current game,
you are asked to confirm your intent before quitting.
Default response is `{\tt n}' (no); continue playing.
To really quit, respond with `{\tt y}'.
You can set the
{\it paranoid\verb+_+confirmation:quit\/}
option to require a response of ``{\tt yes}'' instead.
%.lp
\item[\tb{\#quiver}]
Select ammunition for quiver. Default key is `{\tt Q}'.
%.lp
\item[\tb{\#read}]
Read a scroll, a spellbook, or something else. Default key is `{\tt r}'.
%.lp
\item[\tb{\#redraw}]
Redraw the screen.
Default key is `{\tt \^{}R}',
and also `{\tt \^{}L}' if {\it number\verb+_+pad\/} is on.
%.lp
\item[\tb{\#remove}]
Remove an accessory (ring, amulet, etc). Default key is `{\tt R}'.
%.lp
\item[{\tb{\#repeat}}]
Repeat the previous command.
Default key is~`{\tt \^{}A}'.
%.lp
\item[\tb{\#reqmenu}]
Prefix key to modify the behavior or request menu from some commands.
Prevents autopickup when used with movement commands.
Default key is `{\tt m}'.
%.lp
\item[\tb{\#retravel}]
Travel to a previously selected travel destination.
Default key is `{\tt C-\verb+_+}'.
See also {\tt \#travel}.
%.lp
\item[\tb{\#ride}]
Ride (or stop riding) a saddled creature. Autocompletes.
Default key is `{\tt M-R}'.
%.lp
\item[\tb{\#rub}]
Rub a lamp or a stone. Autocompletes. Default key is `{\tt M-r}'.
%.lp
\item[\tb{\#run}]
Prefix key to run towards a direction.
Default key is `{\tt G}' when
{\it number\verb+_+pad\/}
is off,
`{\tt 5}' when
{\it number\verb+_+pad\/}
is set to 1~or~3,
otherwise `{\tt M-5}' when it is set to 2~or~4.
%.lp
\item[\tb{\#rush}]
Prefix key to rush towards a direction.
Default key is `{\tt g}' when
{\it number\verb+_+pad\/}
is off,
`{\tt M-5}' when
{\it number\verb+_+pad\/}
is set to 1~or~3,
otherwise `{\tt 5}' when it is set to 2~or~4.
%.lp
\item[\tb{\#save}]
Save the game and exit the program.
Default key is `{\tt S}'.
%.lp
\item[\tb{\#saveoptions}]
Save configuration options to the config file.
This will overwrite the file, removing all comments, so if you have
manually edited the config file, don't use this.
%.lp
\item[\tb{\#search}]
Search for traps and secret doors around you. Default key is `{\tt s}'.
%.lp
\item[\tb{\#seeall}]
Show all equipment in use. Default key is `{\tt *}'.
%.lp
\item[\tb{\#seeamulet}]
Show the amulet currently worn. Default key is `{\tt "}'.
%.lp
\item[\tb{\#seearmor}]
Show the armor currently worn. Default key is `{\tt [}'.
%.lp
\item[\tb{\#seerings}]
Show the ring(s) currently worn. Default key is `{\tt =}'.
%.lp
\item[\tb{\#seetools}]
Show the tools currently in use. Default key is `{\tt (}'.
%.lp
\item[\tb{\#seeweapon}]
Show the weapon currently wielded. Default key is `{\tt )}'.
%.lp
\item[\tb{\#shell}]
Do a shell escape, switching from NetHack to a subprocess.
Can be disabled at the time the program is built.
When enabled, access for specific users can be controlled by the system
configuration file.
Use the shell command `{\tt exit}' to return to the game.
Default key is `{\tt !}'.
%.lp
\item[\tb{\#showgold}]
Report the gold in your inventory, including gold you know about in
containers you're carrying.  If you are inside a shop, report any credit
or debt you have in that shop.
Default key is `{\tt \$}'.
%.lp
\item[\tb{\#showspells}]
List and reorder known spells.
Default key is `{\tt +}'.
%.lp
\item[\tb{\#showtrap}]
Describe an adjacent trap, possibly covered by objects or a monster.
To be eligible, the trap must already be discovered.
(The ``{\tt \#terrain}'' command can display your map with all objects and
monsters temporarily removed, making it possible to see all discovered
traps.)
Default key is `{\tt \^{}}'.
%.lp
\item[\tb{\#sit}]
Sit down. Autocompletes. Default key is `{\tt M-s}'.
%.lp
\item[\tb{\#stats}]
Show memory usage statistics.
Autocompletes.
Debug mode only.
%.lp
\item[\tb{\#suspend}]
Suspend the game, switching from NetHack to the terminal it was started
from without performing save-and-exit.
Can be disabled at the time the program is built.
When enabled, mainly useful for {\it tty\/} and {\it curses\/} interfaces on
%.UX \. \" yields "UNIX."
UNIX.
Use the shell command `{\tt fg}' to return to the game.
Default key is `{\tt \^{}Z}'.
%.lp
\item[\tb{\#swap}]
Swap wielded and secondary weapons. Default key is `{\tt x}'.
%.lp
\item[\tb{\#takeoff}]
Take off one piece of armor. Default key is `{\tt T}'.
%.lp
\item[\tb{\#takeoffall}]
Remove all armor. Default key is `{\tt A}'.
%.lp
\item[\tb{\#teleport}]
Teleport around the level. Default key is `{\tt \^{}T}'.
%.lp
\item[\tb{\#terrain}]
Show map without obstructions.
In normal play you can view the explored portion of the current level's
map without monsters; without monsters and objects; or without monsters,
objects, and traps.\\
%.lp ""
In explore mode, you can choose to view the full map rather than just
its explored portion.
In debug mode there are additional choices.\\
%.lp ""
Autocompletes.
Default key is `{\tt <del>}' or `{\tt <delete>}' (see {\it Del\/} above).
%.lp
\item[\tb{\#therecmdmenu}]
Show a menu of possible actions directed at a location next to you.
The menu is limited to a subset of the likeliest actions, not an
exhaustive set of all possibilities.
Autocompletes.
%%--invoking it by mouse seems to be broken
%% \\
%% .lp ""
%% If mouse support is enabled and the {\it herecmd\verb+_+menu\/}
%% option is On, clicking on an adjacent location will execute this command.
%.lp
\item[\tb{\#throw}]
Throw something. Default key is `{\tt t}'.
%.lp
\item[\tb{\#timeout}]
Look at the timeout queue.
Autocompletes.
Debug mode only.
%.lp
\item[\tb{\#tip}]
Tip over a container (bag or box) to pour out its contents.
When there are containers on the floor, the game will prompt to pick one
of them or ``tip something being carried''.
\\
%.lp ""
If the latter is chosen, there will be another prompt for which item
from inventory to tip.
The `{\tt m}' prefix makes the command skip containers on the
floor and pick one from inventory, except for the special case of
{\it menustyle:Traditional\/}
with two or more containers present; that situation will start with the
floor container menu.
\\
%.lp ""
Autocompletes. Default key is `{\tt M-T}'.
%.lp
\item[\tb{\#travel}]
Travel to a specific location on the map.
Default key is `{\tt \verb+_+}'.
Using the ``request menu'' prefix shows a menu of interesting targets in sight
without asking to move the cursor.
When picking a target with cursor and the {\it autodescribe\/}
option is on, the top line will show ``(no travel path)'' if
your character does not know of a path to that location.
See also {\tt \#retravel}.
%.lp
\item[\tb{\#turn}]
Turn undead away. Autocompletes. Default key is `{\tt M-t}'.
%.lp
\item[\tb{\#twoweapon}]
Toggle two-weapon combat on or off. Autocompletes.
Default key is `{\tt X}',
and also `{\tt M-2}' if {\it number\verb+_+pad\/} is off.\\
%.lp ""
Note that you must
use suitable weapons for this type of combat, or it will
be automatically turned off.
%.lp
\item[\tb{\#untrap}]
Untrap something (trap, door, or chest).
Default key is `{\tt M-u}', and `{\tt u}' if {\it number\verb+_+pad\/} is on.\\
%.lp ""
In some circumstances it can also be used to rescue trapped monsters.
%.lp
\item[\tb{\#up}]
Go up a staircase. Default key is `{\tt <}'.
%.lp
\item[\tb{\#vanquished}]
List vanquished monsters by type and count.
\\
%.lp ""
Note that the vanquished monsters list includes all monsters killed by
traps and each other as well as by you, and omits any which got removed
from the game without being killed (perhaps by genocide, or by a mollified
shopkeeper dismissing summoned Kops) or were already corpses when placed
on the map.
\\
%.lp ""
Using the ``request menu'' prefix prior to \#vanquished brings up
a menu of sorting orders available (provided that the vanquished monsters
list contains at least two types of monsters).
Whichever ordering is picked gets assigned to the {\it sortvanquished}
option so is remembered for subsequent \#vanquished requests.
The {\tt \#genocided} command shares this sorting order.
\\
%.lp ""
During end-of-game disclosure, when asked whether to show vanquished
monsters answering `{\tt a}' will let you choose from the sort menu.
\\
%.lp ""
Autocompletes.
Default key is `{\tt M-V}'.
%.lp
\item[\tb{\#version}]
Print compile time options for this version of {\it NetHack\/}.

%.lp
The second paragraph lists the user interface(s) that are included.
If there are more than one, you can use the {\it windowtype\/}
option in your run-time configuration file to select the one you want.

%.lp
Autocompletes. Default key is `{\tt M-v}'.
%.lp
\item[\tb{\#versionshort}]
Show the program's version number, plus the date and time that the
running copy was built from sources (not the version's release date).
Default key is `{\tt v}'.
%.lp
\item[\tb{\#vision}]
Show vision array.
Autocompletes.
Debug mode only.
%.lp
\item[\tb{\#wait}]
Rest one move while doing nothing.
Default key is `{\tt .}', and also `{\tt{ }}' if
{\it rest\verb+_+on\verb+_+space\/} is on.
%.lp
\item[\tb{\#wear}]
Wear a piece of armor. Default key is `{\tt W}'.
%.lp
\item[\tb{\#whatdoes}]
Tell what a key does. Default key is `{\tt \&}'.
%.lp
\item[\tb{\#whatis}]
Show what type of thing a symbol corresponds to. Default key is `{\tt /}'.
%.lp
\item[\tb{\#wield}]
Wield a weapon. Default key is `{\tt w}'.
%.lp
\item[\tb{\#wipe}]
Wipe off your face. Autocompletes. Default key is `{\tt M-w}'.
%.lp
\item[\tb{\#wizborn}]
Show monster birth, death, genocide, and extinct statistics.
Debug mode only.
%.lp
\item[\tb{\#wizbury}]
Bury objects under and around you.
Autocompletes.
Debug mode only.
%.lp
\item[\tb{\#wizcast}]
Cast any spell.
Debug mode only.
%.lp
\item[\tb{\#wizdetect}]
Reveal hidden things (secret doors or traps or unseen monsters)
within a modest radius.
No time elapses.
Autocompletes.
Debug mode only.
Default key is `{\tt \^{}E}'.
%.lp
\item[\tb{\#wizgenesis}]
Create a monster.
May be prefixed by a count to create more than one.
Autocompletes.
Debug mode only.
Default key is `{\tt \^{}G}'.
%.lp
\item[\tb{\#wizidentify}]
Identify all items in inventory.
Autocompletes.
Debug mode only.
Default key is `{\tt \^{}I}'.
%.lp
\item[\tb{\#wizintrinsic}]
Set one or more intrinsic attributes.
Autocompletes.
Debug mode only.
%.lp
\item[\tb{\#wizkill}]
Remove monsters from play by just pointing at them.
By default the hero gets credit or blame for killing the targets.
Precede this command with the `{\tt m}' prefix to override that.
Autocompletes.
Debug mode only.
%.lp
\item[\tb{\#wizlevelport}]
Teleport to another level.
Autocompletes.
Debug mode only.
Default key is `{\tt \^{}V}'.
%.lp
\item[\tb{\#wizmap}]
Map the level.
Autocompletes.
Debug mode only.
Default key is `{\tt \^{}F}'.
%.lp
\item[\tb{\#wizrumorcheck}]
Verify rumor boundaries by displaying first and last true rumors and
first and last false rumors.\\
%.lp ""
Also displays first, second, and last random engravings, epitaphs,
and hallucinatory monsters.\\
%.lp ""
Autocompletes.
Debug mode only.
%.lp
\item[\tb{\#wizseenv}]
Show map locations' seen vectors.
Autocompletes.
Debug mode only.
%.lp
\item[\tb{\#wizsmell}]
Smell monster.
Autocompletes.
Debug mode only.
%.lp
\item[\tb{\#wizwhere}]
Show locations of special levels.
Autocompletes.
Debug mode only.
%.lp
\item[\tb{\#wizwish}]
Wish for something.
Autocompletes.
Debug mode only.
Default key is `{\tt \^{}W}'.
%.lp
\item[\tb{\#wmode}]
Show wall modes.
Autocompletes.
Debug mode only.
%.lp
\item[\tb{\#zap}]
Zap a wand. Default key is `{\tt z}'.
%.lp
\item[\tb{\#?}]
Help menu:  get the list of available extended commands.
\elist

%.pg
\nd If your keyboard has a meta key (which, when pressed in combination
with another key, modifies it by setting the `meta' [8th, or `high']
bit), you can invoke many extended commands by meta-ing the first
letter of the command.

On {\it Windows\/} and {\it MS-DOS\/},
the `Alt' key can be used in this fashion.
On other systems, if typing `Alt' plus another key transmits a
two character sequence consisting of an {\tt Escape}
followed by the other key, you may set the {\it altmeta\/}
option to have {\it NetHack\/} combine them into {\tt meta+<key>}.
(This combining action only takes place when NetHack is expecting a
command to execute, not when accepting input to name something or to
make a wish.)

%.pg
Unlike control characters, where {\tt \^{}x} and {\tt \^{}X} denote the same
thing, meta characters are case-sensitive:  {\tt M-x} and {\tt M-X}
represent different things.  Some commands which can be run via a meta
character require that the letter be capitalized because the lower-case
equivalent is used for another command, so the three key combination
{\tt meta+Shift+letter} is needed.

%.BR 1
\blist{}
%.lp
\item[\tb{M-?}]
{\tt\#?} (not supported by all platforms)
%.lp
\item[\tb{M-2}]
{\tt\#twoweapon} (unless the {\it number\verb+_+pad\/} option is enabled)
%.lp
\item[\tb{M-a}]
{\tt\#adjust}
%.lp
\item[\tb{M-A}]
{\tt\#annotate}
%.lp
\item[\tb{M-c}]
{\tt\#chat}
%.lp
\item[\tb{M-C}]
{\tt\#conduct}
%.lp
\item[\tb{M-d}]
{\tt\#dip}
%.lp
\item[\tb{M-e}]
{\tt\#enhance}
%.lp
\item[\tb{M-f}]
{\tt\#force}
%.lp
\item[\tb{M-g}]
{\tt\#genocided}
%.lp
\item[\tb{M-i}]
{\tt\#invoke}
%.lp
\item[\tb{M-j}]
{\tt\#jump}
%.lp
\item[\tb{M-l}]
{\tt\#loot}
%.lp
\item[\tb{M-m}]
{\tt\#monster}
%.lp
\item[\tb{M-n}]
{\tt\#name}
%.lp
\item[\tb{M-o}]
{\tt\#offer}
%.lp
\item[\tb{M-O}]
{\tt\#overview}
%.lp
\item[\tb{M-p}]
{\tt\#pray}
%.Ip
\item[\tb{M-r}]
{\tt\#rub}
%.lp
\item[\tb{M-R}]
{\tt\#ride}
%.lp
\item[\tb{M-s}]
{\tt\#sit}
%.lp
\item[\tb{M-t}]
{\tt\#turn}
%.lp
\item[\tb{M-T}]
{\tt\#tip}
%.lp
\item[\tb{M-u}]
{\tt\#untrap}
%.lp
\item[\tb{M-v}]
{\tt\#version}
%.lp
\item[\tb{M-V}]
{\tt\#vanquished}
%.lp
\item[\tb{M-w}]
{\tt\#wipe}
%.lp
\item[\tb{M-X}]
{\tt\#exploremode}
\elist

%.pg
\nd If the {\it number\verb+_+pad\/} option is on, some additional letter commands
are available:
\blist{}
%.lp
\item[\tb{h}]
{\tt\#help}
%.lp
\item[\tb{j}]
{\tt\#jump}
%.lp
\item[\tb{k}]
{\tt\#kick}
%.lp
\item[\tb{l}]
{\tt\#loot}
%.lp
\item[\tb{N}]
{\tt\#name}
%.lp
\item[\tb{u}]
{\tt\#untrap}
\elist

%.BR 1 \"blank line for extra separation; plain text output looks better

%.hn 1
\section{Rooms and corridors}

%.pg
Rooms and corridors in the dungeon are either lit or dark.
Any lit areas within your line of sight will be displayed;
dark areas are only displayed if they are within one space of you.
Walls and corridors remain on the map as you explore them.

%.pg
Secret corridors are hidden and appear to be solid rock.
You can find them with the `{\tt s}' (search) command when adjacent
to them.
Multiple search attempts may be needed.
When searching is successful, secret corridors become ordinary open
corridor locations.
Mapping magic reveals secret corridors, so converts them into ordinary
corridors and shows them as such.

%.hn 2
\subsection*{Doorways}

%.pg
Doorways connect rooms and corridors.
Some doorways have no doors; you can walk right through.
Others have doors in them, which may be open, closed, or locked.
To open a closed door, use the `{\tt o}' (open)
command; to close it again, use the `{\tt c}' (close) command.
By default the
{\it autoopen}
option is enabled, so simply attempting to walk onto a closed door's
location will attempt to open it without needing `{\tt o}'.
Opening via
{\it autoopen}
will not work if you are {\it confused\/} or {\it stunned\/} or suffer from
the {\it fumbling\/} attribute.

%.pg
Open doors cannot be entered diagonally; you must approach them
straight on, horizontally or vertically.
Doorways without doors are
not restricted in this fashion except on one particular level
%.\" the rogue level
(described by ``{\tt \#overview}'' as ``a primitive area'').

%.pg
Unlocking magic exists but usually won't be available early on.
You can get through a locked door without magic by first using an
unlocking tool with the `{\tt a}' (apply) command, and then opening it.
By default the
{\it autounlock}
option is also enabled, so if you attempt to open (via `{\tt o}' or
{\it autoopen})
a locked door while carrying an unlocking tool, you'll be asked whether
to use it on the door's lock.
Alternatively, you can break a closed door (whether locked or not) down
by kicking it via the ``{\tt \^{}D}'' (kick) command.
Kicking down a door destroys it and makes a lot of noise which might
wake sleeping monsters.

%.pg
Some closed doors are booby-trapped and will explode if an attempt is made
to open (when unlocked) or unlock (when locked) or kick down.
Like kicking, an explosion destroys the door and makes a lot of noise.
The ``{\tt \#untrap}'' command can be used to search a door for traps but
might take multiple attempts to find one.
When one is found, you'll be asked whether to try to disarm it.
If you accede, success will eliminate the trap but
failure will set off the trap's explosion.
(If you decline, you effectively forget that a trap was found there.)

%.pg
Closed doors can be useful for shutting out monsters.
Most monsters cannot open closed doors, although a few don't need to
(for example, ghosts can walk through doors and fog clouds can flow
under them).
Some monsters who can open doors can also use unlocking tools.
And some (giants) can smash doors.

%.pg
Secret doors are hidden and appear to be ordinary wall (from inside a
room) or solid rock (from outside).
You can find them with the `{\tt s}' (search) command but it might
take multiple tries (possibly many tries if your luck is poor).
Once found they are in all ways equivalent to normal doors.
Mapping magic does not reveal secret doors.

%.hn 2
\subsection*{Traps (`{\tt \^{}}')}

%.pg
There are traps throughout the dungeon to snare the unwary intruder.
For example, you may suddenly fall into a pit and be stuck for a few
turns trying to climb out (see below).
A trap usually won't appear on your map until you trigger it by moving
onto it, you see someone else trigger it, or you discover it with
the `{\tt s}' (search) command (multiple attempts are often needed;
if your luck is poor, many attempts might be needed).
{\it Wands of secret door detection\/} and the spell of {\it detect unseen}
also reveal traps within a modest radius but only if the trap is also within
line-of-sight (whether you can see at the time or not).
There is also other magic which can reveal traps.

%.pg
Monsters can fall prey to
traps, too, which can potentially be used as a defensive strategy.
Unfortunately traps can be harmful to your pet(s) as well.
Monsters, including pets, usually will avoid moving onto a trap which
is shown on your map if they have encountered that type of trap before.

%.pg
Some traps such as pits, bear traps, and webs hold you in one place.
You can escape by simply trying to move to an adjacent spot and repeat
as needed; eventually you will get free.

%.pg
Other traps can send you to different locations.
Teleporters send you elsewhere on the same dungeon level.
Level teleporters send you to a random dungeon level, the destination
chosen from a few levels lower all the way to the top.
These traps choose a new destination each time they're activated.
Trap doors and holes also send you to another level, but one which is
always below the current level.
Usually that will be the next level down but it can be farther.
Unlike (level) teleporters, the destination level of a particular trap door
or hole is persistent, so falling into one will bring you to the same level
each time---though not necessarily the same spot on the level.
Magic portals behave similarly, but with some additional variation.
Some portals are two-way and their remote destination is always the same:
another portal which can take you back.
Others are one-way and send you to a specific destination level but not
necessarily to a specific location there.

%.pg
There is a special multi-level branch of the dungeon with pre-mapped levels
based on the classic computer game ``{\it Sokoban}.''
In that game, you operate as a warehouse worker who pushes crates around
obstacles to position them at designated locations.
In NetHack, the goal is to push boulders into pits or holes until those
traps have all been nullified, giving access to whatever is beyond them.
In the Sokoban game, you can only move in the four cardinal compass
directions, and a crate in its final destination blocks further access
to that spot.
In the Sokoban levels of NetHack, you can move diagonally (unless that
would let you pass between two neighboring boulders) but you can only
push boulders in the four cardinal directions, and a boulder which fills
a pit or hole removes both the boulder and the trap so opens up normal
access to that spot.
With careful foresight, it is possible to complete all of the levels
according to the traditional rules of Sokoban.
(Hint: to solve Sokoban puzzles, you often need to move things away from
their eventual destinations in order to open up more room to maneuver.)
Since NetHack does not support an {\it undo\/} capability, some allowances
are permitted in case you get stuck.
For example, each level has at least one extra boulder.
Also, it is possible to drop everything in order to be able to squeeze
into the same location as a boulder (and then presumably move past it),
or to destroy a boulder with magic or tools, or to create new boulders
with a {\it scroll of earth}.
However, doing such things will lower your luck without any specific
message given about that.
See the {\it Conduct\/} section for information about getting feedback for
your actions in Sokoban.

%.hn 2
\subsection*{Stairs and ladders (`{\tt <}', `{\tt >}')}

%.pg
In general, each level in the dungeon will have a staircase going up
(`{\tt <}') to the previous level and another going down (`{\tt >}')
to the next
level.  There are some exceptions though.  For instance, fairly early
in the dungeon you will find a level with two down staircases, one
continuing into the dungeon and the other branching into an area
known as the Gnomish Mines.  Those mines eventually hit a dead end,
so after exploring them (if you choose to do so), you'll need to
climb back up to the main dungeon.

%.pg
When you traverse a set of stairs, or trigger a trap which sends you
to another level, the level you're leaving will be deactivated and
stored in a file on disk.  If you're moving to a previously visited
level, it will be loaded from its file on disk and reactivated.  If
you're moving to a level which has not yet been visited, it will be
created (from scratch for most random levels, from a template for
some ``special'' levels, or loaded from the remains of an earlier game
for a ``bones'' level as briefly described below).  Monsters are only
active on the current level; those on other levels are essentially
placed into stasis.

%.pg
Ordinarily when you climb a set of stairs, you will arrive on the
corresponding staircase at your destination.  However, pets (see below)
and some other monsters will follow along if they're close enough when
you travel up or down stairs, and occasionally one of these creatures
will displace you during the climb.  When that occurs, the pet or other
monster will arrive on the staircase and you will end up nearby.

%.pg
Ladders serve the same purpose as staircases, and the two types of
inter-level connections are nearly indistinguishable during game play.

%.hn 2
\subsection*{Shops and shopping}

%.pg
Occasionally you will run across a room with a shopkeeper near the door
and many items lying on the floor.  You can buy items by picking them
up and then using the `{\tt p}' command.  You can inquire about the price
of an item prior to picking it up by using the ``{\tt \#chat}'' command
while standing on it.  Using an item prior to paying for it will incur a
charge, and the shopkeeper won't allow you to leave the shop until you
have paid any debt you owe.

%.pg
You can sell items to a shopkeeper by dropping them to the floor while
inside a shop.  You will either be offered an amount of gold and asked
whether you're willing to sell, or you'll be told that the shopkeeper
isn't interested (generally, your item needs to be compatible with the
type of merchandise carried by the shop).

%.pg
If you drop something in a shop by accident, the shopkeeper will usually
claim ownership without offering any compensation.  You'll have to buy
it back if you want to reclaim it.

%.pg
Shopkeepers sometime run out of money.
When that happens, you'll be
offered credit instead of gold when you try to sell something.
Credit can be used to pay for purchases, but it is only good in the shop
where it was obtained; other shopkeepers won't honor it.
(If you happen to
find a ``credit card'' in the dungeon, don't bother trying to use it in
shops; shopkeepers will not accept it.)

%.pg
The {\tt \$} command, which reports the amount of gold you are carrying,
will also show current shop debt or credit, if any.
The {\tt Iu} command lists unpaid items (those which still belong to the
shop) if you are carrying any.
The {\tt Ix} command shows an inventory-like display of any unpaid items
which have been used up, along with other shop fees, if any.

%.hn 3
\subsubsection*{Shop idiosyncrasies}

%.pg
Several aspects of shop behavior might be unexpected.

\begin{itemize}
% note: a bullet is the default item label so we could omit [$\bullet$] here
%.lp \(bu 2
\item[$\bullet$]
The price of a given item can vary due to a variety of factors.
%.lp \(bu 2
\item[$\bullet$]
A shopkeeper treats the spot immediately inside the door as if it were
outside the shop.
%.lp \(bu 2
\item[$\bullet$]
While the shopkeeper watches you like a hawk, he or she will generally ignore
any other customers.
%.lp \(bu 2
\item[$\bullet$]
If a shop is ``closed for inventory,'' it will not open of its own accord.
%.lp \(bu 2
\item[$\bullet$]
Shops do not get restocked with new items, regardless of inventory depletion.
\end{itemize}

%.hn 2
\subsection*{Movement feedback}

%.pg
Moving around the map usually provides no
feedback---other than drawing the hero at the new location---unless
you step on an object or pile of objects,
or on a trap, or attempt to move onto a spot where a monster is located.
There are several options which can be used to augment the normal feedback.

%.pg
The
{\it pile\verb+_+limit\/}
option controls how many objects can be in a
pile---sharing the same map location---for
the game to state ``there are objects here'' instead of listing them.
The default is {\tt 5}.
Setting it to {\tt 1} would always give that message instead of listing
any objects.
Setting it to {\tt 0} is a special case which will always list all
objects no matter how big a pile is.
Note that the number refers to the count of separate stacks of objects
present rather than the sum of the quantities of those stacks (so
{\tt 7 arrows} or {\tt 25 gold pieces} will each count as 1 rather
than as 7 and 25, respectively, and total to 2 when both are at the
same location).

%.pg
The {\tt nopickup} command prefix (default `{\tt m}') can be
used before a movement direction to step on objects without attempting
auto-pickup and without giving feedback about them.

%.pg
The
{\it mention\verb+_+walls\/}
option controls whether you get feedback if you try to walk into a wall
or solid stone or off the edge of the map.
Normally nothing happens (unless the hero is blind and no wall is shown,
then the wall that is being bumped into will be drawn on the map).
This option also gives feedback when rushing or running stops for
some non-obvious reason.

%.pg
The
{\it mention\verb+_+decor\/}
option controls whether you get feedback when walking on ``furniture.''
Normally stepping onto stairs or a fountain or an altar or various other
things doesn't elicit anything unless it is covered by one or more objects
so is obscured on the map.
Setting this option to true will describe such things even when they
aren't obscured.
Doorless doorways and open doors aren't considered worthy of mention;
closed doors (if you can move onto their spots) and broken doors are.
Assuming that you're able to do so, moving onto water or lava or ice
will give feedback if not yet on that type of terrain but not repeat it
(unless there has been some intervening message) when moving from water
to another water spot, or lava to lava, or ice to ice.
Moving off of any of those back onto ``normal'' terrain will give one
message too, unless there is feedback about one or more objects, in which
case the back on land circumstance is implied.

%.pg
The
{\it confirm\/}
and
{\it safe\verb+_+pet\/}
options control what happens when you try to move onto a peaceful monster's
spot or a tame one's spot.

%.\" getting away from "Movement feedback" here; oh well...
%.pg
The {\tt nopickup} command prefix (default `{\tt m}') is
also the move-without-attacking prefix and can be used to try to step
onto a visible monster's spot without the move being considered an attack
(see the {\it Fighting\/} subsection of {\it Monsters\/} below).
The `{\tt fight}' command prefix (default `{\tt F}';
also `{\tt -}' if
{\it number\verb+_+pad\/}
is on) can be used to force an attack, when guessing where an unseen
monster is or when deliberately attacking a peaceful or tame creature.

%.pg
The
{\it run\verb+_+mode}
option controls how frequently the map gets redrawn when moving more
than one step in a single command (so when rushing, running, or traveling).

%.hn 2
\subsection*{Rogue level}

%.pg
One dungeon level (occurring in mid to late teens of the main dungeon)
is a tribute to the ancestor game {\it hack}'s inspiration {\it rogue}.

%.pg
It is usually displayed differently from other levels: possibly in
characters instead of tiles, or without line-drawing symbols if already
in characters; also, gold is shown as {\tt *} rather than {\tt \verb+$+}
and stairs are shown as {\tt \verb+%+} rather than {\tt <} and {\tt >}.
There are some minor differences in actual game play: doorways lack
doors; a scroll, wand, or spell of light used in a room lights up the
whole room rather than within a radius around your character.
And monsters represented by lower-case letters aren't randomly
generated on the rogue level.

%.pg
The slight strangeness of this level is a feature, not a bug....

%.hn 1
\section{Monsters}

%.pg
Monsters you cannot see are not displayed on the screen.  Beware!
You may suddenly come upon one in a dark place.  Some magic items can
help you locate them before they locate you (which some monsters can do
very well).

%.pg
The commands `{\tt /}' and `{\tt ;}' may be used to obtain information
about those
monsters who are displayed on the screen.  The command ``{\tt \#name}''
(by default bound to `{\tt C}'), allows you
to assign a name to a monster, which may be useful to help distinguish
one from another when multiple monsters are present.  Assigning a name
which is just a space will remove any prior name.

%.pg
The extended command ``{\tt \#chat}'' can be used to interact with an adjacent
monster.  There is no actual dialog (in other words, you don't get to
choose what you'll say), but chatting with some monsters such as a
shopkeeper or the Oracle of Delphi can produce useful results.

%.hn 2
\subsection*{Fighting}

%.pg
If you see a monster and you wish to fight it, just attempt to walk
into it.  Many monsters you find will mind their own business unless
you attack them.  Some of them are very dangerous when angered.
Remember:  discretion is the better part of valor.

%.pg
In most circumstances, if you attempt to attack a peaceful monster by
moving into its location, you'll be asked to confirm your intent.  By
default an answer of `{\tt y}' acknowledges that intent,
which can be error prone if you're using `{\tt y}' to move.  You can set the
{\it paranoid\verb+_+confirmation\/}
option to require a response of ``{\tt yes}'' instead.
%.pg

If you can't see a monster (if it is invisible, or if you are blinded),
the symbol `{\tt I}' will be shown when you learn of its presence.
If you attempt to walk into it, you will try to fight it just like
a monster that you can see; of course,
if the monster has moved, you will attack empty air.  If you guess
that the monster has moved and you don't wish to fight, you can use the
`{\tt m}' command to move without fighting; likewise, if you don't remember
a monster but want to try fighting anyway, you can use the `{\tt F}' command.

%.hn 2
\subsection*{Your pet}

%.pg
You start the game with a little dog (`{\tt d}'), kitten (`{\tt f}'),
or pony (`{\tt u}'), which follows
you about the dungeon and fights monsters with you.
Like you, your pet needs food to survive.
Dogs and cats usually feed themselves on fresh carrion and other meats;
horses need vegetarian food which is harder to come by.
If you're worried about your pet or want to train it, you
can feed it, too, by throwing it food.
A properly trained pet can be very useful under certain circumstances.

%.pg
Your pet also gains experience from killing monsters, and can grow
over time, gaining hit points and doing more damage.  Initially, your
pet may even be better at killing things than you, which makes pets
useful for low-level characters.

%.pg
Your pet will follow you up and down staircases if it is next to you
when you move.  Otherwise your pet will be stranded and may become
wild.  Similarly, when you trigger certain types of traps which alter
your location (for instance, a trap door which drops you to a lower
dungeon level), any adjacent pet will accompany you and any non-adjacent
pet will be left behind.  Your pet may trigger such traps itself; you
will not be carried along with it even if adjacent at the time.

%.hn 2
\subsection*{Steeds}

%.pg
Some types of creatures in the dungeon can actually be ridden if you
have the right equipment and skill.  Convincing a wild beast to let
you saddle it up is difficult to say the least.  Many a dungeoneer
has had to resort to magic and wizardry in order to forge the alliance.
Once you do have the beast under your control however, you can
easily climb in and out of the saddle with the ``{\tt \#ride}'' command.  Lead
the beast around the dungeon when riding, in the same manner as
you would move yourself.  It is the beast that you will see displayed
on the map.

%.pg
Riding skill is managed by the ``{\tt \#enhance}'' command.  See the section
on Weapon proficiency for more information about that.

%.pg
Use the `{\tt a}' (apply) command and pick a saddle in your inventory to
attempt to put that saddle on an adjacent creature.  If successful,
it will be transferred to that creature's inventory.

%.pg
Use the ``{\tt \#loot}'' command while adjacent to a saddled creature to
try to remove the saddle from that creature.  If successful, it will
be transferred to your inventory.

%.hn 2
\subsection*{Bones levels}

%.pg
You may encounter the shades and corpses of other adventurers (or even
former incarnations of yourself!) and their personal effects.  Ghosts
are hard to kill, but easy to avoid, since they're slow and do little
damage.  You can plunder the deceased adventurer's possessions;
however, they are likely to be cursed.  Beware of whatever killed the
former player; it is probably still lurking around, gloating over its
last victory.

%.hn 2
\subsection*{Persistence of Monsters}

%.pg
Monsters (a generic reference which also includes humans and pets) are only
shown while they can be seen or otherwise sensed.
Moving to a location where you can't see or sense a monster any more
will result in it disappearing from your map, similarly if it is the
one who moved rather than you.

%.pg
However, if you encounter a monster which you can't see or
sense---perhaps it is invisible and has just tapped you on the noggin---a
special ``remembered, unseen monster'' marker will be displayed at
the location where you think it is.
That will persist until you have
proven that there is no monster there, even if the unseen monster
moves to another location or you move to a spot where the marker's
location ordinarily wouldn't be seen any more.

%.hn 1
\section{Objects}

%.pg
When you find something in the dungeon, it is common to want to pick
it up.  In {\it NetHack}, this is accomplished by using the `{\tt ,}' command.
If {\it autopickup\/} option is on, you will automatically pick up the object
by walking over it, unless you move with the `{\tt m}' prefix.
%.pg
If you're carrying too many items, {\it NetHack\/} will tell you so and you
won't be able to pick up anything more.  Otherwise, it will add the object(s)
to your pack and tell you what you just picked up.
%.pg
As you add items to your inventory, you also add the weight of that object
to your load.  The amount that you can carry depends on your strength and
your constitution.  The
stronger and sturdier
you are, the less the additional load will affect you.  There comes
a point, though, when the weight of all of that stuff you are carrying around
with you through the dungeon will encumber you.  Your reactions
will get slower and you'll burn calories faster, requiring food more frequently
to cope with it.  Eventually, you'll be so overloaded that you'll either have
to discard some of what you're carrying or collapse under its weight.
%.pg
{\it NetHack\/} will tell you how badly you have loaded yourself.
If you are encumbered, one of the conditions
{\tt Burdened}, {\tt Stressed}, {\tt Strained},
{\tt Overtaxed}, or {\tt Overloaded} will be
shown on the bottom line status display.

%.pg
When you pick up an object, it is assigned an inventory letter.  Many
commands that operate on objects must ask you to find out which object
you want to use.  When {\it NetHack\/} asks you to choose a particular object
you are carrying, you are usually presented with a list of inventory
letters to choose from (see Commands, above).

%.pg
Some objects, such as weapons, are easily differentiated.  Others, like
scrolls and potions, are given descriptions which vary according to
type.  During a game, any two objects with the same description are
the same type.  However, the descriptions will vary from game to game.

%.pg
When you use one of these objects, if its effect is obvious, {\it NetHack\/}
will remember what it is for you.  If its effect isn't extremely
obvious, you will be asked what you want to call this type of object
so you will recognize it later.  You can also use the ``{\tt \#name}''
command, for the same purpose at any time, to name
all objects of a particular type or just an individual object.
When you use ``{\tt \#name}'' on an object which has already been named,
specifying a space as the value will remove the prior name instead
of assigning a new one.

%.hn 2
\subsection*{Curses and Blessings}

%.pg
Any object that you find may be cursed, even if the object is
otherwise helpful.  The most common effect of a curse is being stuck
with (and to) the item.  Cursed weapons weld themselves to your hand
when wielded, so you cannot unwield them.  Any cursed item you wear
is not removable by ordinary means.  In addition, cursed arms and armor
usually, but not always, bear negative enchantments that make them
less effective in combat.  Other cursed objects may act poorly or
detrimentally in other ways.

%.pg
Objects can also be blessed instead.
Blessed items usually work better or more beneficially than normal
uncursed items.
For example, a blessed weapon will do slightly more damage against demons.

%.pg
Objects which are neither cursed nor blessed are referred to as uncursed.
They could just as easily have been described as unblessed, but the
uncursed designation is what you will see within the game.  A ``glass
half full versus glass half empty'' situation; make of that what you will.

%.pg
There are magical means of bestowing or removing curses upon objects,
so even if you are stuck with one, you can still have the curse
lifted and the item removed.
Priests and Priestesses have an innate
sensitivity to this property in any object, so they can more easily avoid
cursed objects than other character roles.
Dropping objects onto an altar will reveal their bless or curse state
provided that you can see them land.

%.pg
An item with unknown status will be reported in your inventory with no prefix.
An item which you know the state of will be distinguished in your inventory
by the presence of the word {\tt cursed}, {\tt uncursed} or
{\tt blessed} in the description of the item.
In some cases {\tt uncursed} will be omitted as being redundant when
enough other information is displayed.
The
{\it implicit\verb+_+uncursed\/}
option can be used to control this; toggle it off to have {\tt uncursed}
be displayed even when that can be deduced from other attributes.

%.pg
Sometimes the bless or curse state of objects is referred to as their
``{\tt BUC}'' attribute, for Blessed, Uncursed, or Cursed state,
or ``{\tt BUCX}'' for Blessed, Uncursed, Cursed, or unknown.
(The term {\it beatitude\/} is occasionally used as well.)

%.hn 2
\subsection*{Weapons (`{\tt )}')}

%.pg
Given a chance, most monsters in the Mazes of Menace will gratuitously try to
kill you.
You need weapons for self-defense (killing them first).
Without a
weapon, you do only 1--2 hit points of damage (plus bonuses, if any).
Monk characters are an exception; they normally do more damage with
bare (or gloved) hands than they do with weapons.

%.pg
There are wielded weapons, like maces and swords, and thrown weapons,
like arrows and spears.  To hit monsters with a weapon, you must wield it and
attack them, or throw it at them.  You can simply elect to throw a spear.
To shoot an arrow, you should first wield a bow, then throw the arrow.
Crossbows shoot crossbow bolts.  Slings hurl rocks and (other) stones
(like gems).

%.pg
Enchanted weapons have a ``plus'' (or ``to hit enhancement'' which can be
either positive or negative) that adds to your chance to
hit and the damage you do to a monster.  The only way to determine a weapon's
enchantment is to have it magically identified somehow.
Most weapons are subject to some type of damage like rust.  Such
``erosion'' damage can be repaired.

%.pg
The chance that an attack will successfully hit a monster, and the amount
of damage such a hit will do, depends upon many factors.  Among them are:
type of weapon, quality of weapon (enchantment and/or erosion), experience
level, strength, dexterity, encumbrance, and proficiency (see below).  The
monster's armor
class---a general defense rating, not necessarily due to wearing of armor---is
a factor too; also, some monsters are particularly
vulnerable to certain types of weapons.

%.pg
Many weapons can be wielded in one hand; some require both hands.
When wielding a two-handed weapon, you can not wear a shield, and
vice versa.  When wielding a one-handed weapon, you can have another
weapon ready to use by setting things up with the `{\tt x}' command, which
exchanges your primary (the one being wielded) and alternate weapons.
And if you have proficiency in the ``two weapon combat'' skill, you
may wield both weapons simultaneously as primary and secondary; use the
`{\tt X}' command to engage or disengage that.
Only some types of characters (barbarians, for instance) have the necessary
skill available.  Even with that skill, using two weapons at once incurs
a penalty in the chance to hit your target compared to using just one
weapon at a time.

%.pg
There might be times when you'd rather not wield any weapon at all.
To accomplish that, wield `{\tt -}', or else use the `{\tt A}' command which
allows you to unwield the current weapon in addition to taking off
other worn items.

%.pg
Those of you in the audience who are AD\&D players, be aware that each
weapon which existed in AD\&D does roughly the same damage to monsters in
{\it NetHack}.  Some of the more obscure weapons (such as the
{\it aklys}, {\it lucern hammer}, and {\it bec-de-corbin\/}) are defined
in an appendix to {\it Unearthed Arcana}, an AD\&D supplement.

%.pg
The commands to use weapons are `{\tt w}' (wield), `{\tt t}' (throw),
`{\tt f}' (fire), `{\tt Q}' (quiver),
`{\tt x}' (exchange), `{\tt X}' (twoweapon), and ``{\tt \#enhance}''
(see below).

%.hn 3
\subsection*{Throwing and shooting}

%.pg
You can throw just about anything via the `{\tt t}' command.  It will prompt
for the item to throw; picking `{\tt ?}' will list things in your inventory
which are considered likely to be thrown, or picking `{\tt *}' will list
your entire inventory.  After you've chosen what to throw, you will
be prompted for a direction rather than for a specific target.  The
distance something can be thrown depends mainly on the type of object
and your strength.  Arrows can be thrown by hand, but can be thrown
much farther and will be more likely to hit when thrown while you are
wielding a bow.

%.pg
Some weapons will return when thrown.
A boomerang---provided it fails to hit anything---is an obvious example.
If an aklys (thonged club) is thrown while it is wielded, it will return
even when it hits something.
A sufficiently strong hero can throw the warhammer {\it Mjollnir\/};
when thrown by a {\it Valkyrie\/} it will return too.
However, aklyses and {\it Mjollnir\/} occasionally fail to return.
Returning thrown objects occasionally fail to be caught, sometimes even
hitting the thrower, but when caught they become re-wielded.

%.pg
You can simplify the throwing operation by using the `{\tt Q}' command to
select your preferred ``missile'', then using the `{\tt f}' command to
throw it.  You'll be prompted for a direction as above, but you don't
have to specify which item to throw each time you use `{\tt f}'.  There is
also an option,
{\it autoquiver},
which has {\it NetHack\/} choose another item to automatically fill your
quiver (or quiver sack, or have at the ready) when the inventory slot used
for `{\tt Q}' runs out.
If your quiver is empty, {\it autoquiver\/}
is false, and you are wielding a weapon which returns when thrown,
you will throw that weapon instead of filling the quiver.
The fire command also has extra assistance, if {\it fireassist\/}
is on it will try to wield a launcher matching the ammo in the quiver.

%.pg
Some characters have the ability to throw or shoot a volley of multiple
items (from the same stack) in a single action.
Knowing how to load several rounds of ammunition at
once---or hold several missiles in your hand---and
still hit a target is not an easy task.
Rangers are among those who are adept
at this task, as are those with a high level of proficiency in the
relevant weapon skill (in bow skill if you're wielding one to
shoot arrows, in crossbow skill if you're wielding one to shoot bolts,
or in sling skill if you're wielding one to shoot stones).
The number of items that the character has a chance to fire varies from
turn to turn.  You can explicitly limit the number of shots by using a
numeric prefix before the `{\tt t}' or `{\tt f}' command.
For example, ``{\tt 2f}'' (or ``{\tt n2f}'' if using
{\it number\verb+_+pad\/}
mode) would ensure that at most 2 arrows are shot
even if you could have fired 3.  If you specify
a larger number than would have been shot (``{\tt 4f}'' in this example),
you'll just end up shooting the same number (3, here) as if no limit
had been specified.  Once the volley is in motion, all of the items
will travel in the same direction; if the first ones kill a monster,
the others can still continue beyond that spot.

%.hn 3
\subsection*{Weapon proficiency}

%.pg
You will have varying degrees of skill in the weapons available.
Weapon proficiency, or weapon skills, affect how well you can use
particular types of weapons, and you'll be able to improve your skills
as you progress through a game, depending on your role, your experience
level, and use of the weapons.

%.pg
For the purposes of proficiency, weapons have
been divided up into various groups such as daggers, broadswords, and
polearms.  Each role has a limit on what level of proficiency a character
can achieve for each group.  For instance, wizards can become highly
skilled in daggers or staves but not in swords or bows.

%.pg
The ``{\tt \#enhance}'' extended command is used to review current weapons
proficiency
(also spell proficiency) and to choose which skill(s) to improve when
you've used one or more skills enough to become eligible to do so.  The
skill rankings are ``none'' (sometimes also referred to as ``restricted'',
because you won't be able to advance), ``unskilled'', ``basic'', ``skilled'',
and ``expert''.  Restricted skills simply will not appear in the list
shown by ``{\tt \#enhance}''.
(Divine intervention might unrestrict a particular
skill, in which case it will start at unskilled and be limited to basic.)
Some characters can enhance their barehanded combat or martial arts skill
beyond expert to ``master'' or ``grand master''.

%.pg
Use of a weapon in which you're restricted or unskilled
will incur a modest penalty in the chance to hit a monster and also in
the amount of damage done when you do hit; at basic level, there is no
penalty or bonus; at skilled level, you receive a modest bonus in the
chance to hit and amount of damage done; at expert level, the bonus is
higher.  A successful hit has a chance to boost your training towards
the next skill level (unless you've already reached the limit for this
skill).  Once such training reaches the threshold for that next level,
you'll be told that you feel more confident in your skills.  At that
point you can use ``{\tt \#enhance}'' to increase one or more skills.
Such skills
are not increased automatically because there is a limit to your total
overall skills, so you need to actively choose which skills to enhance
and which to ignore.

%.hn 3
\subsection*{Two-Weapon combat}

%.pg
Some characters can use two weapons at once.  Setting things up to
do so can seem cumbersome but becomes second nature with use.
To wield two weapons, you need to use the ``{\tt \#twoweapon}'' command.
But first you need to have a weapon in each hand.
(Note that your two weapons are not fully equal; the one in the
hand you normally wield with is considered primary and the other
one is considered secondary.
The most noticeable difference is after you
stop---or before you begin, for that matter---wielding
two weapons at once.
The primary is your wielded weapon and the
secondary is just an item in your inventory that's been designated
as alternate weapon.)

%.pg
If your primary weapon is wielded but your off hand is empty or has
the wrong weapon, use the sequence `{\tt x}', `{\tt w}', `{\tt x}' to
first swap your
primary into your off hand, wield whatever you want as secondary
weapon, then swap them both back into the intended hands.
If your secondary or alternate weapon is correct but your primary
one is not, simply use `{\tt w}' to wield the primary.
Lastly, if neither hand holds the correct weapon,
use `{\tt w}', `{\tt x}', `{\tt w}'
to first wield the intended secondary, swap it to off hand, and then
wield the primary.

%.pg
The whole process can be simplified via use of the
{\it pushweapon\/}
option.  When it is enabled, then using `{\tt w}' to wield something
causes the currently wielded weapon to become your alternate weapon.
So the sequence `{\tt w}', `{\tt w}' can be used to first wield the weapon you
intend to be secondary, and then wield the one you want as primary
which will push the first into secondary position.

%.pg
When in two-weapon combat mode, using the `{\tt X}' command
toggles back to single-weapon mode.
Throwing or dropping either of the
weapons or having one of them be stolen or destroyed will also make you
revert to single-weapon combat.

%.hn 2
\subsection*{Armor (`{\tt [}')}

%.pg
Lots of unfriendly things lurk about; you need armor to protect
yourself from their blows.  Some types of armor offer better
protection than others.  Your armor class is a measure of this
protection.  Armor class (AC) is measured as in AD\&D, with 10 being
the equivalent of no armor, and lower numbers meaning better armor.
Each suit of armor which exists in AD\&D gives the same protection in
{\it NetHack}.

Here is a list of the armor class values provided by suits of armor:

\begin{center}
\begin{tabular}{lllll}
dragon scale mail      & 1 & \makebox[20mm]{}  & plate mail            & 3\\
crystal plate mail     & 3 &                   & bronze plate mail     & 4\\
splint mail            & 4 &                   & banded mail           & 4\\
dwarvish mithril-coat  & 4 &                   & elven mithril-coat    & 5\\
chain mail             & 5 &                   & orcish chain mail     & 6\\
scale mail             & 6 &                   & dragon scales         & 7\\
studded leather armor  & 7 &                   & ring mail             & 7\\
orcish ring mail       & 8 &                   & leather armor         & 8\\
leather jacket         & 9 &                   & no armor              & 10\\
\end{tabular}
\end{center}

%.pg
\nd You can also wear other pieces of armor (cloak over suit, shirt under
suit, helmet, gloves, boots, shield) to lower your armor class even
further.
%--too obvious to mention unless we include polymorph into ettin or maralith
% You can wear at most one item of each category (one suit of armor, one
% cloak, one helmet, one shield, and so on) at a time.
Most of these provide a one or two point improvement to AC (making the
overall value smaller and eventually negative) but can also be
enchanted.
Shirts are an exception; they don't provide any protection unless enchanted.
Some cloaks also don't improve AC when unenchanted but all cloaks offer
some protection against rust or corrosion to suits worn under them and
against some monster {\it touch\/} attacks.

%.pg
If a piece of armor is enchanted, its armor protection will be better
(or worse) than normal, and its ``plus'' (or minus) will subtract from
your armor class.  For example, a +1 chain mail would give you
better protection than normal chain mail, lowering your armor class one
unit further to 4.  When you put on a piece of armor, you immediately
find out the armor class and any ``plusses'' it provides.  Cursed
pieces of armor usually have negative enchantments (minuses) in
addition to being unremovable.

%.pg
Many types of armor are subject to some kind of damage like rust.  Such
damage can be repaired.  Some types of armor may inhibit spell casting.

%.pg
The {\it nudist\/}
option can be set (prior to game start) to attempt to play the entire
game without wearing any armor (a self-imposed challenge which is
extremely difficult to accomplish).

%.pg
The commands to use armor are `{\tt W}' (wear) and `{\tt T}' (take off).
The `{\tt A}' command can be used to take off armor as well as other
worn items.
Also, `{\tt P}' (put on) and `{\tt R}' (remove) which are normally for
accessories can be used for armor, but pieces of armor won't be shown
as likely candidates in a prompt for choosing what to put on or remove.

%.hn 2
\subsection*{Food (`{\tt \%}')}

%.pg
Food is necessary to survive.  If you go too long without eating you
will faint, and eventually die of starvation.
Some types of food will spoil, and become unhealthy to eat,
if not protected.
Food stored in ice boxes or tins (``cans'')
will usually stay fresh, but ice boxes are heavy, and tins
take a while to open.

%.pg
When you kill monsters, they usually leave corpses which are also
``food.''  Many, but not all, of these are edible; some also give you
special powers when you eat them.  A good rule of thumb is ``you are
what you eat.''

%.pg
Some character roles and some monsters are vegetarian.  Vegetarian monsters
will typically never eat animal corpses, while vegetarian players can,
but with some rather unpleasant side-effects.

%.pg
You can name one food item after something you like to eat with the
{\it fruit\/} option.

%.pg
The command to eat food is `{\tt e}'.

%.hn 2
\subsection*{Scrolls (`{\tt ?}')}

%.pg
Scrolls are labeled with various titles, probably chosen by ancient wizards
for their amusement value (for example, ``READ ME,'' or ``THANX MAUD'' backwards).
Scrolls disappear after you read them (except for blank ones, without
magic spells on them).

%.pg
One of the most useful of these is the %
{\it scroll of identify}, which
can be used to determine what another object is, whether it is cursed or
blessed, and how many uses it has left.  Some objects of subtle
enchantment are difficult to identify without these.

%.pg
A mail daemon may run up and deliver mail to you as a %
{\it scroll of mail} (on versions compiled with this feature).
To use this feature on versions where {\it NetHack\/}
mail delivery is triggered by electronic mail appearing in your system mailbox,
you must let {\it NetHack\/} know where to look for new mail by setting the
{\tt MAIL} environment variable to the file name of your mailbox.
You may also want to set the {\tt MAILREADER} environment variable to the
file name of your favorite reader, so {\it NetHack\/} can shell to it when you
read the scroll.
On versions of {\it NetHack\/} where mail is randomly
generated internal to the game, these environment variables are ignored.
You can disable the mail daemon by turning off the
{\it mail\/} option.

%.pg
The command to read a scroll is `{\tt r}'.

%.hn 2
\subsection*{Potions (`{\tt !}')}

%.pg
Potions are distinguished by the color of the liquid inside the flask.
They disappear after you quaff them.

%.pg
Clear potions are potions of water.  Sometimes these are
blessed or cursed, resulting in holy or unholy water.  Holy water is
the bane of the undead, so potions of holy water are good things to
throw (`{\tt t}') at them.  It is also sometimes very useful to dip
(``{\tt \#dip}'') an object into a potion.

%.pg
The command to drink a potion is `{\tt q}' (quaff).

%.hn 2
\subsection*{Wands (`{\tt /}')}

%.pg
Wands usually have multiple magical charges.
Some types of wands require a direction in which to zap them.
You can also
zap them at yourself (just give a `{\tt .}' or `{\tt s}' for the direction).
Be warned, however, for this is often unwise.
Other types of wands
don't require a direction.  The number of charges in a
wand is random and decreases by one whenever you use it.

%.pg
When the number of charges left in a wand becomes zero, attempts to use the
wand will usually result in nothing happening.  Occasionally, however, it may
be possible to squeeze the last few mana points from an otherwise spent wand,
destroying it in the process.  A wand may be recharged by using suitable
magic, but doing so runs the risk of causing it to explode.  The chance
for such an explosion starts out very small and increases each time the
wand is recharged.

%.pg
In a truly desperate situation, when your back is up against the wall, you
might decide to go for broke and break your wand.  This is not for the faint
of heart.  Doing so will almost certainly cause a catastrophic release of
magical energies.

%.pg
When you have fully identified a particular wand, inventory display will
include additional information in parentheses: the number of times it has
been recharged followed by a colon and then by its current number of charges.
A current charge count of {\tt -1} is a special case indicating that the wand
has been cancelled.

%.pg
The command to use a wand is `{\tt z}' (zap).  To break one, use the `{\tt a}'
(apply) command.

%.hn 2
\subsection*{Rings (`{\tt =}')}

%.pg
Rings are very useful items, since they are relatively permanent
magic, unlike the usually fleeting effects of potions, scrolls, and
wands.

%.pg
Putting on a ring activates its magic.
You can wear at most two rings at any time, one on the ring finger of
each hand.

%.pg
Most worn rings also cause you to grow hungry more rapidly, the rate
varying with the type of ring.

%.pg
When wearing gloves, rings are worn underneath.
If the gloves are cursed, rings cannot be put on and any already being
worn cannot be removed.
When worn gloves aren't cursed, you don't have to manually take them
off before putting on or removing a ring and then re-wear them after.
That's done implicitly to avoid unnecessary tedium.

%.pg
The commands to use rings are `{\tt P}' (put on) and `{\tt R}' (remove).
`{\tt A}', `{\tt W}', and `{\tt T}' can also be used; see {\it Amulets\/}.

%.hn 2
\subsection*{Spellbooks (`{\tt +}')}

%.pg
Spellbooks are tomes of mighty magic.  When studied with the `{\tt r}' (read)
command, they transfer to the reader the knowledge of a spell (and
therefore eventually become
unreadable)---unless the attempt backfires.
Reading a cursed spellbook or one with mystic runes beyond
your ken can be harmful to your health!

%.pg
A spell (even when learned) can also backfire when you cast it.  If you
attempt to cast a spell well above your experience level, or if you have
little skill with the appropriate spell type, or cast it at
a time when your luck is particularly bad, you can end up wasting both the
energy and the time required in casting.

%.pg
Casting a spell calls forth magical energies and focuses them with
your naked mind.  Some of the magical energy released comes from within
you.
Casting temporarily drains your magical power, which will slowly be
recovered, and causes you to need additional food.
Casting of spells also requires practice.  With practice, your
skill in each category of spell casting will improve.  Over time, however,
your memory of each spell will dim, and you will need to relearn it.

%.pg
Some spells require a direction in which to cast them, similar to wands.
To cast one at yourself, just give a `{\tt .}' or `{\tt s}' for the direction.
A few spells require you to pick a target location rather than just specify
a particular direction.
Other spells don't require any direction or target.

%.pg
Just as weapons are divided into groups in which a character can become
proficient (to varying degrees), spells are similarly grouped.
Successfully casting a spell exercises its skill group; using the
``{\tt \#enhance}'' command to advance a sufficiently exercised skill
will affect all spells within the group.  Advanced skill may increase the
potency of spells, reduce their risk of failure during casting attempts,
and improve the accuracy of the estimate for how much longer they will
be retained in your memory.
Skill slots are shared with weapons skills.  (See also the section on
``Weapon proficiency''.)

%.pg
Casting a spell also requires flexible movement, and wearing various types
of armor may interfere with that.

%.pg
The command to read a spellbook is the same as for scrolls, `{\tt r}' (read).
The `{\tt +}' command lists each spell you know along with its level, skill
category, chance of failure when casting, and an estimate of how strongly
it is remembered.
The `{\tt Z}' (cast) command casts a spell.

%.hn 2
\subsection*{Tools (`{\tt (}')}

%.pg
Tools are miscellaneous objects with various purposes.  Some tools
have a limited number of uses, akin to wand charges.  For example, lamps burn
out after a while.  Other tools are containers, which objects can
be placed into or taken out of.

%.pg
Some tools (such as a blindfold) can be {\it worn\/} and can be put on and
removed like other accessories (rings, amulets); see {\it Amulets\/}.
Other tools (such as pick-axe) can be wielded as weapons in addition to
being applied for their usual purpose, and in some cases (again, pick-axe)
become wielded as a weapon even when applied.

%.pg
% Mentioned here because of the old method of attempting "Zen" conduct:
% restart until there's a blindfold in starting inventory and put it on
% first thing.
The {\it blind\/}
option can be set (prior to game start) to attempt to play the entire
game without being able to see (a self-imposed challenge which is
very difficult to accomplish).

%.pg
The command to use a tool is `{\tt a}' (apply).

%.hn 3
\subsection*{Containers}

%.pg
You may encounter bags, boxes, and chests in your travels.  A tool of
this sort can be opened with the ``{\tt \#loot}'' extended command when
you are standing on top of it (that is, on the same floor spot),
or with the `{\tt a}' (apply) command when you are carrying it.  However,
chests are often locked, and are in any case unwieldy objects.
You must set one down before unlocking it by
using a key or lock-picking tool with the `{\tt a}' (apply) command,
by kicking it with the `{\tt \^{}D}' command,
or by using a weapon to force the lock with the ``{\tt \#force}''
extended command.

%.pg
Some chests are trapped, causing nasty things to happen when you
unlock or open them.  You can check for and try to deactivate traps
with the ``{\tt \#untrap}'' extended command.

%.hn 2
\subsection*{Amulets (`{\tt "}')}

%.pg
Amulets are very similar to rings, and often more powerful.  Like
rings, amulets have various magical properties, some beneficial,
some harmful, which are activated by putting them on.

%.pg
Only one amulet may be worn at a time, around your neck.
Like wearing rings, wearing an amulet affects your metabolism, causing
you to grow hungry more rapidly.

%.pg
The commands to use amulets are the same as for rings, `{\tt P}' (put on)
and `{\tt R}' (remove).
`{\tt A}' can be used to remove various worn items including amulets.
Also, '{\tt W}' (wear) and `{\tt T}' (take off) which are normally for
armor can be used for amulets and other accessories (rings and eyewear),
but accessories won't be shown as likely candidates in a prompt for
choosing what to wear or take off.

%.hn 2
\subsection*{Gems (`{\tt *}')}

%.pg
Some gems are valuable, and can be sold for a lot of gold.
They are also a far more efficient way of carrying your riches.
Valuable gems increase your score if you bring them with you when you exit.

%.pg
Other small rocks are also categorized as gems, but they are much less
valuable.
All rocks, however, can be used as projectile weapons (if you have a sling).
In the most desperate of cases, you can still throw them by hand.

%.hn 2
\subsection*{Large rocks (`{\tt `}')}
%.pg
Statues and boulders are not particularly useful, and are generally heavy.
It is rumored that some statues are not what they seem.

%.pg
Boulders occasionally block your path.
You can push one forward (by attempting to walk onto its spot)
when nothing blocks {\it its\/} path, or you can
smash it into a pile of small rocks with breaking magic or a pick-axe.
It is possible to move onto a boulder's location if certain conditions
are met; ordinarily one of those conditions is that pushing it any
further be blocked.
Using the move-without-picking-up prefix (default key `{\tt m}')
prior to the direction of movement will attempt to move to a boulder's
location without pushing it in addition to the prefix's usual action of
suppressing auto-pickup at the destination.

%.pg
Very large humanoids (giants and their ilk) have been known to pick up
boulders and use them as missile weapons.

%.pg
Unlike boulders, statues can't be pushed, but don't need to be because
they don't block movement.
%.\" 'rumor' above is about statue traps; this is a hint about statue contents
They can be smashed into rocks though.

%.pg
For some configurations of the program, statues are no longer shown
as `{\tt `}'
but by the letter representing the monster they depict instead.

%.hn 2
\subsection*{Gold (`{\tt \$}')}

%.pg
Gold adds to your score, and you can buy things in shops with it.
There are a number
of monsters in the dungeon that may be influenced by the amount of gold
you are carrying (shopkeepers aside).

%.pg
Gold pieces are the only type of object where bless/curse state does not
apply.
They're always uncursed but never described as uncursed even if you turn
off the ``{\it implicit\verb+_+uncursed\/}'' option.
You can set the ``{\it goldX\/}''
option if you prefer to have gold pieces be treated as bless/curse state
{\it unknown\/} rather than as known to be uncursed.
Only matters when you're using an object selection prompt that can filter
by ``{\tt BUCX}'' state.

%.hn 2
\subsection*{Persistence of Objects}

%.pg
Normally, if you have seen an object at a particular map location and
move to another location where you can't directly see that object any
more, it will continue to be displayed on your map.
That remains the case even if it is not actually there any
more---perhaps a monster has picked it up or it has rotted away---until
you can see or feel that location again.
One notable exception is that if the object gets covered by the
``remembered, unseen monster'' marker.
When that marker is later removed
after you've verified that no monster is there, you will have forgotten that
there was any object there regardless of whether the unseen monster
actually took the object.
If the object is still there, then once you see or feel that location
again you will re-discover the object and resume remembering it.

%.pg
The situation is the same for a pile of objects, except that only the
top item of the pile is displayed.
The
{\it hilite\verb+_+pile\/}
option can be enabled in order to show an item differently when it is
the top one of a pile.

%.hn 1
\section{Conduct}

%.pg
As if winning {\it NetHack\/} were not difficult enough, certain players
seek to challenge themselves by imposing restrictions on the
way they play the game.  The game automatically tracks some of
these challenges, which can be checked at any time with the {\tt \#conduct}
command or at the end of the game.  When you perform an action which
breaks a challenge, it will no longer be listed.  This gives
players extra ``bragging rights'' for winning the game with these
challenges.  Note that it is perfectly acceptable to win the game
without resorting to these restrictions and that it is unusual for
players to adhere to challenges the first time they win the game.

%.pg
Several of the challenges are related to eating behavior.  The most
difficult of these is the foodless challenge.  Although creatures
can survive long periods of time without food, there is a physiological
need for water; thus there is no restriction on drinking beverages,
even if they provide some minor food benefits.
Calling upon your god for help with starvation does
not violate any food challenges either.

%.pg
A strict vegan diet is one which avoids any food derived from animals.
The primary source of nutrition is fruits and vegetables.  The
corpses and tins of blobs (`{\tt b}'), jellies (`{\tt j}'), and fungi
(`{\tt F}') are also considered to be vegetable matter.  Certain human
food is prepared without animal products; namely, lembas wafers, cram
rations, food rations (gunyoki), K-rations, and C-rations.
Metal or another normally indigestible material eaten while polymorphed
into a creature that can digest it is also considered vegan food.
Note however that eating such items still counts against foodless conduct.

%.pg
Vegetarians do not eat animals;
however, they are less selective about eating animal byproducts than vegans.
In addition to the vegan items listed above, they may eat any kind
of pudding (`{\tt P}') other than the black puddings,
eggs and food made from eggs (fortune cookies and pancakes),
food made with milk (cream pies and candy bars), and lumps of
royal jelly.  Monks are expected to observe a vegetarian diet.

%.pg
Eating any kind of meat violates the vegetarian, vegan, and foodless
conducts.  This includes tripe rations, the corpses or tins of any
monsters not mentioned above, and the various other chunks of meat
found in the dungeon.  Swallowing and digesting a monster while polymorphed
is treated as if you ate the creature's corpse.
Eating leather, dragon hide, or bone items while
polymorphed into a creature that can digest it, or eating monster brains
while polymorphed into a mind flayer, is considered eating
an animal, although wax is only an animal byproduct.

%.pg
Regardless of conduct, there will be some items which are indigestible,
and others which are hazardous to eat.  Using a swallow-and-digest
attack against a monster is equivalent to eating the monster's corpse.
Please note that the term ``vegan'' is used here only in the context of
diet.  You are still free to choose not to use or wear items derived
from animals (e.g. leather, dragon hide, bone, horns, coral), but the
game will not keep track of this for you.  Also note that ``milky''
potions may be a translucent white, but they do not contain milk,
so they are compatible with a vegan diet.  Slime molds or
player-defined ``fruits'', although they could be anything
from ``cherries'' to ``pork chops'', are also assumed to be vegan.

%.pg
An atheist is one who rejects religion.  This means that you cannot
{\tt \#pray}, {\tt \#offer} sacrifices to any god,
{\tt \#turn} undead, or {\tt \#chat} with a priest.
Particularly selective readers may argue that playing Monk or Priest
characters should violate this conduct; that is a choice left to the
player.  Offering the Amulet of Yendor to your god is necessary to
win the game and is not counted against this conduct.  You are also
not penalized for being spoken to by an angry god, priest(ess), or
other religious figure; a true atheist would hear the words but
attach no special meaning to them.

%.pg
Most players fight with a wielded weapon (or tool intended to be
wielded as a weapon).  Another challenge is to win the game without
using such a wielded weapon.  You are still permitted to throw,
fire, and kick weapons; use a wand, spell, or other type of item;
or fight with your hands and feet.

%.pg
In {\it NetHack}, a pacifist refuses to cause the death of any other monster
(i.e. if you would get experience for the death).  This is a particularly
difficult challenge, although it is still possible to gain experience
by other means.

%.pg
An illiterate character does not read or write.  This includes reading
a scroll, spellbook, fortune cookie message, or t-shirt; writing a
scroll; or making an engraving of anything other than a single ``X'' (the
traditional signature of an illiterate person).  Reading an engraving,
or any item that is absolutely necessary to win the game, is not counted
against this conduct.  The identity of scrolls and spellbooks (and
knowledge of spells) in your starting inventory is assumed to be
learned from your teachers prior to the start of the game and isn't
counted.

%.pg
There is a side-branch to the main dungeon called ``Sokoban,'' briefly
described in the earlier section about {\it Traps}.
As mentioned there, the goal is to push boulders into pits and/or holes
to plug those in order to both get the boulders out of your way and be
able to go past the traps.
There are some special ``rules'' that are active when in that branch
of the dungeon.
Some rules can't be bypassed, such as being unable to push a boulder
diagonally.
Other rules can, such as not smashing boulders with magic or tools,
but doing so causes you to receive a luck penalty.
No message about that is given at the time, but it is tracked as a conduct.
The {\tt \#conduct} command and end of game disclosure will report whether
you have abided by the special rules of Sokoban, and if not, how many
times you violated them, providing you with a way to discover which
actions incur bad luck so that you can be better informed about whether
or not to avoid repeating those actions in the future.
(Note:  the Sokoban conduct will only be displayed if you have
entered the Sokoban branch of the dungeon during the current game.
Once that has happened, it becomes part of disclosed conduct even if
you haven't done anything interesting there.
Ending the game with ``never broke the Sokoban rules'' conduct is most
meaningful if you also manage to perform
the ``obtained the Sokoban prize'' achievement
(see {\it Achievements\/} below).)

%.pg
There are several other challenges tracked by the game.  It is possible
to eliminate one or more species of monsters by genocide; playing without
this feature is considered a challenge.  When the game offers you an
opportunity to genocide monsters, you may respond with the monster type
``none'' if you want to decline.  You can change the form of an item into
another item of the same type (``polypiling'') or the form of your own
body into another creature (``polyself'') by wand, spell, or potion of
polymorph; avoiding these effects are each considered challenges.
Polymorphing monsters, including pets, does not break either of these
challenges.
Finally, you may sometimes receive wishes; a game without an attempt to
wish for any items is a challenge, as is a game without wishing for
an artifact (even if the artifact immediately disappears).  When the
game offers you an opportunity to make a wish for an item, you may
choose ``nothing'' if you want to decline.

%.hn 2
\subsection*{Achievements}

%.pg
End of game disclosure will also display various achievements
representing progress toward ultimate ascension, if any have been
attained.
They aren't directly related to {\it conduct\/} but are grouped with
it because they fall into the same category of ``bragging rights''
and to limit the number of questions during disclosure.
Listed here roughly in order of difficulty and not necessarily in the order
in which you might accomplish them.

% [length stuff copied from paranoid_confirmation]
\newlength{\achwidth}
%.PS "Mines'\~End\~"
\settowidth{\achwidth}{\tt Mines'~End~}
\addtolength{\achwidth}{\labelsep}
\blist{\leftmargin \achwidth \topsep 1mm \itemsep 0mm}
%.PL Shop
\item[{\tt <Rank>}]
Attained rank title {\it Rank}.
\item[{\tt Shop}]
Entered a shop.
\item[{\tt Temple}]
Entered a temple.
\item[{\tt Mines}]
Entered the Gnomish Mines.
\item[{\tt Town}]
Entered Mine Town.
\item[{\tt Oracle}]
Consulted the Oracle of Delphi.
\item[{\tt Novel}]
Read a passage from a Discworld Novel.
\item[{\tt Sokoban}]
Entered Sokoban.
\item[{\tt "Big~Room"}]
Entered the Big Room.
\item[{\tt "Soko-Prize"}]
Explored to the top of Sokoban and found a special item there.
\item[{\tt Mines'~End}]
Explored to the bottom of the Gnomish Mines and found a special item there.
\item[{\tt Medusa}]
Defeated Medusa.
\item[{\tt Tune}]
Discovered the tune that can be used to open and close the drawbridge on
the Castle level.
\item[{\tt Bell}]
Acquired the Bell of Opening.
\item[{\tt Gehennom}]
Entered Gehennom.
\item[{\tt Candle}]
Acquired the Candelabrum of Invocation.
\item[{\tt Book}]
Acquired the Book of the Dead.
\item[{\tt Invocation}]
Gained access to the bottommost level of Gehennom.
\item[{\tt Amulet}]
Acquired the fabled Amulet of Yendor.
\item[{\tt Endgame}]
Reached the Elemental Planes.
\item[{\tt Astral}]
Reached the Astral Plane level.
\item[{\tt Blind}]
Blind from birth.
\item[{\tt Deaf}]
Deaf from birth.
\item[{\tt Nudist}]
Never wore any armor.
\item[{\tt Ascended}]
Delivered the Amulet to its final destination.
\elist
%.PE
%.ED

%.lp "Notes:  "
\noindent
Notes:

%.pg
Achievements are recorded and subsequently reported in the order in which
they happen during your current game rather than the order listed here.

%.pg
There are nine {\it Rank} titles for each role, bestowed at experience
levels 1, 3, 6, 10, 14, 18, 22, 26, and 30.
The one for experience level 1 is not recorded as an achievement.
Losing enough levels to revert to lower rank(s) does not discard the
corresponding achievement(s).

%.pg
There's no guaranteed {\it Novel} so the achievement to read one might
not always be attainable (except perhaps by wishing).
Similarly, the {\it Big Room} level is not always present.
Unlike with the Novel, there's no way to wish for this opportunity.

%.pg
The ``special items'' hidden in {\it Mines'~End\/} and {\it Sokoban\/}
are not unique but are considered to be prizes or rewards
for exploring those levels since doing so is not necessary to complete
the game.
Finding other instances of the same objects doesn't record the
corresponding achievement.

%.pg
The {\it Medusa\/} achievement is recorded if she dies for any reason,
even if you are not directly responsible, and only if she dies.

%.pg
The 5-note {\it tune\/} can be learned via trial and error with a musical
instrument played closely
enough---but not too close!---to
the Castle level's drawbridge or can be given to you via prayer boon.

%.pg
{\it Blind\/}, {\it Deaf\/}, and {\it Nudist\/} are also conducts, and they can only be
enabled by setting the correspondingly named option in {\tt NETHACKOPTIONS}
or run-time configuration file prior to game start.
In the case of {\it Blind\/} and {\it Deaf\/}, the option also enforces the conduct.
They aren't really significant accomplishments unless/until you make
substantial progress into the dungeon.

%.hn 1
\section{Options}

%.pg
Due to variations in personal tastes and conceptions of how {\it NetHack\/}
should do things, there are options you can set to change how {\it NetHack\/}
behaves.

%.hn 2
\subsection*{Setting the options}

%.pg
Options may be set in a number of ways.  Within the game, the `{\tt O}'
command allows you to view all options and change most of them.
You can also set options automatically by placing them in a configuration
file, or in the ``{\tt NETHACKOPTIONS}'' environment variable.
Some versions of {\it NetHack\/} also have front-end programs that allow
you to set options before starting the game or a global configuration
for system administrators.

%.hn 2
\subsection*{Using a configuration file}

%.pg
The default name and location of the configuration file varies on different
operating systems.\\

%.lp ""
On UNIX, Linux and macOS it is \mbox{``.nethackrc''} in the user's home
directory.
The file may not exist, but it is a normal ASCII text file and
can be created with any text editor.\\

%.lp ""
On Windows, the name is \mbox{``.nethackrc''} location in the folder
\mbox{{``\%USERPROFILE\%\textbackslash NetHack\textbackslash''}}.
The file may not exist,
but it is a normal ASCII text file and can be created with any
text editor.
After runing {\it NetHack\/} for the first time, you should find a default
template for ths configuration file named \mbox{``.nethackrc.template''} in
\mbox{{``\%USERPROFILE\%\textbackslash NetHack\textbackslash''}}.
If you have not created the configuration file, {\it NetHack\/} will create
the configuration file for you using the default template file.\\

%.lp ""
On MS-DOS it is \mbox{``defaults.nh''} in the same folder as
\mbox{{\it nethack.exe\/}}.\\

%.lp ""
Any line in the configuration file starting with `{\tt \#}' is treated
as a comment and ignored.
Empty lines are ignored.

Any line beginning with `{\tt \verb+[+}' and ending in `{\tt \verb+]+}'
is a section marker (the closing `{\tt \verb+]+}' can be followed
by whitespace and then an arbitrary comment beginning with `{\tt \#}').
The text between the square brackets is the section name.
Section markers are only valid after a CHOOSE directive and their names
are case insensitive.
Lines after a section marker belong to that section up until another
section starts or a marker without a name is encountered or the file ends.
Lines within sections are ignored unless a CHOOSE directive has selected
that section.

%.pg
You can use different configuration directives in the file, some
of which can be used multiple times.
In general, the directives are
written in capital letters, followed by an equals sign, followed by
settings particular to that directive.

%.pg
Here is a list of allowed directives:

%.lp
\blist{}
\item[\bb{OPTIONS}]
There are two types of options, boolean and compound options.
Boolean options toggle a setting on or off, while compound options
take more diverse values.
Prefix a boolean option with `no' or `!' to turn it off.
For compound options, the option name and value are separated by a colon.
Some options are persistent, and apply only to new games.
You can specify multiple OPTIONS directives, and multiple options
separated by commas in a single OPTIONS directive.
(Comma separated options are processed from right to left.)

%.lp ""
Example:
%.sd
\begin{verbatim}
    OPTIONS=dogname:Fido
    OPTIONS=!legacy,autopickup,pickup_types:$"=/!?+
\end{verbatim}
%.ed

%.lp
\item[\bb{HACKDIR}]
Default location of files {\it NetHack\/} needs. On Windows HACKDIR
defaults to the location of the {\it NetHack.exe\/} or {\it NetHackw.exe\/} file
so setting HACKDIR to override that is not usually necessary or recommended.
%.lp
\item[\bb{LEVELDIR}]
The location that in-progress level files are stored. Defaults to HACKDIR,
must be writable.
%.lp
\item[\bb{SAVEDIR}]
The location where saved games are kept. Defaults to HACKDIR, must be
writable.
%.lp
\item[\bb{BONESDIR}]
The location that bones files are kept. Defaults to HACKDIR, must be
writable.
%.lp
\item[\bb{LOCKDIR}]
The location that file synchronization locks are stored. Defaults to
HACKDIR, must be writable.
%.lp
\item[\bb{TROUBLEDIR}]
The location that a record of game aborts and self-diagnosed game problems
is kept. Defaults to HACKDIR, must be writable.
%
% config file entries beyond this point are shown alphabetically
%
%.lp
\item[\bb{AUTOCOMPLETE}]
Enable or disable an extended command autocompletion.
Autocompletion has no effect for the X11 windowport.
You can specify multiple autocompletions. To enable
autocompletion, list the extended command. Prefix the
command with ``{{\tt !}}'' to disable the autocompletion
for that command.

%.lp ""
Example:
%.sd
\begin{verbatim}
   AUTOCOMPLETE=zap,!annotate
\end{verbatim}
%.ed

%.lp
\item[\bb{AUTOPICKUP\_EXCEPTION}]
Set exceptions to the {{\it pickup\_types\/}}
option. See the ``Configuring Autopickup Exceptions'' section.
%.lp
\item[\bb{BINDINGS}]
Change the key bindings of some special keys, menu accelerators,
extended commands, or mouse buttons. You can specify multiple bindings.
Format is key followed by the command, separated by a colon.
See the ``Changing Key Bindings`` section for more information.

%.lp ""
Example:
%.sd
\begin{verbatim}
   BIND=^X:getpos.autodescribe
\end{verbatim}
%.ed

%.lp
\item[\bb{CHOOSE}]
Chooses at random one of the comma-separated parameters as an active
section name.
Lines in other sections are ignored.

%.lp ""
Example:
%.sd
\begin{verbatim}
   OPTIONS=color
   CHOOSE=char A,char B
   [char A]
   OPTIONS=role:arc,race:dwa,align:law,gender:fem
   [char B]
   OPTIONS=role:wiz,race:elf,align:cha,gender:mal
   [] #end of CHOOSE
   OPTIONS=!rest_on_space
\end{verbatim}
%.ed

%.lp ""
If {\tt []} is present, the preceding section is closed and no new
section begins; whatever follows will be common to all sections.
Otherwise the last section extends to the end of the options file.

%.lp
\item[\bb{MENUCOLOR}]
Highlight menu lines with different colors.
See the ``Configuring Menu Colors`` section.
%.lp
\item[\bb{MSGTYPE}]
Change the way messages are shown in the top status line.
See the ``Configuring Message Types`` section.
%.lp
\item[\bb{ROGUESYMBOLS}]
Custom symbols for for the rogue level's symbol set.
See {\it SYMBOLS} below.
%.lp
\item[\bb{SOUND}]
Define a sound mapping.
See the ``Configuring User Sounds'' section.
%.lp
\item[\bb{SOUNDDIR}]
Define the directory that contains the sound files.
See the ``Configuring User Sounds'' section.
%.lp
\item[\bb{SYMBOLS}]
Override one or more symbols in the symbol set used for all dungeon
levels except for the special rogue level.
See the ``Modifying {\it NetHack\/} Symbols'' section.
%.pg

%.lp ""
Example:
%.sd
\begin{verbatim}
   # replace small punctuation (tick marks) with digits
   SYMBOLS=S_boulder:0,S_golem:7
\end{verbatim}
%.ed

%.lp
\item[\bb{WIZKIT}]
Debug mode only:  extra items to add to initial inventory.
Value is the name of a text file containing a list of item names,
one per line, up to a maximum of 128 lines.
Each line is processed by the function that handles wishing.

%.lp ""
Example:
%.sd
\begin{verbatim}
   WIZKIT=~/wizkit.txt
\end{verbatim}
%.ed
\elist

%.lp ""
%.pg
Here is an example of configuration file contents:
%.sd
\begin{verbatim}
    # Set your character's role, race, gender, and alignment.
    OPTIONS=role:Valkyrie, race:Human, gender:female, align:lawful

    # Turn on autopickup, set automatically picked up object types
    OPTIONS=autopickup,pickup_types:$"=/!?+

    # Map customization
    OPTIONS=color           # Display things in color if possible
    OPTIONS=lit_corridor    # Show lit corridors differently
    OPTIONS=hilite_pet,hilite_pile
    # Replace small punctuation (tick marks) with digits
    SYMBOLS=S_boulder:0,S_golem:7

    # No startup splash screen. Windows GUI only.
    OPTIONS=!splash_screen
\end{verbatim}
%.ed
%.BR 2

%.hn 2
\subsection*{Using the {\tt NETHACKOPTIONS} environment variable}

%.pg
The NETHACKOPTIONS variable is a comma-separated list of initial
values for the various options.  Some can only be turned on or off.
You turn one of these on by adding the name of the option to the list,
and turn it off by typing a `{\tt !}' or ``{\tt no}'' before the name.
Others take a
character string as a value.  You can set string options by typing
the option name, a colon or equals sign, and then the value of the string.
The value is terminated by the next comma or the end of string.

%.pg
For example, to set up an environment variable so that
{\it color\/} is {\tt on},
{\it legacy\/} is {\tt off},
character {\it name\/} is set to ``{\tt Blue Meanie}'',
and named {\it fruit\/} is set to ``{\tt lime}'',
you would enter the command
%.SD i
\begin{verbatim}
    setenv NETHACKOPTIONS "color,\!leg,name:Blue Meanie,fruit:lime"
\end{verbatim}
%.ED

\nd in {\it csh}
(note the need to escape the `!' since it's special
to that shell), or the pair of commands
%.SD i
\begin{verbatim}
    NETHACKOPTIONS="color,!leg,name:Blue Meanie,fruit:lime"
    export NETHACKOPTIONS
\end{verbatim}
%.ED

\nd in {\it sh}, {\it ksh}, or {\it bash}.

%.pg
The NETHACKOPTIONS value is effectively the same as a single OPTIONS
directive in a configuration file.
The ``OPTIONS='' prefix is implied and comma separated options are
processed from right to left.
Other types of configuration directives such as BIND or MSGTYPE are
not allowed.

%.pg
Instead of a comma-separated list of options,
NETHACKOPTIONS can be set to the full name of a configuration file you
want to use.
If that full name doesn't start with a slash, precede it with `{\tt @}'
(at-sign) to let NetHack know that the rest is intended as a file name.
If it does start with `{\tt /}', the at-sign is optional.

%.hn 2
\subsection*{Customization options}

%.pg
Here are explanations of what the various options do.
Character strings that are too long may be truncated.
Some of the options listed may be inactive in your dungeon.

%.pg
Some options are persistent, and are saved and reloaded along with
the game.  Changing a persistent option in the configuration file
applies only to new games.

\blist{}
%.lp
\item[\ib{acoustics}]
Enable messages about what your character hears (default on).
Note that this has nothing to do with your computer's audio capabilities.
Persistent.
%.lp
\item[\ib{alignment}]
Your starting alignment ({\tt align:lawful}, {\tt align:neutral},
or {\tt align:chaotic}).
You may specify just the first letter.
Many roles and the non-human races restrict which alignments are allowed.
See {\it role\/}
for a description of how to use negation to exclude choices.
%.lp ""
\\
Default is random.
Cannot be set with the `{\tt O}' command.  Persistent.
%.lp
\item[\ib{autodescribe}]
Automatically describe the terrain under cursor when asked to get a location
on the map (default true).
The {\it whatis\verb+_+coord\/}
option controls whether the description includes map coordinates.
%.lp
\item[\ib{autodig}]
Automatically dig if you are wielding a digging tool and moving into a place
that can be dug (default false).  Persistent.
%.lp
\item[\ib{autoopen}]
Walking into a closed door attempts to open it (default true).
Persistent.
%.lp
\item[\ib{autopickup}]
Automatically pick up things onto which you move (default off).
Persistent.
\\
%.lp ""
See ``{\it pickup\verb+_+types\/}'' and also
``{\it autopickup\verb+_+exception\/}'' for ways to refine the behavior.
\\
%.lp ""
Note: prior to version 3.7.0, the default for {\it autopickup\/} was {\it on}.
%.lp
\item[\ib{autoquiver}]
This option controls what happens when you attempt the `{\tt f}' (fire)
command when nothing is quivered or readied (default false).
When true, the computer will fill
your quiver or quiver sack or make ready some suitable weapon.
Note that it will not take
into account the blessed/cursed status, enchantment, damage, or
quality of the weapon; you are free to manually fill your quiver
or quiver sack or make ready
with the `{\tt Q}' command instead.
If no weapon is found or the option is
false, the `{\tt t}' (throw) command is executed instead.  Persistent.
%.lp
\item[\ib{autounlock}]
%\hyphenation{apply\-key}%this needs to be tested...
Controls what action to take when attempting to walk into a locked door
or to loot a locked container.
Takes a plus-sign separated list of values:
% paranoid
% au => autounlock
\newlength{\auwidth}
%.PS Apply-Key
\settowidth{\auwidth}{\tt Apply-Key}
\addtolength{\auwidth}{\labelsep}
\blist{\leftmargin \auwidth \topsep 1mm \itemsep 0mm}
%.PL Untrap
\item[{\tt Untrap}]
prompt about whether to attempt to find a trap;
it might fail to find one even when present; if it does find one, it
will ask whether you want to try to disarm the trap; if you decline,
your character will forget that the door or box is trapped;
%.PL Apply-Key
\item[{\tt Apply-Key}]
if carrying a key or other unlocking tool, prompt about using it;
%.PL Kick
\item[{\tt Kick}]
kick the door (if you omit untrap or decline to attempt untrap and
you omit apply-key or you lack a key or you decline to use the key;
has no effect on containers);
%.PL Force
\item[{\tt Force}]
try to force a container's lid with your currently
wielded weapon (if you omit untrap or decline to attempt untrap and
you omit apply-key or you lack a key or you decline to use the key;
has no effect on doors);
%.PL None
\item[{\tt None}]
none of the above; can't be combined with the other choices.
%.PE
\elist
Omitting the value is treated as if {\tt autounlock:apply-key}.
Preceding {\tt autounlock} with `{\tt !}' or ``{\tt no}'' is treated as
{\tt autounlock:none}.
\\
%.lp ""
Applying a key might set off a trap if the door or container is trapped.
Successfully kicking a door will break it and wake up nearby monsters.
Successfully forcing a container open will break its lock and might also
destroy some of its contents or damage your weapon or both.
\\
%.lp ""
The default is Apply-Key.
Persistent.
%.lp
\item[\ib{blind}]
Start the character permanently blind (default false).  Persistent.
%.lp
\item[\ib{bones}]
Allow saving and loading bones files (default true).  Persistent.
%.lp
\item[\ib{boulder}]
Set the character used to display boulders (default is the ``large rock''
class symbol, `{\tt `}').
%.lp
\item[\ib{catname}]
Name your starting cat (for example, ``{\tt catname:Morris}'').
Cannot be set with the `{\tt O}' command.
%.lp character
\item[\ib{character}]
Synonym for ``{\tt role}'' to pick the type of your character
(for example ``{\tt character:Monk}'').  See {\it role\/} for more details.
%.lp
\item[\ib{checkpoint}]
Save game state after each level change, for possible recovery after
program crash (default on).  Persistent.
%.lp
\item[\ib{cmdassist}]
Have the game provide some additional command assistance for new
players if it detects some anticipated mistakes (default on).
%.lp
\item[\ib{confirm}]
Have user confirm attacks on pets, shopkeepers, and other
peaceable creatures (default on).  Persistent.
%.lp
\item[\ib{dark\verb+_+room}]
Show out-of-sight areas of lit rooms (default on).  Persistent.
%.lp
\item[\ib{deaf}]
Start the character permanently deaf (default false).  Persistent.
%.lp
\item[\ib{disclose}]
Controls what information the program reveals when the game ends.
Value is a space separated list of prompting/category pairs
(default is `{\tt ni na nv ng nc no}',
prompt with default response of `{\tt n}' for each candidate).
Persistent.
The possibilities are:

%.sd
%.si
{\tt i} --- disclose your inventory;\\
{\tt a} --- disclose your attributes;\\
{\tt v} --- summarize monsters that have been vanquished;\\
{\tt g} --- list monster species that have been genocided;\\
{\tt c} --- display your conduct; also achievements, if any;\\
{\tt o} --- display dungeon overview.
%.ei
%.ed

Each disclosure possibility can optionally be preceded by a prefix which
lets you refine how it behaves.  Here are the valid prefixes:

%.sd
%.si
{\tt y} --- prompt you and default to yes on the prompt;\\
{\tt n} --- prompt you and default to no on the prompt;\\
{\tt +} --- disclose it without prompting;\\
{\tt -} --- do not disclose it and do not prompt.
%.ei
%.ed

The listing of vanquished monsters can be sorted,
so there are two additional choices for `{\tt v}':
The listings of vanquished monsters and of genocided types can be sorted,
so there are two additional choices for `q' and `g':
%.sd
%.si
{\tt ?} --- prompt you and default to ask on the prompt;\\
{\tt\#} --- disclose it without prompting, ask for sort order.
%.ei
%.ed

Asking refers to picking one of the orderings from a menu.
The `{\tt +}' disclose without prompting choice,
or being prompted and answering `{\tt y}' rather than `{\tt a}',
will default to showing monsters in the order specified by the
{\it sortvanquished\/} option.
\\
%.lp ""
Omitted categories are implicitly added with `{\tt n}' prefix.
Specified categories with omitted prefix implicitly use `{\tt +}' prefix.
Order of the disclosure categories does not matter, program display for
end-of-game disclosure follows a set sequence.

%.lp ""
(for example, ``{\tt disclose:yi na +v -g o}'')
The example sets
{\tt inventory} to {\it prompt\/} and default to {\it yes\/},
{\tt attributes} to {\it prompt\/} and default to {\it no\/},
{\tt vanquished} to {\it disclose without prompting\/},
{\tt genocided} to {\it not disclose\/} and {\it not prompt\/},
{\tt conduct} to implicitly {\it prompt\/} and default to {\it no\/},
{\tt overview} to {\it disclose without prompting\/}.

%.lp ""
Note that the vanquished monsters list includes all monsters killed by
traps and each other as well as by you.
And the dungeon overview shows all levels you had visited but does not
reveal things about them that you hadn't discovered.
%.lp
\item[\ib{dogname}]
Name your starting dog (for example, ``{\tt dogname:Fang}'').
Cannot be set with the `{\tt O}' command.
%.lp
\item[\ib{extmenu}]
Changes the extended commands interface to pop-up a menu of available commands.
It is keystroke compatible with the traditional interface except that it does
not require that you hit Enter.
It is implemented for the tty interface (default off).
.lp ""
For the X11 interface, which always uses a menu for choosing an extended
command, it controls whether the menu shows all available commands (on)
or just the subset of commands which have traditionally been considered
extended ones (off).
%.lp
\item[\ib{female}]
An obsolete synonym for ``{\tt gender:female}''.  Cannot be set with the
`{\tt O}' command.
%.lp
\item[\ib{fireassist}]
This option controls what happens when you attempt the `{\tt f}' (fire)
and don't have an appropriate launcher, such as a bow or a sling, wielded.
If on, you will automatically wield the launcher. Default is on.
%.lp
\item[\ib{fixinv}]
An object's inventory letter sticks to it when it's dropped (default on).
If this is off, dropping an object shifts all the remaining inventory letters.
Persistent.
%.lp
\item[\ib{force\_invmenu}]
Commands asking for an inventory item show a menu instead of
a text query with possible menu letters. Default is off.
%.lp
\item[\ib{fruit}]
Name a fruit after something you enjoy eating (for example, ``{\tt fruit:mango}'')
(default ``{\tt slime mold}''). Basically a nostalgic whimsy that
{\it NetHack\/} uses from time to time.  You should set this to something you
find more appetizing than slime mold.  Apples, oranges, pears, bananas, and
melons already exist in {\it NetHack\/}, so don't use those.
%.lp
\item[\ib{gender}]
Your starting gender ({\tt gender:male} or {\tt gender:female}).
You may specify just the first letter.
Although you can
still denote your gender using either of the deprecated
``{\it male\/}'' and ``{\it female\/}''
options, the ``{\it gender\/}'' option will take precedence.
See {\it role\/}
for a description of how to use negation to exclude choices.
%.lp ""
\\
Default is random.
Cannot be set with the `{\tt O}' command.  Persistent.
%.lp
\item[\ib{goldX}]
When filtering objects based on bless/curse state (BUCX), whether to
treat gold pieces as {\tt X} (unknown bless/curse state, when `on')
or {\tt U} (known to be uncursed, when `off', the default).
Gold is never blessed or cursed, but it is not described as ``uncursed''
even when the {\it implicit\verb+_+uncursed\/} option is `off'.
%.lp
\item[\ib{help}]
If more information is available for an object looked at
with the `{\tt /}' command, ask if you want to see it (default on).
Turning help off makes just looking at things faster, since you aren't
interrupted with the ``{\tt More info?}'' prompt, but it also means that you
might miss some interesting and/or important information.  Persistent.
%.lp
\item[\ib{herecmd\verb+_+menu}]
When using a windowport that supports mouse and clicking on yourself or
next to you, show a menu of possible actions for the location.
Same as ``{\tt \#herecmdmenu}'' and ``{\tt \#therecmdmenu}'' commands.
%.lp
\item[\ib{hilite\verb+_+pet}]
Visually distinguish pets from similar animals (default off).
The behavior of this option depends on the type of windowing you use.
In text windowing, text highlighting or inverse video is often used;
with tiles, generally displays a heart symbol near pets.

%.lp ""
With the curses interface, the {\it petattr\/}
option controls how to highlight pets and setting it will turn the
{\it hilite\verb+_+pet\/} option on or off as warranted.
%.lp
\item[\ib{hilite\verb+_+pile}]
Visually distinguish piles of objects from individual objects (default off).
The behavior of this option depends on the type of windowing you use.
In text windowing, text highlighting or inverse video is often used;
with tiles, generally displays a small plus-symbol beside the object
on the top of the pile.
%.lp
\item[\ib{hitpointbar}]
Show a hit point bar graph behind your name and title.
Only available for TTY and Windows GUI, and only when statushilites is on.
%.lp
\item[\ib{horsename}]
Name your starting horse (for example, ``{\tt horsename:Trigger}'').
Cannot be set with the `{\tt O}' command.
%.lp
\item[\ib{ignintr}]
Ignore interrupt signals, including breaks (default off).  Persistent.
%.lp
\item[\ib{implicit\verb+_+uncursed}]
Omit ``uncursed'' from object descriptions when it can be deduced from
other aspects of the description (default on).
Persistent.

%.lp ""
If you use menu coloring, you may want to turn this off.
%.lp
\item[\ib{legacy}]
Display an introductory message when starting the game (default on).
Persistent.
%.lp
\item[\ib{lit\verb+_+corridor}]
Show corridor squares seen by night vision or a light source held by your
character as lit (default off).  Persistent.
%.lp
\item[\ib{lootabc}]
When using a menu to interact with a container,
use the old `{\tt a}', `{\tt b}', and `{\tt c}' keyboard shortcuts
rather than the mnemonics `{\tt o}', `{\tt i}', and `{\tt b}' (default off).
Persistent.
%.lp
\item[\ib{mail}]
Enable mail delivery during the game (default on).  Persistent.
%.lp
\item[\ib{male}]
An obsolete synonym for ``{\tt gender:male}''.  Cannot be set with the
`{\tt O}' command.
%.lp
\item[\ib{mention\verb+_+decor}]
Give feedback when walking onto various dungeon features such as stairs,
fountains, or altars which are ordinarily only described when covered
by one or more objects (default off).  Persistent.
%.lp
\item[\ib{mention\verb+_+walls}]
Give feedback when walking against a wall (default off).  Persistent.
%.lp
\item[\ib{menucolors}]
Enable coloring menu lines (default off).
See ``{\it Configuring Menu Colors\/}'' on how to configure the colors.
%.lp
\item[\ib{menustyle}]
Controls the interface used when you need to choose various objects (in
response to the Drop command, for instance).
The value specified should be the first letter of one of the following:
traditional, combination, full, or partial.
Persistent.
\\
%.lp ""
Traditional was the only interface available for very
early versions; it consists of a prompt for object class characters,
followed by an object-by-object prompt for all items matching the selected
object class(es).
Combination starts with a prompt for object class(es)
of interest, but then displays a menu of matching objects rather than
prompting one-by-one.
Full displays a menu of
object classes rather than a character prompt, and then a menu of matching
objects for selection.
Partial skips the object class filtering and
immediately displays a menu of all objects.
\item[\ib{menu\verb+_+deselect\verb+_+all}]
Key to deselect all items in a menu.
Implemented by the Amiga, Gem, X11 and tty ports.
Default `-'.
\item[\ib{menu\verb+_+deselect\verb+_+page}]
Key to deselect all items on this page of a menu.
Implemented by the Amiga, Gem and tty ports.
Default `\verb+\+'.
\item[\ib{menu\verb+_+first\verb+_+page}]
Key to jump to the first page in a menu.
Implemented by the Amiga, Gem and tty ports.
Default `\verb+^+'.
\item[\ib{menu\verb+_+headings}]
Controls how the headings in a menu are highlighted.
Values are ``{\tt none}'', ``{\tt bold}'', ``{\tt dim}'',
``{\tt italic}'', ``{\tt underline}'',``{\tt blink}'', or
``{\tt inverse}''.
Not all ports can actually display all types.
\item[\ib{menu\verb+_+invert\verb+_+all}]
Key to invert all items in a menu.
Implemented by the Amiga, Gem, X11 and tty ports.
Default `@'.
\item[\ib{menu\verb+_+invert\verb+_+page}]
Key to invert all items on this page of a menu.
Implemented by the Amiga, Gem and tty ports.
Default `\verb+~+'.
\item[\ib{menu\verb+_+last\verb+_+page}]
Key to jump to the last page in a menu.
Implemented by the Amiga, Gem and tty ports.
Default `\verb+|+'.
\item[\ib{menu\verb+_+next\verb+_+page}]
Key to go to the next menu page.
Implemented by the Amiga, Gem and tty ports.
Default `\verb+>+'.
\item[\ib{menu\verb+_+objsyms}]
Show object symbols in menu headings in menus where
the object symbols act as menu accelerators (default off).
\item[\ib{menu\verb+_+overlay}]
Do not clear the screen before drawing menus, and align
menus to the right edge of the screen. Only for the tty port.
(default on)
\item[\ib{menu\verb+_+previous\verb+_+page}]
Key to go to the previous menu page.
Implemented by the Amiga, Gem and tty ports.
Default `\verb+<+'.
\item[\ib{menu\verb+_+search}]
Key to search for some text and toggle selection state of matching menu items.
Default `:'.
\item[\ib{menu\verb+_+select\verb+_+all}]
Key to select all items in a menu.
Implemented by the Amiga, Gem, X11 and tty ports.
Default `.'.
\item[\ib{menu\verb+_+select\verb+_+page}]
Key to select all items on this page of a menu.
Implemented by the Amiga, Gem and tty ports.
Default `,'.

%.lp
\item[\ib{menu\verb+_+shift\verb+_+left}]
Key to scroll a menu---one which has been
scrolled right---back to the left.
Implemented for {\it perm\verb+_+invent\/} only by curses and X11.
Default `{\tt \verb+{+}'.

%.lp
\item[\ib{menu\verb+_+shift\verb+_+right}]
Key to scroll a menu which has text beyond the
right edge to the right.
Implemented for {\it perm\verb+_+invent\/} only by curses by X11.
Default `{\tt \verb+}+}'.
% %.lp
% \item[\ib{menu\verb+_+tab\verb+_+sep}]
% Format menu entries using TAB to separate columns (default off).
% Only applicable to some menus, and only useful to some interfaces.
% Debug mode only.
%.lp
\item[\ib{monpolycontrol}]
Prompt for new form whenever any monster changes shape (default off).
Debug mode only.
%.lp
\item[\ib{mouse\verb+_+support}]
Allow use of the mouse for input and travel.
Valid settings are:

%.sd
%.si
{\tt 0} --- disabled\\
{\tt 1} --- enabled and make OS adjustments to support mouse use\\
{\tt 2} --- like {\tt 1}, but does not make any OS adjustments\\
%.ei
%.ed

Omitting a value is the same as specifying {\tt 1}
and negating
{\it mouse\verb+_+support\/}
is the same as specifying {\tt 0}.
%.lp
\item[\ib{msghistory}]
The number of top line messages to save (and be able to recall
with `{\tt \^{}P}') (default 20).
Cannot be set with the `{\tt O}' command.
%.lp
\item[\ib{msg\verb+_+window}]
Allows you to change the way recalled messages are displayed.
Currently it is only supported for tty (all four choices) and for curses
(`{\tt f}' and `{\tt r}' choices, default `{\tt r}').
The possible values are:

%.sd
%.si
{\tt s} --- single message (default; only choice prior to 3.4.0);\\
{\tt c} --- combination, two messages as {\it single\/}, then as {\it full\/};\\
{\tt f} --- full window, oldest message first;\\
{\tt r} --- full window reversed, newest message first.
%.ei
%.ed

For backward compatibility, no value needs to be specified (which
defaults to {\it full\/}), or it can be negated (which defaults
to {\it single\/}).
%.lp
\item[\ib{name}]
Set your character's name (defaults to your user name).  You can also
set your character's role by appending a dash and one or more letters of
the role (that is, by suffixing one of
``{\tt -A -B -C -H -K -M -P -Ra -Ro -S -T -V -W}'').
If ``{\tt -@}'' is used for the role, then a random one will be
automatically chosen.
Cannot be set with the `{\tt O}' command.
%.lp
\item[\ib{news}]
Read the {\it NetHack\/} news file, if present (default on).
Since the news is shown at the beginning of the game, there's no point
in setting this with the `{\tt O}' command.
%.lp
\item[\ib{nudist}]
Start the character with no armor (default false).  Persistent.
%.lp
\item[\ib{null}]
Send padding nulls to the terminal (default on).  Persistent.
%.lp
\item[\ib{number\verb+_+pad}]
Use digit keys instead of letters to move (default 0 or off).\\
Valid settings are:

%.sd
%.si
\newlength{\mwidth}
\settowidth{\mwidth}{\tt -0}
\newcommand{\numbox}[1]{\makebox[\mwidth][r]{{\tt #1}}}
\numbox{0} --- move by letters; `{\tt yuhjklbn}'\\
\numbox{1} --- move by numbers; digit `{\tt 5}' acts as `{\tt G}' movement prefix\\
\numbox{2} --- like {\tt 1} but `{\tt 5}' works as `{\tt g}' prefix instead of as `{\tt G}'\\
\numbox{3} --- by numbers using phone key layout; {\tt 123} above, {\tt 789} below\\
\numbox{4} --- combines {\tt 3} with {\tt 2}; phone layout plus MS-DOS compatibility\\
\numbox{-1} --- by letters but use `{\tt z}' to go northwest, `{\tt y}' to zap wands
%.ei
%.ed

For backward compatibility, omitting a value is the same as specifying {\tt 1}
and negating
{\it number\verb+_+pad\/}
is the same as specifying {\tt 0}.
(Settings {\tt 2} and {\tt 4} are for compatibility with MS-DOS or old PC Hack;
in addition to the different behavior for `{\tt 5}', `{\tt Alt-5}' acts as `{\tt G}'
and `{\tt Alt-0}' acts as `{\tt I}'.
Setting {\tt -1} is to accommodate some QWERTZ keyboards which have the
location of the `{\tt y}' and `{\tt z}' keys swapped.)
When moving by numbers, to enter a count prefix for those commands
which accept one (such as ``{\tt 12s}'' to search twelve times), precede it
with the letter `{\tt n}' (``{\tt n12s}'').
%.lp
\item[\ib{packorder}]
Specify the order to list object types in (default
``\verb&")[%?+!=/(*`0_&''). The value of this option should be a string
containing the symbols for the various object types.  Any omitted types
are filled in at the end from the previous order.
%.lp
\item[\ib{paranoid\verb+_+confirmation}]
A space separated list of specific situations where alternate
prompting is desired.  The default is ``{\it paranoid\verb+_+confirmation:pray swim}''.
%.sd
%.si
\newlength{\pcwidth}
\settowidth{\pcwidth}{\tt Were-change}
\addtolength{\pcwidth}{\labelsep}
\blist{\leftmargin \pcwidth \topsep 1mm \itemsep 0mm}
\item[{\tt Confirm}]
for any prompts which are set to require ``yes''
rather than `y', also require ``no'' to reject instead
of accepting any non-yes response as no;
\item[{\tt quit~~~}]
require ``{\tt yes}'' rather than `{\tt y}' to confirm quitting
the game or switching into non-scoring explore mode;
\item[{\tt die~~~~}]
require ``{\tt yes}'' rather than `{\tt y}' to confirm dying (not
useful in normal play; applies to explore mode);
\item[{\tt bones~~}]
require ``{\tt yes}'' rather than `{\tt y}' to confirm saving
bones data when dying in debug mode
\item[{\tt attack~}]
require ``{\tt yes}'' rather than `{\tt y}' to confirm attacking
a peaceful monster;
\item[{\tt wand-break}]
require ``{\tt yes}'' rather than `{\tt y}' to confirm breaking
a wand;
\item[{\tt eating}]
require ``{\tt yes}'' rather than `{\tt y}' to confirm whether to
continue eating;
\item[{\tt Were-change}]
require ``{\tt yes}'' rather than `{\tt y}' to confirm changing form
due to lycanthropy
when hero has polymorph control;
\item[{\tt pray~~~}]
require `{\tt y}' to confirm an attempt to pray rather
than immediately praying; on by default;
\item[{\tt Remove~}] require selection from inventory for `{\tt R}'
and `{\tt T}'
commands even when wearing just one applicable item.
\item[{\tt swim~~~}]
prevent walking into water or lava.
\item[{\tt all~~~~}]
turn on all of the above.
\elist
%.ei
%.ed
By default, the pray choice is enabled, the others disabled.
To disable it without setting
any of the other choices, use ``{\it paranoid\verb+_+confirmation:none}''.  To keep
it enabled while setting any of the others, include it in the list,
such as ``{\it par\-a\-noid\verb+_+con\-fir\-ma\-tion:attack~pray~Remove}''.
%.lp
\item[\ib{perm\verb+_+invent}]
If true, always display your current inventory in a window.  This only
makes sense for windowing system interfaces that implement this feature.
%.lp
%.\" petattr is a wincap option but we'll document it here...
\item[\ib{petattr}]
Specifies one or more text highlighting attributes to use when showing
pets on the map.
Effectively a superset of the {\it hilite\verb+_+pet\/} boolean option.
Curses interface only; value is one or more of the following letters.

%.sd
%.si
{\tt n} --- Normal text (no highlighting)\\
{\tt i} --- Inverse video (default)\\
{\tt b} --- Bold text\\
{\tt u} --- Underlined text\\
{\tt k} --- blinKing text\\
{\tt d} --- Dim text\\
{\tt t} --- iTalic text\\
{\tt l} --- Left line indicator\\
{\tt r} --- Right line indicator\\
%.ei
%.ed

Some of those choices might not work, particularly the final three,
depending upon terminal hardware or terminal emulation software.

%.lp ""
Currently multiple highlight-style letters can be combined by simply
stringing them together (for example, ``bk''), but in the future
they might require being separated by plus signs (such as ``b\verb&+&k'',
which works already).
When using the `n' choice, it should be specified on its own,
not in combination with any of the other letters.

%.lp
\item[\ib{pettype}]
Specify the type of your initial pet, if you are playing a character class
that uses multiple types of pets; or choose to have no initial pet at all.
Possible values are ``{\tt cat}'', ``{\tt dog}'', ``{\tt horse}''
and ``{\tt none}''.
If the choice is not allowed for the role you are currently playing,
it will be silently ignored.  For example, ``{\tt horse}'' will only be
honored when playing a knight.
Cannot be set with the `{\tt O}' command.
%.lp
\item[\ib{pickup\verb+_+burden}]
When you pick up an item that would exceed this encumbrance
level (Unencumbered, Burdened, streSsed, straiNed, overTaxed,
or overLoaded), you will be asked if you want to continue.
(Default `S').  Persistent.
%.lp
\item[\ib{pickup\verb+_+thrown}]
If this option is on and ``{\it autopickup\/}'' is also on, try to pick up
things that you threw, even if they aren't in
``{\it pickup\verb+_+types\/}'' or
match an autopickup exception.
Default is on.
Persistent.
%.lp
\item[\ib{pickup\verb+_+types}]
Specify the object types to be picked up when ``{\it autopickup\/}''
is on.
Default is all types.
Persistent.
\\
%.lp ""
The value is a list of object symbols, such as
{\tt \verb&pickup_types:$?!&} to pick up gold, scrolls, and potions.
You can use
``{\it autopickup\verb+_+exception\/}''
configuration file lines to further refine ``{\it autopickup\/}'' behavior.
\\
%.lp ""
There is no way to set {\it pickup\verb+_+types\/} to ``{\it none}''.
(Setting it to an empty value reverts to ``{\it all}''.)
If you want to avoid automatically picking up any types of items but do
want to have {\it autopickup\/} on in order to have
{\it autopickup\verb+_+exceptions\/} control what you do and don't pick
up, you can set {\it pickup\verb+_+types\/} to `{\tt .}'.
That is the type symbol for {\it venom\/} and you won't come across
any venom items so won't unintentionally pick such up.
%.lp
\item[\ib{pile\verb+_+limit}]
When walking across a pile of objects on the floor, threshold at which
the message ``there are few/several/many objects here'' is given instead
of showing a popup list of those objects.  A value of 0 means ``no limit''
(always list the objects); a value of 1 effectively means ``never show
the objects'' since the pile size will always be at least that big;
default value is 5.  Persistent.
%.lp
\item[\ib{playmode}]
Values are {\it normal\/}, {\it explore\/}, or {\it debug\/}.
Allows selection of explore mode (also known as discovery mode) or debug
mode (also known as wizard mode) instead of normal play.
Debug mode might only be allowed for someone logged in under a particular
user name (on multi-user systems) or specifying a particular character
name (on single-user systems) or it might be disabled entirely.  Requesting
it when not allowed or not possible results in explore mode instead.
Default is normal play.
%.lp
\item[\ib{pushweapon}]
Using the `{\tt w}' (wield) command when already wielding
something pushes the old item into your alternate weapon slot (default off).
Likewise for the `{\tt a}' (apply) command if it causes the applied item to
become wielded.  Persistent.
%.lp
\item[\ib{quick\verb+_+farsight}]
When set, usually prevents the ``you sense your surroundings'' message
where play pauses to allow you to browse the map whenever clairvoyance
randomly activates.
Some situations, such as being underwater or engulfed, ignore this option.
It does not affect the clairvoyance spell where pausing to examine revealed
objects or monsters is less intrusive.
Default is off.  Persistent.
%.lp
\item[\ib{race}]
Choices are {\tt human}, {\tt dwarf}, {\tt elf}, {\tt gnome}, and
{\tt orc} but most roles restrict which of the non-human races are allowed.
See {\it role\/}
for a description of how to use negation to exclude choices.
%.lp ""
\\
Default is random.
Cannot be set with the `{\tt O}' command.  Persistent.
%.lp
\item[\ib{rest\verb+_+on\verb+_+space}]
Make the space bar a synonym for the `{\tt .}' (\#wait) command (default off).
Persistent.
%.lp
\item[\ib{role}]
Pick your type of character (for example, ``{\tt role:Samurai}'');
synonym for ``{\it character\/}''.
See ``{\it name\/}'' for an alternate method of specifying your role.
%.\" Normally only the first letter of the
%.\" value is examined; `r' is an exception with ``{\tt Rogue}'',
%.\" ``{\tt Ranger}'', and ``{\tt random}'' values.
%.lp ""
This option can also be used to limit selection when role is chosen
randomly.
Use a space-separated list of roles and either negate each one or negate
the option itself instead.
Negation is accomplished in the same manner as with {\it boolean options\/},
by prefixing the option or its value(s) with `{\tt \verb+!+}' or ``{\tt no}''.
%.BR 0
\\
Examples:
\\
\begin{verbatim}
OPTIONS=role:!arc !bar !kni
OPTIONS=!role:arc bar kni
\end{verbatim}
There can be multiple instances of the {\it role\/}
option if they're all negations.
%.\" Only one positive value is allowed, and if present, it overrides any
%.\" negations.
%.lp ""
\\
Default is random.
Cannot be set with the `{\tt O}' command.  Persistent.
%.lp
\item[\ib{roguesymset}]
This option may be used to select one of the named symbol sets found within
{\tt symbols} to alter the symbols displayed on the screen on the
rogue level.
%.lp
\item[\ib{rlecomp}]
When writing out a save file, perform run length compression of the map.
Not all ports support run length compression. It has no
effect on reading an existing save file.
%.lp
\item[\ib{runmode}]
Controls the amount of screen updating for the map window when engaged
in multi-turn movement (running via {\tt shift}+direction
or {\tt control}+direction
and so forth, or via the travel command or mouse click).
The possible values are:

%.sd
%.si
{\tt teleport} --- update the map after movement has finished;\\
{\tt run} --- update the map after every seven or so steps;\\
{\tt walk} --- update the map after each step;\\
{\tt crawl} --- like {\it walk\/}, but pause briefly after each step.
%.ei
%.ed

This option only affects the game's screen display, not the actual
results of moving.  The default is {\it run\/}; versions prior to 3.4.1
used {\it teleport\/} only.  Whether or not the effect is noticeable will
depend upon the window port used or on the type of terminal.  Persistent.
%.lp
\item[\ib{safe\verb+_+pet}]
Prevent you from (knowingly) attacking your pets (default on).  Persistent.
%.lp
\item[\ib{safe\verb+_+wait}]
Prevents you from waiting or searching when next to a hostile monster
(default on).  Persistent.
%.lp
\item[\ib{sanity\verb+_+check}]
Evaluate monsters, objects, and map prior to each turn (default off).
Debug mode only.
%.lp
\item[\ib{scores}]
Control what parts of the score list you are shown at the end (for example,
``{\tt scores:5top scores/4around my score/own scores}'').  Only the first
letter of each category (`{\tt t}', `{\tt a}' or `{\tt o}') is necessary.
Persistent.
%.lp
\item[\ib{showexp}]
Show your accumulated experience points on bottom line (default off).
Persistent.
%.lp
\item[\ib{showrace}]
Display yourself as the glyph for your race, rather than the glyph
for your role (default off).  Note that this setting affects only
the appearance of the display, not the way the game treats you.
Persistent.
%.lp
\item[\ib{showscore}]
Show your approximate accumulated score on bottom line (default off).
Persistent.
%.lp
\item[\ib{silent}]
Suppress terminal beeps (default on).  Persistent.
%.lp
\item[\ib{sortdiscoveries}]
Controls the sorting behavior for the output of the `{\tt $\backslash$}'
and `{\tt \`{}}' commands.
Persistent.
\\
%.lp ""
The possible values are:
\\
%.PS
{\tt o} --- list object types by class, in discovery order within each class;
default;
\\
{\tt s} --- list object types by {\it sortloot\/}
classification: by class, by sub-class within class for classes which
have substantial groupings (like helmets, boots, gloves, and so forth
for armor), with object types partly-discovered via assigned name coming
before fully identified types;
\\
{\tt c} --- list by class, alphabetically within each class;\\
{\tt a} --- list alphabetically across all classes.\\
%.PE
Can be interactively set via the `{\tt O}' command or via using
the `{\tt m}' prefix before the `{\tt $\backslash$}'
or `{\tt \`{}}' command.
%.lp
\item[\ib{sortloot}]
Controls the sorting behavior of pickup lists for inventory
and \#loot commands and some others.  Persistent.
\\
The possible values are:
\\
%.sd
%.si
{\tt full} --- always sort the lists;\\
{\tt loot} --- only sort the lists that don't use inventory
       letters, like with the \#loot and pickup commands;\\
{\tt none} --- show lists the traditional way without sorting; default.
%.ei
%.ed
%.lp
\item[\ib{sortpack}]
Sort the pack contents by type when displaying inventory (default on).
Persistent.
%.lp
\item[\tb{sortvanquished}]
Controls the sorting behavior for the output of the {\tt \#vanquished} command
and also for the {\tt \#genocided} command.
Persistent.
\\
%.lp ""
The possible values are:
\\
%.PS
{\tt t} ---
traditional: order by monster level; ties are broken by internal
monster index;
default;
\\
{\tt d} ---
order by monster difficulty rating; ties broken by internal index;
\\
{\tt a} ---
order alphabetically, first any unique monsters then all the others;
\\
%note: 'A' and 'C' can be set in RC file or NETHACKOPTIONS but not by 'O'
% {\tt A} ---
% order alphabetically, unique monsters intermixed with other monsters;
% \\
% {\tt C} ---
% order by monster class, by high to low level within each class;
% \\
{\tt c} ---
order by monster class, by low to high level within each class;
\\
{\tt n} ---
order by count, high to low; ties are broken by internal monster index;
\\
{\tt z} ---
order by count, low to high; ties broken by internal index.
\\
%.PE
Can be interactively set via the `{\tt m O}' command or via using
the `{\tt m}' prefix before either the {\tt \#vanquished} command
or the {\tt \#genocided} command.
%.lp
\item[\ib{sounds}]
Allow sounds to be emitted from an integrated sound library (default on).
%.lp
\item[\ib{sparkle}]
Display a sparkly effect when a monster (including yourself) is hit by an
attack to which it is resistant (default on).  Persistent.
%.lp
\item[\ib{standout}]
Boldface monsters and ``{\tt --More--}'' (default off).  Persistent.
%.lp
\item[\ib{statushilites}]
Controls how many turns status hilite behaviors highlight
the field. If negated or set to zero, disables status hiliting.
See ``{\it Configuring Status Hilites\/}'' for further information.
%.lp
\item[\ib{status\verb+_+updates}]
Allow updates to the status lines at the bottom of the screen (default true).
%.lp
\item[\ib{suppress\verb+_+alert}]
This option may be set to a {\it NetHack\/} version level to suppress
alert notification messages about feature changes for that
and prior versions (for example, ``{\tt suppress\verb+_+alert:3.3.1}'')
%.lp
\item[\ib{symset}]
This option may be used to select one of the named symbol sets found within
{\tt symbols} to alter the symbols displayed on the screen.
Use ``{\tt symset:default}'' to explicitly select the default symbols.
%.lp
\item[\ib{time}]
Show the elapsed game time in turns on bottom line (default off).  Persistent.
%.lp
\item[\ib{timed\verb+_+delay}]
When pausing momentarily for display effect, such as with explosions and
moving objects, use a timer rather than sending extra characters to the
screen.  (Applies to ``tty'' and ``curses'' interfaces only; ``X11'' interface always
uses a timer-based delay.  The default is on if configured into the
program.)  Persistent.
%.lp
\item[\ib{tips}]
Show some helpful tips during gameplay (default on).  Persistent.
%.lp
\item[\ib{tombstone}]
Draw a tombstone graphic upon your death (default on).  Persistent.
%.lp
\item[\ib{toptenwin}]
Put the ending display in a {\it NetHack\/} window instead of on stdout (default off).
Setting this option makes the score list visible when a windowing version
of {\it NetHack\/} is started without a parent window, but it no longer leaves
the score list around after game end on a terminal or emulating window.
%.lp
\item[\ib{travel}]
Allow the travel command via mouse click (default on).
Turning this option off will prevent the game from attempting unintended
moves if you make inadvertent mouse clicks on the map window.
Does not affect traveling via the `{\tt \verb+_+}' (``{\tt \#travel}'')
command.  Persistent.
% %.lp
% \item[ib{travel\verb+_+debug}]
% Display intended path during each step of travel (default off).
% Debug mode only.
%.lp
\item[\ib{tutorial}]
Play a tutorial level at the start of the game.
Setting this option on or off in the config file will skip the query.
%.lp
\item[\ib{verbose}]
Provide more commentary during the game (default on).  Persistent.
%.lp
\item[\ib{whatis\verb+_+coord}]
When using the `{\tt /}' or `{\tt ;}' commands to look around on the map with
``{\tt autodescribe}''
on, display coordinates after the description.
Also works in other situations where you are asked to pick a location.\\

%.lp ""
The possible settings are:

%.sd
%.si
{\tt c} --- \verb#compass ('east' or '3s' or '2n,4w')#;\\
{\tt f} --- \verb#full compass ('east' or '3south' or '2north,4west')#;\\
{\tt m} --- \verb#map <x,y> (map column x=0 is not used)#;\\
{\tt s} --- \verb#screen [row,column] (row is offset to match tty usage)#;\\
{\tt n} --- \verb#none (no coordinates shown) [default]#.
%.ei
%.ed

%.lp ""
The
{\it whatis\verb+_+coord\/}
option is also used with
the `{\tt /m}', `{\tt /M}', `{\tt /o}', and `{\tt /O}' sub-commands
of `{\tt /}',
where the `{\it none\/}' setting is overridden with `{\it map}'.
%.lp
\item[\ib{whatis\verb+_+filter}]
When getting a location on the map, and using the keys to cycle through
next and previous targets, allows filtering the possible targets.
(default none)\\
%.lp ""
The possible settings are:

%.sd
%.si
{\tt n} --- \verb#no filtering#;\\
{\tt v} --- \verb#in view only#;\\
{\tt a} --- \verb#in same area (room, corridor, etc)#.
%.ei
%.ed
%.lp ""
The area-filter tries to be slightly
predictive---if
you're standing on a doorway, it will consider the area on the side of
the door you were last moving towards.\\
%.lp ""
Filtering can also be changed when getting a location with
the ``getpos.filter'' key.
%.lp
\item[\ib{whatis\verb+_+menu}]
When getting a location on the map, and using a key to cycle through
next and previous targets, use a menu instead to pick a target.
(default off)
%.lp
\item[\ib{whatis\verb+_+moveskip}]
When getting a location on the map, and using shifted movement keys or
meta-digit keys to fast-move, instead of moving 8 units at a time,
move by skipping the same glyphs.
(default off)
%.lp
\item[\ib{windowtype}]
When the program has been built to support multiple interfaces,
select whichone to use, such as ``{\tt tty}'' or ``{\tt X11}''
(default depends on build-time settings; use ``{\tt \#version}'' to check).
Cannot be set with the `{\tt O}' command.

%.lp ""
When used, it should be the first option set since its value might
enable or disable the availability of various other options.
For multiple lines in a configuration file, that would be the first
non-comment line.
For a comma-separated list in NETHACKOPTIONS or an OPTIONS line in a
configuration file, that would be the {\it rightmost\/} option in the list.
%.lp
\item[\ib{wizweight}]
Augment object descriptions with their objects' weight (default off).
Debug mode only.
%.lp
\item[\ib{zerocomp}]
When writing out a save file, perform zero-comp compression of the
contents. Not all ports support zero-comp compression. It has no effect
on reading an existing save file.
\elist

%.hn 2
\subsection*{Window Port Customization options}

%.pg
Here are explanations of the various options that are
used to customize and change the characteristics of the
windowtype that you have chosen.
Character strings that are too long may be truncated.
Not all window ports will adjust for all settings listed
here.  You can safely add any of these options to your
configuration file, and if the window port is capable of adjusting
to suit your preferences, it will attempt to do so. If it
can't it will silently ignore it.  You can find out if an
option is supported by the window port that you are currently
using by checking to see if it shows up in the Options list.
Some options are dynamic and can be specified during the game
with the `{\tt O}' command.

\blist{}
%.lp
\item[\ib{align\verb+_+message}]
 Where to align or place the message window (top, bottom, left, or right)
%.lp
\item[\ib{align\verb+_+status}]
 Where to align or place the status window (top, bottom, left, or right).
%.lp
\item[\ib{ascii\verb+_+map}]
%.hw DECgraphics IBMgraphics \% don't hyphenate these
\hyphenation{DECgraphics IBMgraphics}
If {\it NetHack\/} can, it should display the map using simple
characters (letters and punctuation) rather than {\it tiles\/} graphics.
In some cases, characters can be augmented with line-drawing symbols;
use the {\tt symset}
option to select a symbol set such as {\it DECgraphics\/}
or {\it IBMgraphics\/} if your display supports them.
Setting {\tt ascii\verb+_+map} to {\it True\/} forces
{\tt tiled\verb+_+map} to be {\it False}.
%.lp
\item[\ib{color}]
If {\it NetHack\/} can, it should display color for different monsters,
objects, and dungeon features (default on).
%.lp
\item[\ib{eight\verb+_+bit\verb+_+tty}]
If {\it NetHack\/} can, it should pass eight-bit character values (for example, specified with the
{\it traps \/} option) straight through to your terminal (default off).
%.lp
\item[\ib{font\verb+_+map}]
If {\it NetHack\/} can, it should use a font by the chosen name for the
map window.
%.lp
\item[\ib{font\verb+_+menu}]
If {\it NetHack\/} can, it should use a font by the chosen name for menu
windows.
%.lp
\item[\ib{font\verb+_+message}]
If {\it NetHack\/} can, it should use a font by the chosen name for the message window.
%.lp
\item[\ib{font\verb+_+status}]
If {\it NetHack\/} can, it should use a font by the chosen name for the status window.
%.lp
\item[\ib{font\verb+_+text}]
If {\it NetHack\/} can, it should use a font by the chosen name for text windows.
%.lp
\item[\ib{font\verb+_+size\verb+_+map}]
If {\it NetHack\/} can, it should use this size font for the map window.
%.lp
\item[\ib{font\verb+_+size\verb+_+menu}]
If {\it NetHack\/} can, it  should use this size font for menu windows.
%.lp
\item[\ib{font\verb+_+size\verb+_+message}]
If {\it NetHack\/} can, it should use this size font for the message window.
%.lp
\item[\ib{font\verb+_+size\verb+_+status}]
If {\it NetHack\/} can, it should use this size font for the status window.
%.lp
\item[\ib{font\verb+_+size\verb+_+text}]
If {\it NetHack\/} can, it should use this size font for text windows.
%.lp
\item[\ib{fullscreen}]
If {\it NetHack\/} can, it should try and display on the entire screen rather than in a window.
%.lp
\item[\ib{guicolor}]
Use color text and/or highlighting attributes when displaying some
non-map data (such as menu selector letters).
Curses interface only; default is on.
%.lp
\item[\ib{large\verb+_+font}]
If {\it NetHack\/} can, it should use a large font.
%.lp
\item[\ib{map\verb+_+mode}]
If {\it NetHack\/} can, it should display the map in the manner specified.
%.lp
\item[\ib{player\verb+_+selection}]
If {\it NetHack\/} can, it should pop up dialog boxes or use prompts for character selection.
%.lp
\item[\ib{popup\verb+_+dialog}]
If {\it NetHack\/} can, it should pop up dialog boxes for input.
%.lp
\item[\ib{preload\verb+_+tiles}]
If {\it NetHack\/} can, it should preload tiles into memory.
For example, in the protected mode MS-DOS version, control whether tiles
get pre-loaded into RAM at the start of the game.  Doing so
enhances performance of the tile graphics, but uses more memory. (default on).
Cannot be set with the `{\tt O}' command.
%.lp
\item[\ib{scroll\verb+_+amount}]
If {\it NetHack\/} can, it should scroll the display by this number of cells
when the hero reaches the scroll\verb+_+margin.
%.lp
\item[\ib{scroll\verb+_+margin}]
If {\it NetHack\/} can, it should scroll the display when the hero or cursor
is this number of cells away from the edge of the window.
%.lp
\item[\ib{selectsaved}]
If {\it NetHack\/} can, it should display a menu of existing saved games for the player to
choose from at game startup, if it can. Not all ports support this option.
%.lp
\item[\ib{softkeyboard}]
If {\it NetHack\/} can, it should display an onscreen keyboard.
Handhelds are most likely to support this option.
%.lp
\item[\ib{splash\verb+_+screen}]
If {\it NetHack\/} can, it should display an opening splash screen when
it starts up (default yes).
%.lp
\item[\ib{statuslines}]
Number of lines for traditional below-the-map status display.
Acceptable values are {\tt 2} and {\tt 3} (default is {\tt 2}).

%.lp ""
When set to {\tt 3}, the {\tt tty} interface moves some fields around and
mainly shows status conditions on their own line.
A display capable of showing at least 25 lines is recommended.
The value can be toggled back and forth during the game with the `{\tt O}'
command.

%.lp ""
The {\tt curses} interface does likewise if the
{\it align\verb+_+status\/}
option is set to {\it top\/} or {\it bottom\/} but ignores
{\it statuslines\/}
when set to {\it left\/} or {\it right}.

%.lp ""
The {\tt Qt} interface already displays more than 3 lines for status
so uses the
{\it statuslines\/}
value differently.
A value of {\tt 3} renders status in the {\tt Qt} interface's
original format, with the status window spread out vertically.
A value of {\tt 2} makes status be slightly condensed, moving some
fields to different lines to eliminate one whole line, reducing the
height needed.
(If NetHack has been built using a version of {\tt Qt}
older than {\tt qt-5.9},
{\it statuslines\/}
can only be set in the run-time configuration file or via NETHACKOPTIONS,
not during play with the `{\tt O}' command.)
%.lp
\item[\ib{term\verb+_+cols} {\normalfont and}]
%.lp
\item[\ib{term\verb+_+rows}]
Curses interface only.
Number of columns and rows to use for the display.
Curses will attempt to resize to the values specified but will settle
for smaller sizes if they are too big.
Default is the current window size.
%.lp
\item[\ib{tile\verb+_+file}]
Specify the name of an alternative tile file to override the default.
\\
%.lp ""
Note: the X11 interface uses X resources rather than NetHack's options
to select an alternate tile file.
See {\tt NetHack.ad}, the sample X ``application defaults'' file.
%.lp
\item[\ib{tile\verb+_+height}]
Specify the preferred height of each tile in a tile capable port.
%.lp
\item[\ib{tile\verb+_+width}]
Specify the preferred width of each tile in a tile capable port
%.lp
\item[\ib{tiled\verb+_+map}]
If {\it NetHack\/} can, it should display the map using {\it tiles} graphics
rather than simple characters (letters and punctuation, possibly
augmented by line-drawing symbols).
Setting {\tt tiled\verb+_+map} to {\it True\/} forces
{\tt ascii\verb+_+map} to be {\it False}.
%.lp
\item[\ib{black}]
Set the intensity of color black (tty).
If negated or set to zero, use darkgray (bright black).
Two to use blue instead.
%.lp
\item[\ib{setpalette}]
Let {\it NetHack\/} set the color palette.
%.lp
\item[\ib{vary\verb+_+msgcount}]
If {\it NetHack\/} can, it should display this number of messages at a time
in the message window.
%.lp
\item[\ib{windowborders}]
Whether to draw boxes around the map, status area, message area, and
persistent inventory window if enabled.
Curses interface only.
Acceptable values are

%.sd
%.si
{\tt 0} --- off, never show borders\\
{\tt 1} --- on, always show borders\\
{\tt 2} --- auto, on display is at least
(\verb&24+2&)x(\verb&80+2&) [default]\\
{\tt 3} --- on, except forced off for perm\verb+_+invent\\
{\tt 4} --- auto, except forced off for perm\verb+_+invent\\
%.ei
%.ed

%.lp ""
(The 26x82 size threshold for `2' refers to number of rows and
columns of the display.
A width of at least 110 columns (\verb&80+2+26+2&) is needed for
{\it align_status\/}
set to {\tt left} or {\tt right}.)

%.lp ""
The persistent inventory window, when enabled, can grow until it is
too big to fit on most displays, resulting in truncation of its contents.
If borders are forced on (1) or the display is big enough to show them (2),
setting the value to 3 or 4 instead will keep borders for the map, message,
and status windows but have room for two additional lines of inventory
plus widen each inventory line by two columns.
%.lp
\item[\ib{windowcolors}]
If {\it NetHack\/} can, it should display windows with the specified
foreground/background colors.
Windows GUI only.
The format is
\begin{verbatim}
    OPTION=windowcolors:wintype foreground/background
\end{verbatim}

%.pg
where wintype is one of {\it menu}, {\it message}, {\it status}, or {\it text}, and
foreground and background are colors, either a hexadecimal {\it \#rrggbb},
one of the named colors ({\it black}, {\it red}, {\it green}, {\it brown},
{\it blue}, {\it magenta}, {\it cyan}, {\it orange},
{\it brightgreen}, {\it yellow}, {\it brightblue}, {\it brightmagenta},
{\it brightcyan}, {\it white}, {\it trueblack}, {\it gray}, {\it purple},
{\it silver}, {\it maroon}, {\it fuchsia}, {\it lime}, {\it olive},
{\it navy}, {\it teal}, {\it aqua}), or one of Windows UI colors ({\it activeborder},
{\it activecaption}, {\it appworkspace}, {\it background}, {\it btnface}, {\it btnshadow},
{\it btntext}, {\it captiontext}, {\it graytext}, {\it greytext}, {\it highlight},
{\it highlighttext}, {\it inactiveborder}, {\it inactivecaption}, {\it menu},
{\it menutext}, {\it scrollbar}, {\it window}, {\it windowframe}, {\it windowtext}).

%.lp
\item[\ib{wraptext}]
If {\it NetHack\/} can, it should wrap long lines of text if they don't fit
in the visible area of the window.
\elist

%.hn 2
\subsection*{Platform-specific Customization options}

%.pg
Here are explanations of options that are used by specific platforms
or ports to customize and change the port behavior.

\blist{}
%.lp
\item[\ib{altkeyhandling}]
Select an alternate way to handle keystrokes ({\it Win32 tty\/ NetHack\/} only).
The name of the handling type is one of {\it default}, {\it ray}, {\it 340}
%.\" \item[\ib{altmeta}]
%.\" On Amiga, this option controls whether typing ``Alt'' plus another key
%.\" functions as a meta-shift for that key (default on).
%.lp
\item[\ib{altmeta}]
%.\" On other (non-Amiga) systems where this option is available, it can be
On systems where this option is available, it can be
set to tell {\it NetHack\/} to convert a two character sequence beginning with
ESC into a meta-shifted version of the second character (default off).

%.lp ""
This conversion is only done for commands, not for other input prompts.
Note that typing one or more digits as a count prefix prior to a
command---preceded by {\tt n} if the {\it number\verb+_+pad\/}
option is set---is
also subject to this conversion, so attempting to
abort the count by typing ESC will leave {\it NetHack\/} waiting for another
character to complete the two character sequence.
Type a second ESC to finish cancelling such a count.
At other prompts a single ESC suffices.
%.lp
\item[\ib{BIOS}]
Use BIOS calls to update the screen display quickly and to read the keyboard
(allowing the use of arrow keys to move) on machines with an IBM PC
compatible BIOS ROM (default off, {\it OS/2, PC\/ {\rm and} ST NetHack\/} only).
%.lp
\item[\ib{rawio}]
Force raw (non-cbreak) mode for faster output and more
bulletproof input (MS-DOS sometimes treats `{\tt \^{}P}' as a printer toggle
without it) (default off, {\it OS/2, PC\/ {\rm and} ST NetHack\/} only).
Note:  DEC Rainbows hang if this is turned on.
Cannot be set with the `{\tt O}' command.
%.lp
\item[\ib{subkeyvalue}]
({\it Win32 tty NetHack \/} only).
May be used to alter the value of keystrokes that the operating system
returns to {\it NetHack\/} to help compensate for international keyboard
issues.
OPTIONS=subkeyvalue:171/92
will return 92 to {\it NetHack\/}, if 171 was originally going to be returned.
You can use multiple subkeyvalue assignments in the configuration file
if needed.
Cannot be set with the `{\tt O}' command.
%.lp
\item[\ib{video}]
Set the video mode used ({\it PC\/ NetHack\/} only).
Values are {\it autodetect\/}, {\it default\/}, {\it vga\/}, or {\it vesa\/}.
Setting {\it vesa\/} will cause the game to display tiles, using the full
capability of the VGA hardware.
Setting {\it vga\/} will cause the game to display tiles, fixed at 640x480
in 16 colors, a mode that is compatible with all VGA hardware. Third party
tilesets will probably not work.
Setting {\it autodetect\/} attempts {\it vesa\/}, then {\it vga\/}, and
finally sets {\it default\/} if neither of those modes works.
Cannot be set with the `{\tt O}' command.
%.lp
\item[\ib{video\verb+_+height}]
Set the VGA mode resolution height (MS-DOS only, with video:vesa)
%.lp
\item[\ib{video\verb+_+width}]
Set the VGA mode resolution width (MS-DOS only, with video:vesa)
%.lp
\item[\ib{videocolors}]
\begin{sloppypar}
Set the color palette for PC systems using NO\verb+_+TERMS
(default 4-2-6-1-5-3-15-12-10-14-9-13-11, {\it PC\/ NetHack\/} only).
The order of colors is red, green, brown, blue, magenta, cyan,
bright.white, bright.red, bright.green, yellow, bright.blue,
bright.magenta, and bright.cyan.
Cannot be set with the `{\tt O}' command.
\end{sloppypar}
%.lp
\item[\ib{videoshades}]
Set the intensity level of the three gray scales available
(default dark normal light, {\it PC\/ NetHack\/} only).
If the game display is difficult to read, try adjusting these scales;
if this does not correct the problem, try {\tt !color}.
Cannot be set with the `{\tt O}' command.
\elist

%.hn 2
\subsection*{Regular Expressions}

%.pg
Regular expressions are normally POSIX extended regular expressions. It is
possible to compile {\it NetHack\/} without regular expression support on
a platform where
there is no regular expression library. While this is not true of any modern
platform, if your {\it NetHack\/} was built this way, patterns are instead glob
patterns. This applies to Autopickup exceptions, Message types, Menu colors,
and User sounds.

%.hn 2
\subsection*{Configuring Autopickup Exceptions}

%.pg
You can further refine the behavior of the ``{\tt autopickup}'' option
beyond what is available through the ``{\tt pickup\verb+_+types}'' option.

%.pg
By placing ``{\tt autopickup\verb+_+exception}'' lines in your configuration
file, you can define patterns to be checked when the game is about to
autopickup something.

\blist{}
%.lp
\item[\ib{autopickup\verb+_+exception}]
Sets an exception to the ``{\it pickup\verb+_+types}'' option.
The {\it autopickup\verb+_+exception\/} option should be followed by a regular
expression to be used as a pattern to match against the singular form of the
description of an object at your location.

In addition, some characters are treated specially if they occur as the first
character in the pattern, specifically:

%.sd
%.si
{\tt <} --- always pickup an object that matches rest of pattern;\\
{\tt >} --- never pickup an object that matches rest of pattern.
%.ei
%.ed

The {\it autopickup\verb+_+exception\/} rules are processed in the order
in which they appear in your configuration file, thus allowing a
later rule to override an earlier rule.

%.lp ""
Exceptions can be set with the `{\tt O}' command, but because they are not
included in your configuration file, they won't be in effect if you save
and then restore your game.
{\it autopickup\verb+_+exception\/} rules are not saved with the game.
\elist

%.lp "Here are some examples:"
Here are some examples:
\begin{verbatim}
    autopickup_exception="<*arrow"
    autopickup_exception=">*corpse"
    autopickup_exception=">* cursed*"
\end{verbatim}

%.pg
The first example above will result in autopickup of any type of arrow.
The second example results in the exclusion of any corpse from autopickup.
The last example results in the exclusion of items known to be cursed from
autopickup.

%.lp

%.hn 2
\subsection*{Changing Key Bindings}

%.pg
It is possible to change the default key bindings of some special commands,
menu accelerator keys, extended commands, by using BIND stanzas in the
configuration file. Format is key, followed by the command to bind to,
separated by a colon. The key can be a single character (``{\tt x}''),
a control key (``{\tt \^{}X}'', ``{\tt C-x}''), a meta key (``{\tt M-x}''),
a mouse button, or a three-digit decimal ASCII code.

%.pg
For example:

\begin{verbatim}
    BIND=^X:getpos.autodescribe
    BIND=\:menu_first_page
    BIND=v:loot
\end{verbatim}

\blist{}
%.lp "Extended command keys"
\item[\tb{Extended command keys}]
You can bind multiple keys to the same extended command. Unbind a key by
using ``{\tt nothing}'' as the extended command to bind to. You can also bind
the ``{\tt <esc>}'', ``{\tt <enter>}'', and ``{\tt <space>}'' keys.

%.lp "Menu accelerator keys"
\item[\tb{Menu accelerator keys}]
The menu control or accelerator keys can also be rebound via OPTIONS lines
in the configuration file.
You cannot bind object symbols or selection letters into menu accelerators.
Some interfaces only support some of the menu accelerators.

%.lp "Mouse buttons"
\item[\tb{Mouse buttons}]
You can bind ``mouse1'' or ``mouse2'' to ``{\tt nothing}'',
``{\tt therecmdmenu}'', ``{\tt clicklook}'', or ``{\tt mouseaction}''.

%.lp "Special command keys"
\item[\tb{Special command keys}]
Below are the special commands you can rebind. Some of them can be bound to
same keys with no problems, others are in the same ``context'', and if bound
to same keys, only one of those commands will be available. Special command
can only be bound to a single key.
\elist

%.pg
\blist{\itemindent 10mm \labelwidth 15mm \rightmargin 15mm}
%.lp
\item[{\bb{count}}]
Prefix key to start a count, to repeat a command this many times.
With {\it number\verb+_+pad\/} only. Default is~`{\tt n}'.
%.lp
\item[{\bb{getdir.help}}]
When asked for a direction, the key to show the help. Default is~`{\tt ?}'.
%.lp
\item[{\bb{getdir.mouse}}]
When asked for a direction, the key to initiate a simulated mouse click.
You will be asked to pick a location.
Use movement keystrokes to move the cursor around the map, then type
the getpos.pick.once key (default `{\tt ,}')
or the getpos.pick key (default `{\tt .}')
to finish as if performing a left or right click.
Only useful when using the {\tt \#therecmdmenu} command.
Default is~`{\tt \verb+_+}'.
%.lp
\item[{\bb{getdir.self}}]
When asked for a direction, the key to target yourself. Default is~`{\tt .}'.
%.lp
\item[{\bb{getdir.self2}}]
When asked for a direction, an alternate key to target yourself.
Default is~`{\tt s}'.
%.lp
\item[{\bb{getpos.autodescribe}}]
When asked for a location, the key to toggle {\it autodescribe\/}.
Default is~`{\tt \#}'.
%.lp
\item[{\bb{getpos.all.next}}]
When asked for a location, the key to go to next closest interesting thing.
Default is~`{\tt a}'.
%.lp
\item[{\bb{getpos.all.prev}}]
When asked for a location, the key to go to previous closest interesting thing.
Default is~`{\tt A}'.
%.lp
\item[{\bb{getpos.door.next}}]
When asked for a location, the key to go to next closest door or doorway.
Default is~`{\tt d}'.
%.lp
\item[{\bb{getpos.door.prev}}]
When asked for a location, the key to go to previous closest door or doorway.
Default is~`{\tt D}'.
%.lp
\item[{\bb{getpos.help}}]
When asked for a location, the key to show help. Default is~`{\tt ?}'.
%.lp
\item[{\bb{getpos.mon.next}}]
When asked for a location, the key to go to next closest monster.
Default is~`{\tt m}'.
%.lp
\item[{\bb{getpos.mon.prev}}]
When asked for a location, the key to go to previous closest monster.
Default is~`{\tt M}'.
%.lp
\item[{\bb{getpos.obj.next}}]
When asked for a location, the key to go to next closest object.
Default is~`{\tt o}'.
%.lp
\item[{\bb{getpos.obj.prev}}]
When asked for a location, the key to go to previous closest object.
Default is~`{\tt O}'.
%.lp
\item[{\bb{getpos.menu}}]
When asked for a location, and using one of the next or previous keys to
cycle through targets, toggle showing a menu instead. Default is~`{\tt !}'.
%.lp
\item[{\bb{getpos.moveskip}}]
When asked for a location, and using the shifted movement keys or
meta-digit keys to fast-move around, move by skipping the same glyphs
instead of by 8 units.
Default is~`{\tt *}'.
%.lp
\item[{\bb{getpos.filter}}]
When asked for a location, change the filtering mode when using one of
the next or previous keys to cycle through targets. Toggles between no
filtering, in view only, and in the same area only. Default is~`{\tt "}'.
%.lp
\item[{\bb{getpos.pick}}]
When asked for a location, the key to choose the location, and possibly
ask for more info.
When simulating a mouse click after being asked for a direction (see
getdir.mouse above), the key to use to respond as right click.
Default is~`{\tt .}'.
%.lp
\item[{\bb{getpos.pick.once}}]
When asked for a location, the key to choose the location, and skip
asking for more info.
When simulating a mouse click after being asked for a direction,
the key to respond as left click.
Default is~`{\tt ,}'.
%.lp
\item[{\bb{getpos.pick.quick}}]
When asked for a location, the key to choose the location, skip asking
for more info, and exit the location asking loop. Default is~`{\tt ;}'.
%.lp
\item[{\bb{getpos.pick.verbose}}]
When asked for a location, the key to choose the location, and show more
info without asking. Default is~`{\tt :}'.
%.lp
\item[{\bb{getpos.self}}]
When asked for a location, the key to go to your location.
Default is~`{\tt @}'.
%.lp
\item[{\bb{getpos.unexplored.next}}]
When asked for a location, the key to go to next closest unexplored location.
Default is~`{\tt x}'.
%.lp
\item[{\bb{getpos.unexplored.prev}}]
When asked for a location, the key to go to previous closest unexplored
location. Default is~`{\tt X}'.
%.lp
\item[{\bb{getpos.valid}}]
When asked for a location, the key to go to show valid target locations.
Default is~`{\tt \$}'.
%.lp
\item[{\bb{getpos.valid.next}}]
When asked for a location, the key to go to next closest valid location.
Default is~`{\tt z}'.
%.lp
\item[{\bb{getpos.valid.prev}}]
When asked for a location, the key to go to previous closest valid location.
Default is~`{\tt Z}'.
\elist


%.hn 2
\subsection*{Configuring Message Types}

%.pg
You can change the way the messages are shown in the message area, when
the message matches a user-defined pattern.

%.pg
In general, the configuration file entries to describe the message types
look like this:
\begin{verbatim}
    MSGTYPE=type "pattern"
\end{verbatim}
\blist{}
%.lp
\item[\ib{type}]
how the message should be shown:
%.sd
%.si
\\
{\tt show}  --- show message normally.\\
{\tt hide}  --- never show the message.\\
{\tt stop}  --- wait for user with more-prompt.\\
{\tt norep} --- show the message once, but not again if no other message is
shown in between.
%.ei
%.ed
%.lp
\item[\ib{pattern}]
the pattern to match. The pattern should be a regular expression.
\elist

%.lp ""
Here's an example of message types using {\it NetHack's\/} internal
pattern matching facility:

\begin{verbatim}
    MSGTYPE=stop "You feel hungry."
    MSGTYPE=hide "You displaced *."
\end{verbatim}

specifies that whenever a message ``You feel hungry'' is shown,
the user is prompted with more-prompt, and a message matching
``You displaced  \verb+<+something\verb+>+'' is not shown at all.

%.lp
The order of the defined MSGTYPE lines is important; the last matching
rule is used. Put the general case first, exceptions below them.

%.pg

%.lp
%.hn 2
\subsection*{Configuring Menu Colors}

%.pg
Some platforms allow you to define colors used in menu lines when the
line matches a user-defined pattern.
At this time the tty, curses, win32tty and
win32gui interfaces support this.

%.pg
In general, the configuration file entries to describe the menu color mappings
look like this:
\begin{verbatim}
    MENUCOLOR="pattern"=color&attribute
\end{verbatim}

\blist{}
%.lp
\item[\ib{pattern}]
the pattern to match;
%.lp
\item[\ib{color}]
the color to use for lines matching the pattern;
%.lp
\item[\ib{attribute}]
the attribute to use for lines matching the pattern. The attribute is
optional, and if left out, you must also leave out the preceding ampersand.
If no attribute is defined, no attribute is used.
\elist

%.lp ""
The pattern should be a regular expression.

%.lp ""
Allowed colors are {\it black}, {\it red}, {\it green}, {\it brown},
{\it blue}, {\it magenta}, {\it cyan}, {\it gray}, {\it orange},
{\it light-green}, {\it yellow}, {\it light-blue}, {\it light-magenta},
{\it light-cyan}, and {\it white}.
And {\it no-color}, the default foreground color, which isn't necessarily
the same as any of the other colors.

%.lp ""
Allowed attributes are {\it none}, {\it bold}, {\it dim}, {\it italic},
{\it underline},{\it blink}, and {\it inverse}.
{\it Normal\/} is a synonym for {\it none}.
Note that the platform used may interpret the attributes any way it
wants.

%.lp ""
Here's an example of menu colors using {\it NetHack's\/} internal
pattern matching facility:

\begin{verbatim}
    MENUCOLOR="* blessed *"=green
    MENUCOLOR="* cursed *"=red
    MENUCOLOR="* cursed *(being worn)"=red&underline
\end{verbatim}

specifies that any menu line with ``~blessed~'' contained
in it will be shown in green color, lines with ``~cursed~'' will be
shown in red, and lines with ``~cursed~'' followed by ``(being worn)''
on the same line will be shown in red color and underlined.
You can have multiple MENUCOLOR entries in your configuration file,
and the last MENUCOLOR line that matches
a menu line will be used for the line.

%.pg
Note that if you intend to have one or more color specifications match
``~uncursed~'', you will probably want to turn the
{\it implicit\verb+_+uncursed\/}
option off so that all items known to be uncursed are actually
displayed with the ``uncursed'' description.

%.lp
%.hn 2
\subsection*{Configuring User Sounds}

%.pg
Some platforms allow you to define sound files to be played when a message
that matches a user-defined pattern is delivered to the message window.
At this time the Qt port and the win32tty and win32gui ports support the
use of user sounds.

%.pg
The following configuration file entries are relevant to mapping user sounds
to messages:

\blist{}
%.lp
\item[\ib{SOUNDDIR}]
The directory that houses the sound files to be played.
%.lp
\item[\ib{SOUND}]
An entry that maps a sound file to a user-specified message pattern.
Each SOUND entry is broken down into the following parts:

%.sd
%.si
{\tt MESG       } --- message window mapping (the only one supported in 3.7.0);\\
{\tt msgtype    } --- optional; message type to use, see ``Configuring User Sounds''\\
{\tt pattern    } --- the pattern to match;\\
{\tt sound file } --- the sound file to play;\\
{\tt volume     } --- the volume to be set while playing the sound file;\\
{\tt sound index} --- optional; the index corresponding to a sound file.
%.ei
%.ed
\elist

%.lp ""
The pattern should be a POSIX extended regular expression.

For example:

\begin{verbatim}
    SOUNDDIR=C:\\nethack\\sounds
    SOUND=MESG "This door is locked" "lock.wav" 100
    SOUND=MESG hide "^You miss the " "swing.wav" 75
\end{verbatim}
%.pg

%.lp
%.hn 2
\subsection*{Configuring Status Hilites}

%.pg
Your copy of {\it NetHack\/} may have been compiled with support
for {\it Status Hilites}.
If so, you can customize your game display by setting thresholds to
change the color or appearance of fields in the status display.

The format for defining status colors is:\\
\begin{verbatim}
OPTION=hilite_status:field-name/behavior/color&attributes
\end{verbatim}

For example, the following line in your configuration file will cause
the hitpoints field to display in the color red if your hitpoints
drop to or below a threshold of 30%:\\
\begin{verbatim}
OPTION=hilite_status:hitpoints/<=30%/red/normal
\end{verbatim}
(That example is actually specifying {\tt red\&normal} for  {\tt <=30\%}
and {\tt no-color\&normal} for {\tt >30\%}.)\\

For another example, the following line in your configuration file will cause
wisdom to be displayed red if it drops and green if it rises:\\
\begin{verbatim}
OPTION=hilite_status:wisdom/down/red/up/green
\end{verbatim}

Allowed colors are black, red, green, brown, blue, magenta, cyan, gray,
orange, light-green, yellow, light-blue, light-magenta, light-cyan, and white.
And {\it no-color}, the default foreground color on the display, which
is not necessarily the same as black or white or any of the other colors.

Allowed attributes are none, bold, dim, underline, blink, and inverse.
``Normal'' is a synonym for ``none''; they should not be used in
combination with any of the other attributes.

To specify both a color and an attribute, use `\&' to combine them.
To specify multiple attributes, use `+' to combine those.

%.lp ""
For example: {\tt magenta\&inverse+dim}.

Note that the display may substitute or ignore particular attributes
depending upon its capabilities, and in general may interpret the
attributes any way it wants.
For example, on some display systems a request for bold might yield
blink or vice versa.
On others, issuing an attribute request while another is already
set up will replace the earlier attribute rather than combine with it.
Since nethack issues attribute requests sequentially (at least with
the {\it tty} interface) rather than all at once, the only way a
situation like that can be controlled is to specify just one attribute.

You can adjust the display of the following status fields:
%.sd
\begin{center}
\begin{tabular}{lll}
%TABLE_START
title & dungeon-level & experience-level\\
strength & gold & experience\\
dexterity & hitpoints & HD\\
constitution & hitpoints-max & time\\
intelligence & power & hunger\\
wisdom & power-max & carrying-capacity\\
charisma & armor-class & condition\\
alignment &  & score\\
%TABLE_END  Do not delete this line.
\end{tabular}
\end{center}
%.ed
%.lp ""
The pseudo-field `characteristics' can be used to set all six
of Str, Dex, Con, Int, Wis, and Cha at once.  `HD' is `hit dice',
an approximation of experience level displayed when polymorphed.
`experience', `time', and `score' are conditionally displayed
depending upon your other option settings.

%.lp ""
Instead of a behavior, `condition' takes the following condition flags:
{\it stone}, {\it slime}, {\it strngl}, {\it foodpois}, {\it termill},
{\it blind}, {\it deaf}, {\it stun}, {\it conf}, {\it hallu},
{\it lev}, {\it fly}, and {\it ride}.
You can use `major\_troubles' as an alias
for stone through termill, `minor\_troubles' for blind through hallu,
`movement' for lev, fly, and ride, and `all' for every condition.

%.lp ""
Allowed behaviors are ``always'', ``up'', ``down'', ``changed'', a
percentage or absolute number threshold, or text to match against.

\blist{}
%.lp "*"
\item[{\tt always}] will set the default attributes for that field.
%.lp "*"
\item[{\tt up}{\normalfont, }{\tt down}] set the field attributes
for when the field value changes upwards or downwards. This attribute
times out after {\tt statushilites} turns.
%.lp "*"
\item[{\tt changed}] sets the field attribute for when the field value
changes. This attribute times out after {\tt statushilites} turns.
(If a field has both a ``changed'' rule and an ``up'' or ``down''
rule which matches a change in the field's value,
the ``up'' or ``down'' one takes precedence.)
%.lp "*"
\item[{\tt percentage}] sets the field attribute when the field value
matches the percentage.
It is specified as a number between 0 and 100, followed by `{\tt \%}'
(percent sign).
If the percentage is prefixed with `{\tt <=}' or `{\tt >=}',
it also matches when value is below or above the percentage.
Use prefix `{\tt <}' or `{\tt >}' to match when strictly below or above.
(The numeric limit is relaxed slightly for those: {\tt >-1\%}
and {\tt <101\%} are allowed.)
Only four fields support percentage rules.
Percentages for ``{\it hitpoints\/}'' and ``{\it power\/}'' are
straightforward; they're based on the corresponding maximum field.
Percentage highlight rules are also allowed for ``{\it experience level\/}''
and ``{\it experience points\/}'' (valid when the
{\it showexp\/}
option is enabled).
For those, the percentage is based on the progress from the start of
the current experience level to the start of the next level.
So if level 2 starts at 20 points and level 3 starts at 40 points,
having 30 points is 50\% and 35 points is 75\%.
100\% is unattainable for experience because you'll gain a level and
the calculations will be reset for that new level, but a rule for
{\tt =100\%} is allowed and matches the special case of being
exactly 1 experience point short of the next level.
% (If you manage to reach level 30, there is no next level and the
% percentage will remain at 0\% no matter have many additional experience
% points you earn.)
%.lp "*"
\item[{\tt absolute}] value sets the attribute when the field value
matches that number.
The number must be 0 or higher, except for ``{\it armor-class\/} which
allows negative values, and may optionally be preceded by `{\tt =}'.
If the number is preceded by `{\tt <=}' or `{\tt >=}' instead,
it also matches when value is below or above.
If the prefix is `{\tt <}' or `{\tt >}', only match when strictly
above or below.
%.lp "*"
\item[{\tt text}] match sets the attribute when the field value matches the text.
Text matches can only be used for ``{\it alignment\/}'',
``{\it carrying-capacity\/}'', ``{\it hunger\/}'', ``{\it dungeon-level\/}'',
and ``{\it title\/}''.
For title, only the role's rank title
is tested; the character's name is ignored.
%.ei
\elist

The in-game options menu can help you determine the correct syntax for a
configuration file.

The whole feature can be disable by setting option {\it statushilites} to 0.

Example hilites:
\begin{verbatim}
    OPTION=hilite_status: gold/up/yellow/down/brown
    OPTION=hilite_status: characteristics/up/green/down/red
    OPTION=hilite_status: hitpoints/100%/gray&normal
    OPTION=hilite_status: hitpoints/<100%/green&normal
    OPTION=hilite_status: hitpoints/<66%/yellow&normal
    OPTION=hilite_status: hitpoints/<50%/orange&normal
    OPTION=hilite_status: hitpoints/<33%/red&bold
    OPTION=hilite_status: hitpoints/<15%/red&inverse
    OPTION=hilite_status: condition/major/orange&inverse
    OPTION=hilite_status: condition/lev+fly/red&inverse
\end{verbatim}

%.lp
%.hn 2
\subsection*{Modifying {\it NetHack\/} Symbols}

%.pg
{\it NetHack\/} can load entire symbol sets from the symbol file.

%.pg
The options that are used to select a particular symbol set from the
symbol file are:

\blist{}
%.lp
\item[\ib{symset}]
Set the name of the symbol set that you want to load.
{\it symbols\/}.

%.lp
\item[\ib{roguesymset}]
Set the name of the symbol set that you want to load for display
on the rogue level.
\elist

You can also override one or more symbols using the {\it SYMBOLS\/} and
{\it ROGUESYMBOLS\/} configuration file options.
Symbols are specified as {\it name:value\/} pairs.
Note that {\it NetHack\/} escape-processes
the {\it value\/} string in conventional C fashion.
This means that `\verb+\+' is a prefix to take the following character
literally.
Thus `\verb+\+' needs to be represented as `\verb+\\+'.
The special prefix
`\verb+\m+' switches on the meta bit in the symbol value, and the
`{\tt \^{}}' prefix causes the following character to be treated as a control
character.

{
\small
\begin{longtable}{lll}
\caption[]{NetHack Symbols}\\
Default                      & Symbol Name                & Description\\
\hline \hline
\endhead
\verb@ @ & S\verb+_+air                     &	(air)\\
\_ & S\verb+_+altar                   &	(altar)\\
\verb@"@ & S\verb+_+amulet                  &	(amulet)\\
\verb@A@ & S\verb+_+angel                   &	(angelic being)\\
\verb@a@ & S\verb+_+ant                     &	(ant or other insect)\\
\verb@^@ & S\verb+_+anti\verb+_+magic\verb+_+trap       &	(anti-magic field)\\
\verb@[@ & S\verb+_+armor                   &	(suit or piece of armor)\\
\verb@[@ & S\verb+_+armour                  &	(suit or piece of armor)\\
\verb@^@ & S\verb+_+arrow\verb+_+trap             &	(arrow trap)\\
\verb@0@ & S\verb+_+ball                    &	(iron ball)\\
\# & S\verb+_+bars                    &	(iron bars)\\
\verb@B@ & S\verb+_+bat                     &	(bat or bird)\\
\verb@^@ & S\verb+_+bear\verb+_+trap              &	(bear trap)\\
\verb@-@ & S\verb+_+blcorn                  &	(bottom left corner)\\
\verb@b@ & S\verb+_+blob                    &	(blob)\\
\verb@+@ & S\verb+_+book                    &	(spellbook)\\
\verb@)@ & S\verb+_+boomleft                &	(boomerang open left)\\
\verb@(@ & S\verb+_+boomright               &	(boomerang open right)\\
\verb@`@ & S\verb+_+boulder                 &	(boulder)\\
\verb@-@ & S\verb+_+brcorn                  &	(bottom right corner)\\
\verb@>@ & S\verb+_+brdnladder              &	(branch ladder down)\\
\verb@>@ & S\verb+_+brdnstair               &	(branch staircase down)\\
\verb@<@ & S\verb+_+brupladder              &	(branch ladder up)\\
\verb@<@ & S\verb+_+brupstair               &	(branch staircase up)\\
\verb@C@ & S\verb+_+centaur                 &	(centaur)\\
\verb@_@ & S\verb+_+chain                   &	(iron chain)\\
\# & S\verb+_+cloud                   &	(cloud)\\
\verb@c@ & S\verb+_+cockatrice              &	(cockatrice)\\
\$ & S\verb+_+coin                    &	(pile of coins)\\
\# & S\verb+_+corr                    &	(corridor)\\
\verb@-@ & S\verb+_+crwall                  &	(wall)\\
\verb@-@ & S\verb+_+darkroom                &	(dark room)\\
\verb@^@ & S\verb+_+dart\verb+_+trap              &	(dart trap)\\
\verb@&@ & S\verb+_+demon                   &	(major demon)\\
\verb@*@ & S\verb+_+digbeam                 &	(dig beam)\\
\verb@>@ & S\verb+_+dnladder                &	(ladder down)\\
\verb@>@ & S\verb+_+dnstair                 &	(staircase down)\\
\verb@d@ & S\verb+_+dog                     &	(dog or other canine)\\
\verb@D@ & S\verb+_+dragon                  &	(dragon)\\
\verb@;@ & S\verb+_+eel                     &	(sea monster)\\
\verb@E@ & S\verb+_+elemental               &	(elemental)\\
\verb@/@ & S\verb+_+expl\verb+_+tl          &	(explosion top left)\\
\verb@-@ & S\verb+_+expl\verb+_+tc          &	(explosion top center)\\
\verb@\@ & S\verb+_+expl\verb+_+tr          &	(explosion top right)\\
\verb@|@ & S\verb+_+expl\verb+_+ml          &	(explosion middle left)\\
\verb@ @ & S\verb+_+expl\verb+_+mc          &	(explosion middle center)\\
\verb@|@ & S\verb+_+expl\verb+_+mr          &	(explosion middle right)\\
\verb@\@ & S\verb+_+expl\verb+_+bl          &	(explosion bottom left)\\
\verb@-@ & S\verb+_+expl\verb+_+bc          &	(explosion bottom center)\\
\verb@/@ & S\verb+_+expl\verb+_+br          &	(explosion bottom right)\\
\verb@e@ & S\verb+_+eye                     &	(eye or sphere)\\
\verb@^@ & S\verb+_+falling\verb+_+rock\verb+_+trap     &	(falling rock trap)\\
\verb@f@ & S\verb+_+feline                  &	(cat or other feline)\\
\verb@^@ & S\verb+_+fire\verb+_+trap              &	(fire trap)\\
\verb@!@ & S\verb+_+flashbeam               &	(flash beam)\\
\% & S\verb+_+food                    &	(piece of food)\\
\{ & S\verb+_+fountain                &	(fountain)\\
\verb@F@ & S\verb+_+fungus                  &	(fungus or mold)\\
\verb@*@ & S\verb+_+gem                     &	(gem or rock)\\
\verb@ @ & S\verb+_+ghost                   &	(ghost)\\
\verb@H@ & S\verb+_+giant                   &	(giant humanoid)\\
\verb@G@ & S\verb+_+gnome                   &	(gnome)\\
\verb@'@ & S\verb+_+golem                   &	(golem)\\
\verb@|@ & S\verb+_+grave                   &	(grave)\\
\verb@g@ & S\verb+_+gremlin                 &	(gremlin)\\
\verb@-@ & S\verb+_+hbeam                   &	(wall)\\
\# & S\verb+_+hcdbridge               &	(horizontal raised drawbridge)\\
\verb@+@ & S\verb+_+hcdoor                  &	(closed door)\\
\verb@.@ & S\verb+_+hodbridge               &	(horizontal lowered drawbridge)\\
\verb@|@ & S\verb+_+hodoor                  &	(open door)\\
\verb\^\ & S\verb+_+hole                    &	(hole)\\
\verb~@~ & S\verb+_+human                   &	(human or elf)\\
\verb@h@ & S\verb+_+humanoid                &	(humanoid)\\
\verb@-@ & S\verb+_+hwall                   &	(horizontal wall)\\
\verb@.@ & S\verb+_+ice                     &	(ice)\\
\verb@i@ & S\verb+_+imp                     &	(imp or minor demon)\\
\verb@I@ & S\verb+_+invisible               &	(invisible monster)\\
\verb@J@ & S\verb+_+jabberwock              &	(jabberwock)\\
\verb@j@ & S\verb+_+jelly                   &	(jelly)\\
\verb@k@ & S\verb+_+kobold                  &	(kobold)\\
\verb@K@ & S\verb+_+kop                     &	(Keystone Kop)\\
\verb@^@ & S\verb+_+land\verb+_+mine              &	(land mine)\\
\verb@}@ & S\verb+_+lava                    &	(molten lava)\\
\verb@}@ & S\verb+_+lavawall                &	(wall of lava)\\
\verb@l@ & S\verb+_+leprechaun              &	(leprechaun)\\
\verb@^@ & S\verb+_+level\verb+_+teleporter       &	(level teleporter)\\
\verb@L@ & S\verb+_+lich                    &	(lich)\\
\verb@y@ & S\verb+_+light                   &	(light)\\
\# & S\verb+_+litcorr                 &	(lit corridor)\\
\verb@:@ & S\verb+_+lizard                  &	(lizard)\\
\verb@\@ & S\verb+_+lslant                  &	(wall)\\
\verb@^@ & S\verb+_+magic\verb+_+portal           &	(magic portal)\\
\verb@^@ & S\verb+_+magic\verb+_+trap             &	(magic trap)\\
\verb@m@ & S\verb+_+mimic                   &	(mimic)\\
\verb@]@ & S\verb+_+mimic\verb+_+def              &	(mimic)\\
\verb@M@ & S\verb+_+mummy                   &	(mummy)\\
\verb@N@ & S\verb+_+naga                    &	(naga)\\
\verb@.@ & S\verb+_+ndoor                   &	(doorway)\\
\verb@n@ & S\verb+_+nymph                   &	(nymph)\\
\verb@O@ & S\verb+_+ogre                    &	(ogre)\\
\verb@o@ & S\verb+_+orc                     &	(orc)\\
\verb@p@ & S\verb+_+piercer                 &	(piercer)\\
\verb@^@ & S\verb+_+pit                     &	(pit)\\
\# & S\verb+_+poisoncloud             &	(poison cloud)\\
\verb@^@ & S\verb+_+polymorph\verb+_+trap         &	(polymorph trap)\\
\verb@}@ & S\verb+_+pool                    &	(water)\\
\verb@!@ & S\verb+_+potion                  &	(potion)\\
\verb@P@ & S\verb+_+pudding                 &	(pudding or ooze)\\
\verb@q@ & S\verb+_+quadruped               &	(quadruped)\\
\verb@Q@ & S\verb+_+quantmech               &	(quantum mechanic)\\
\verb@=@ & S\verb+_+ring                    &	(ring)\\
\verb@`@ & S\verb+_+rock                    &	(boulder or statue)\\
\verb@r@ & S\verb+_+rodent                  &	(rodent)\\
\verb@^@ & S\verb+_+rolling\verb+_+boulder\verb+_+trap  &	(rolling boulder trap)\\
\verb@.@ & S\verb+_+room                    &	(floor of a room)\\
\verb@/@ & S\verb+_+rslant                  &	(wall)\\
\verb@^@ & S\verb+_+rust\verb+_+trap              &	(rust trap)\\
\verb@R@ & S\verb+_+rustmonst               &	(rust monster or disenchanter)\\
\verb@?@ & S\verb+_+scroll                  &	(scroll)\\
\# & S\verb+_+sink                    &	(sink)\\
\verb@^@ & S\verb+_+sleeping\verb+_+gas\verb+_+trap     &	(sleeping gas trap)\\
\verb@S@ & S\verb+_+snake                   &	(snake)\\
\verb@s@ & S\verb+_+spider                  &	(arachnid or centipede)\\
\verb@^@ & S\verb+_+spiked\verb+_+pit             &	(spiked pit)\\
\verb@^@ & S\verb+_+squeaky\verb+_+board          &	(squeaky board)\\
\verb@0@ & S\verb+_+ss1                     &	(magic shield 1 of 4)\\
\# & S\verb+_+ss2                     &	(magic shield 2 of 4)\\
\verb+@+ & S\verb+_+ss3                     &	(magic shield 3 of 4)\\
\verb@*@ & S\verb+_+ss4                     &	(magic shield 4 of 4)\\
\verb@^@ & S\verb+_+statue\verb+_+trap            &	(statue trap)\\
\verb@ @ & S\verb+_+stone                   &	(solid rock)\\
\verb@]@ & S\verb+_+strange\verb+_+obj      &	(strange object)\\
\verb@-@ & S\verb+_+sw\verb+_+bc                  &	(swallow bottom center)\\
\verb@\@ & S\verb+_+sw\verb+_+bl                  &	(swallow bottom left)\\
\verb@/@ & S\verb+_+sw\verb+_+br                  &	(swallow bottom right	)\\
\verb@|@ & S\verb+_+sw\verb+_+ml                  &	(swallow middle left)\\
\verb@|@ & S\verb+_+sw\verb+_+mr                  &	(swallow middle right)\\
\verb@-@ & S\verb+_+sw\verb+_+tc                  &	(swallow top center)\\
\verb@/@ & S\verb+_+sw\verb+_+tl                  &	(swallow top left)\\
\verb@\@ & S\verb+_+sw\verb+_+tr                  &	(swallow top right)\\
\verb@-@ & S\verb+_+tdwall                  &	(wall)\\
\verb@^@ & S\verb+_+teleportation\verb+_+trap     &	(teleportation trap)\\
\verb@\@ & S\verb+_+throne                  &	(opulent throne)\\
\verb@-@ & S\verb+_+tlcorn                  &	(top left corner)\\
\verb@|@ & S\verb+_+tlwall                  &	(wall)\\
\verb@(@ & S\verb+_+tool                    &	(useful item (pick-axe, key, lamp...))\\
\verb@^@ & S\verb+_+trap\verb+_+door              &	(trap door)\\
\verb@t@ & S\verb+_+trapper                 &	(trapper or lurker above)\\
\verb@-@ & S\verb+_+trcorn                  &	(top right corner)\\
\# & S\verb+_+tree                    &	(tree)\\
\verb@T@ & S\verb+_+troll                   &	(troll)\\
\verb@|@ & S\verb+_+trwall                  &	(wall)\\
\verb@-@ & S\verb+_+tuwall                  &	(wall)\\
\verb@U@ & S\verb+_+umber                   &	(umber hulk)\\
\verb@ @ & S\verb+_+unexplored              &	(unexplored terrain)\\
\verb@u@ & S\verb+_+unicorn                 &	(unicorn or horse)\\
\verb@<@ & S\verb+_+upladder                &	(ladder up)\\
\verb@<@ & S\verb+_+upstair                 &	(staircase up)\\
\verb@V@ & S\verb+_+vampire                 &	(vampire)\\
\verb@|@ & S\verb+_+vbeam                   &	(wall)\\
\# & S\verb+_+vcdbridge               &	(vertical raised drawbridge)\\
\verb@+@ & S\verb+_+vcdoor                  &	(closed door)\\
\verb@.@ & S\verb+_+venom                   &	(splash of venom)\\
\verb@^@ & S\verb+_+vibrating\verb+_+square       &	(vibrating square)\\
\verb@.@ & S\verb+_+vodbridge               &	(vertical lowered drawbridge)\\
\verb@-@ & S\verb+_+vodoor                  &	(open door)\\
\verb@v@ & S\verb+_+vortex                  &	(vortex)\\
\verb@|@ & S\verb+_+vwall                   &	(vertical wall)\\
\verb@/@ & S\verb+_+wand                    &	(wand)\\
\verb@}@ & S\verb+_+water                   &	(water)\\
\verb@)@ & S\verb+_+weapon                  &	(weapon)\\
\verb@"@ & S\verb+_+web                     &	(web)\\
\verb@w@ & S\verb+_+worm                    &	(worm)\\
\verb@~@ & S\verb+_+worm\verb+_+tail              &	(long worm tail)\\
\verb@W@ & S\verb+_+wraith                  &	(wraith)\\
\verb@x@ & S\verb+_+xan                     &	(xan or other extraordinary insect)\\
\verb@X@ & S\verb+_+xorn                    &	(xorn)\\
\verb@Y@ & S\verb+_+yeti                    &	(apelike creature)\\
\verb@Z@ & S\verb+_+zombie                  &	(zombie)\\
\verb@z@ & S\verb+_+zruty                   &	(zruty)\\
\verb@ @ & S\verb+_+pet\verb+_+override     &	(any pet if ACCESSIBILITY=1 is set)\\
\verb@ @ & S\verb+_+hero\verb+_+override    &	(hero if ACCESSIBILITY=1 is set)
\end{longtable}%
}

\hyphenation{sysconf}	%no syllable breaks => don't hyphenate file name
%.lp
Notes:

%.lp "*"
Several symbols in this table appear to be blank.
They are the space character, except for S\verb+_+pet\verb+_+override
and S\verb+_+hero\verb+_+override which don't have any default value
and can only be used if enabled in the ``sysconf'' file.

%.lp "*"
S\verb+_+rock is misleadingly named; rocks and stones use S\verb+_+gem.
Statues and boulders are the rock being referred to, but since
version 3.6.0, statues are displayed as the monster they depict.
So S\verb+_+rock is only used for boulders and not used at all if
overridden by the more specific S\verb+_+boulder.

%.lp
%.hn 2
\subsection*{Customizing Map Glyph Representations Using Unicode}

%.pg
If your platform or terminal supports the display of UTF-8 character
sequences, you can customize your game display by assigning Unicode
codepoint values and red-green-blue colors to glyph
representations. The customizations can be specified for use with a symset that
has a UTF8 handler within the symbols file such as the enhanced1 set, or
individually within your own nethack.rc file.

The format for defining a glyph representation is:\\
\begin{verbatim}
OPTIONS=glyph:glyphid/U+nnnn/R-G-B
\end{verbatim}

The window port that is active needs to provide support for displaying UTF-8
character sequences and explicit 24-bit red-green-blue colors in order for the glyph
representation to be visible as specified.

For example, the following line in your configuration file will cause
the glyph representation for glyphid G\verb+_+pool to use Unicode codepoint U+224B
and the color represented by R-G-B value 0-0-160:\\
\begin{verbatim}
OPTIONS=glyph:G_pool/U+224B/0-0-160
\end{verbatim}

The list of acceptable glyphid's can be produced by 
\begin{verbatim}
    nethack --glyphids
\end{verbatim}
Individual NetHack glyphs can be specified using the G\verb+_+ prefix,
or you can use an S\verb+_+ symbol for a glyphid and store the custom
representation for all NetHack glyphs that would map to that
particular symbol.

You will need to select a symset with a UTF8 handler to enable the
display of the customizations, such as the Enhanced symset.

%.pg
%.hn 2
\subsection*{Configuring {\it NetHack\/} for Play by the Blind}

%.pg
{\it NetHack\/} can be set up to use only standard ASCII characters for making
maps of the dungeons. This makes even the MS-DOS versions of {\it NetHack}
(which use special line-drawing characters by default) completely
accessible to the blind who use speech and/or Braille access technologies.
Players will require a good working knowledge of their screen-reader's
review features, and will have to know how to navigate horizontally and
vertically character by character. They will also find the search
capabilities of their screen-readers to be quite valuable. Be certain to
examine this Guidebook before playing so you have an idea what the screen
layout is like. You'll also need to be able to locate the PC cursor. It is
always where your character is located. Merely searching for an @-sign will
not always find your character since there are other humanoids represented
by the same sign. Your screen-reader should also have a function which
gives you the row and column of your review cursor and the PC cursor.
These co-ordinates are often useful in giving players a better sense of the
overall location of items on the screen.
%.pg
{\it NetHack\/} can also be compiled with support for sending the game
messages to an external program, such as a text-to-speech synthesizer. If
the ``{\tt \#version}'' extended command shows ``external program as a
message handler'', your {\it NetHack\/}
has been compiled with the capability. When compiling {\it NetHack\/}
from source
on Linux and other POSIX systems, define {\tt MSGHANDLER\/} to enable it.
To use
the capability, set the environment variable {\tt NETHACK\_MSGHANDLER\/} to
an executable, which will be executed with the game message as the program's
only parameter.
%.pg

The most crucial settings to make the game more accessible are:
%.pg
\blist{}
%.lp
\item[\ib{symset:plain}]
Load a symbol set appropriate for use by blind players.
%.lp
\item[\ib{menustyle:traditional}]
This will assist in the interface to speech synthesizers.
%.lp
\item[\ib{nomenu\verb+_+overlay}]
Show menus on a cleared screen and aligned to the left edge.
%.lp
\item[\ib{number\verb+_+pad}]
A lot of speech access programs use the number-pad to review the screen.
If this is the case, disable the number\verb+_+pad option and use the
traditional Rogue-like commands.
%.lp
\item[\ib{paranoid\verb+_+confirmation:swim}]
Prevent walking into water or lava.
%.lp
\item[\ib{autodescribe}]
Automatically describe the terrain under the cursor when targeting.
%.lp
\item[\ib{mention\verb+_+walls}]
Give feedback messages when walking towards a wall or when travel command
was interrupted.
%.lp
\item[\ib{whatis\verb+_+coord:compass}]
When targeting with cursor, describe the cursor position with coordinates
relative to your character.
%.lp
\item[\ib{whatis\verb+_+filter:area}]
When targeting with cursor, filter possible locations so only those in
the same area (eg. same room, or same corridor) are considered.
%.lp
\item[\ib{whatis\verb+_+moveskip}]
When targeting with cursor and using fast-move, skip the same glyphs instead
of moving 8 units at a time.
%.lp
\item[\ib{nostatus\verb+_+updates}]
Prevent updates to the status lines at the bottom of the screen, if
your screen-reader reads those lines. The same information can be
seen via the {\tt \#attributes} command.
\elist

%.hn2
\subsection*{Global Configuration for System Administrators}

%.pg
If {\it NetHack\/} is compiled with the SYSCF option, a system administrator
should set up a global configuration; this is a file in the
same format as the traditional per-user configuration file (see above).

This file should be named sysconf and placed in the same directory as
the other {\it NetHack\/} support files.
The options recognized in this file are listed below. Any option not
set uses a compiled-in default (which may not be appropriate for your
system).

%.pg
\blist{}
%.lp
\item[\ib{WIZARDS}]
A space-separated list of user name who are allowed to
play in debug mode (commonly referred to as wizard mode).
A value of a single
asterisk (*) allows anyone to start a game in debug mode.
%.lp
\item[\ib{SHELLERS}]
A list of users who are allowed to use the shell escape command (`{\tt !}').
The syntax is the same as WIZARDS.
%.lp
\item[\ib{EXPLORERS}]
A list of users who are allowed to use the explore mode.
The syntax is the same as WIZARDS.
%.lp
\item[\ib{MAXPLAYERS}]
Limit the maximum number of games that can be running at the same time.
%.lp
\item[\ib{SAVEFORMAT}]
A list of up to two save file formats separated by space. 
The first format in the list will written as well as read. The second format 
will be read only if no save file in the first format exists.
Valid choices are ``{\tt historical}'' for binary writing of entire structs, 
``{\tt lendian}'' for binary writing of each field in little-endian order,
``{\tt ascii}'' for writing the save file content in ascii text.
%.lp
\item[\ib{BONESFORMAT}]
A list of up to two bones file formats separated by space. 
The first format in the list will written as well as read. The second
format will be read only if no bones files in the first format exist.
Valid choices are ``{\tt historical}'' for binary writing of entire structs, 
``{\tt lendian}'' for binary writing of each field in little-endian order,
``{\tt ascii}'' for writing the bones file content in ascii text.
%.lp
\item[\ib{SUPPORT}]
A string explaining how to get local support (no default value).
%.lp
\item[\ib{RECOVER}]
A string explaining how to recover a game on this system (no default value).
%.lp
\item[\ib{SEDUCE}]
0 or 1 to disable or enable, respectively, the SEDUCE option.
When disabled, incubi and succubi behave like nymphs.
%.lp
\item[\ib{CHECK\verb+_+PLNAME}]
Setting this to 1 will make the EXPLORERS, WIZARDS, and SHELLERS check
for the player name instead of the user's login name.
%.lp
\item[\ib{CHECK\verb+_+SAVE\verb+_+UID}]
0 or 1 to disable or enable, respectively, the UID
(used identification number) checking for save files (to verify that the
user who is restoring is the same one who saved).
\elist

%.pg
The following four options affect the score file:
\blist {}
%.pg
%.lp
\item[\ib{PERSMAX}]
Maximum number of entries for one person.
%.lp
\item[\ib{ENTRYMAX}]
Maximum number of entries in the score file.
%.lp
\item[\ib{POINTSMIN}]
Minimum number of points to get an entry in the score file.
%.lp
\item[\ib{PERS\verb+_+IS\verb+_+UID}]
0 or 1 to use user names or numeric userids, respectively, to identify
unique people for the score file.
%.lp
\item[\ib{HIDEUSAGE}]
0 or 1 to control whether the help menu entry for command
line usage is shown or suppressed.
%.lp
\item[\ib{MAX\verb+_+STATUENAME\verb+_+RANK}]
Maximum number of score file entries to use for
random statue names (default is 10).
%.lp
\item[\ib{ACCESSIBILITY}]
0 or 1 to disable or enable, respectively, the ability for players
to set S\verb+_+pet\verb+_+override and S\verb+_+hero\verb+_+override 
symbols in their configuration file.
%.lp
\item[\ib{PORTABLE\verb+_+DEVICE\verb+_+PATHS}]
0 or 1 Windows OS only, the game will look for all of its external
files, and write to all of its output files in one place 
rather than at the standard locations.
%.lp
\item[\ib{DUMPLOGFILE}]
A filename where the end-of-game dumplog is saved.
Not defining this will prevent dumplog from being created.
Only available if your game is compiled with DUMPLOG.
Allows the following placeholders:
% FIXME: this should be changed to a nested list or else be forcibly indented
{\tt \%\%}  --- literal `{\tt \%}'\\
{\tt \%v}  --- version (eg. ``{\tt 3.7.0-0}'')\\
{\tt \%u}  --- game UID\\
{\tt \%t}  --- game start time, UNIX timestamp format\\
{\tt \%T}  --- current time, UNIX timestamp format\\
{\tt \%d}  --- game start time, YYYYMMDDhhmmss format\\
{\tt \%D}  --- current time, YYYYMMDDhhmmss format\\
{\tt \%n}  --- player name\\
{\tt \%N}  --- first character of player name
%.lp
\item[\ib{LIVELOG}]
A bit-mask of types of events that should be written to
the {\it livelog\/} file if one is present.
The sample {\it sysconf\/} file accompanying the program contains a
comment which lists the meaning of the various bits used.
Intended for server systems supporting simultaneous play by multiple
players (to be clear, each one running a separate single player game),
for displaying their game progress to observers.
Only relevant if the program was built with LIVELOG enabled.
When available, it should be left commented out on single player
installations because over time the file could grow to be extremely
large unless it is actively maintained.
\elist

%.hn 1
\section{Scoring}

%.pg
{\it NetHack\/} maintains a list of the top scores or scorers on your machine,
depending on how it is set up.  In the latter case, each account on
the machine can post only one non-winning score on this list.  If
you score higher than someone else on this list, or better your
previous score, you will be inserted in the proper place under your
current name.  How many scores are kept can also be set up when
{\it NetHack\/} is compiled.

%.pg
Your score is chiefly based upon how much experience you gained, how
much loot you accumulated, how deep you explored, and how the game
ended.  If you quit the game, you escape with all of your gold intact.
If, however, you get killed in the Mazes of Menace, the guild will
only hear about 90\,\% of your gold when your corpse is discovered
(adventurers have been known to collect finder's fees).  So, consider
whether you want to take one last hit at that monster and possibly
live, or quit and stop with whatever you have.  If you quit, you keep
all your gold, but if you swing and live, you might find more.

%.pg
If you just want to see what the current top players/games list is, you
can type
\begin{verbatim}
    nethack -s all
\end{verbatim}
on most versions.

%.hn 1
\section{Explore mode}

%.pg
{\it NetHack\/} is an intricate and difficult game.  Novices might falter
in fear, aware of their ignorance of the means to survive.  Well, fear
not.  Your dungeon comes equipped with an ``explore'' or ``discovery''
mode that enables you to keep old save files and cheat death, at the
paltry cost of not getting on the high score list.

%.pg
There are two ways of enabling explore mode.  One is to start the game
with the {\tt -X}
command-line switch or with the
{\it playmode:explore\/}
option.  The other is to issue the `{\tt \#exploremode}' extended command while
already playing the game.  Starting a new game in explore mode provides your
character with a wand of wishing in initial inventory; switching
during play does not.  The other benefits of explore mode are left for
the trepid reader to discover.

%.pg
%.hn 2
\subsection*{Debug mode}

%.pg
Debug mode, also known as wizard mode, is undocumented aside from this
brief description and the various ``debug mode only'' commands listed
among the command descriptions.
It is intended for tracking down problems within the
program rather than to provide god-like powers to your character, and
players who attempt debugging are expected to figure out how to use it
themselves.
It is initiated by starting the game with the
{\tt -D}
command-line switch or with the
{\it playmode:debug\/}
option.

%.pg
For some systems, the player must be logged in
under a particular user name to be allowed to use debug mode; for others,
the hero must be given a particular character name (but may be any role;
there's no connection between ``wizard mode'' and the {\it Wizard\/} role).
Attempting to start a game in debug mode when not allowed
or not available will result in falling back to explore mode instead.

%.hn
\section{Credits}
%.pg
The original %
{\it hack\/} game was modeled on the Berkeley
%.ux
UNIX
{\it rogue\/} game.  Large portions of this document were shamelessly
cribbed from %
{\it A Guide to the Dungeons of Doom}, by Michael C. Toy
and Kenneth C. R. C. Arnold.  Small portions were adapted from
{\it Further Exploration of the Dungeons of Doom}, by Ken Arromdee.

%.pg
{\it NetHack\/} is the product of literally scores of people's work.
Main events in the course of the game development are described below:

%.pg
\bigskip
\nd {\it Jay Fenlason\/} wrote the original {\it Hack}, with help from {\it
Kenny Woodland}, {\it Mike Thome}, and {\it Jon Payne}.

%.pg
\medskip
\nd {\it Andries Brouwer\/} did a major re-write while at
Stichting Mathematisch Centrum (now Centrum Wiskunde \& Informatica),
transforming Hack into a very different game.
He published the Hack source code for use on UNIX
systems by posting that to Usenet
newsgroup {\it net.sources\/} (later renamed {\it comp.sources})
releasing version 1.0 in December of 1984, then versions 1.0.1, 1.0.2,
and finally 1.0.3 in July of 1985.
Usenet newsgroup {\it net.games.hack\/} (later
renamed {\it rec.games.hack}, eventually replaced
by {\it rec.games.roguelike.nethack})
was created for discussing it.

%.pg
\medskip
\nd {\it Don G. Kneller\/} ported {\it Hack\/} 1.0.3 to Microsoft C and MS-DOS,
producing {\it PC Hack\/} 1.01e, added support for DEC Rainbow graphics in
version 1.03g, and went on to produce at least four more versions (3.0, 3.2,
3.51, and 3.6;
note that these are old {\it Hack\/} version numbers, not contemporary
{\it NetHack\/} ones).

%.pg
\medskip
\nd {\it R. Black\/} ported {\it PC Hack\/} 3.51 to Lattice C and the Atari
520/1040ST, producing {\it ST Hack\/} 1.03.

%.pg
\medskip
\nd {\it Mike Stephenson\/} merged these various versions back together,
incorporating many of the added features, and produced {\it NetHack\/} version
1.4 in 1987.
He then coordinated a cast of thousands in enhancing and debugging
{\it NetHack\/} 1.4 and released {\it NetHack\/} versions 2.2 and 2.3.
Like Hack, they were released by posting their source code to Usenet where
they remained available in various archives accessible
via {\it ftp\/} and {\it uucp\/} after expiring from the newsgroup.

%.pg
\medskip
\nd Later, Mike coordinated a major re-write of the game, heading a team which
included {\it Ken Arromdee}, {\it Jean-Christophe Collet}, {\it Steve Creps},
{\it Eric Hendrickson}, {\it Izchak Miller}, {\it Eric S. Raymond}, {\it John
Rupley}, {\it Mike Threepoint}, and {\it Janet Walz}, to produce
{\it NetHack\/} 3.0c.

%.pg
\medskip
\nd {\it NetHack\/} 3.0 was ported to the Atari by {\it Eric R. Smith}, to OS/2 by
{\it Timo Hakulinen}, and to VMS by {\it David Gentzel}.  The three of them
and {\it Kevin Darcy\/} later joined the main {\it NetHack Development Team} to produce
subsequent revisions of 3.0.

%.pg
\medskip
\nd {\it Olaf Seibert\/} ported {\it NetHack\/} 2.3 and 3.0 to the Amiga.  {\it
Norm Meluch}, {\it Stephen Spackman\/} and {\it Pierre Martineau\/} designed
overlay code for {\it PC NetHack\/} 3.0.  {\it Johnny Lee\/} ported {\it
NetHack\/} 3.0 to the Macintosh.  Along with various other Dungeoneers, they
continued to enhance the PC, Macintosh, and Amiga ports through the later
revisions of 3.0.

%.pg
Version 3.0 went through ten relatively rapidly released ``patch-level''
revisions.
Versions at the time were known as 3.0 for the base release and variously
as ``3.0a'' through ``3.0j'',
``3.0~patchlevel~1'' through ``3.0~patchlevel~10'',
or ``3.0pl1'' through ``3.0pl10''
rather than 3.0.0 and 3.0.1 through 3.0.10;
the three component numbering scheme began to be used with 3.1.0.

%.pg
\medskip
\nd Headed by {\it Mike Stephenson\/} and coordinated by {\it Izchak Miller\/}
and {\it Janet Walz}, the {\it NetHack Development Team} which now included
{\it Ken Arromdee},
{\it David Cohrs}, {\it Jean-Christophe Collet}, {\it Kevin Darcy},
{\it Matt Day}, {\it Timo Hakulinen}, {\it Steve Linhart}, {\it Dean Luick},
{\it Pat Rankin}, {\it Eric Raymond}, and {\it Eric Smith\/} undertook a
radical revision of 3.0.
They re-structured the game's design, and re-wrote major
parts of the code.
They added multiple dungeons, a new display, special
individual character quests, a new endgame and many other new features, and
produced {\it NetHack\/} 3.1.
Version 3.1.0 was released in January of 1993.

%.pg
\medskip
\nd {\it Ken Lorber}, {\it Gregg Wonderly\/} and {\it Greg Olson}, with help
from {\it Richard Addison}, {\it Mike Passaretti}, and {\it Olaf Seibert},
developed {\it NetHack\/} 3.1 for the Amiga.

%.pg
\medskip
\nd {\it Norm Meluch\/} and {\it Kevin Smolkowski}, with help from
{\it Carl Schelin}, {\it Stephen Spackman}, {\it Steve VanDevender},
and {\it Paul Winner}, ported {\it NetHack\/} 3.1 to the PC.

%.pg
\medskip
\nd {\it Jon W\{tte} and {\it Hao-yang Wang},
with help from {\it Ross Brown}, {\it Mike Engber}, {\it David Hairston},
{\it Michael Hamel}, {\it Jonathan Handler}, {\it Johnny Lee},
{\it Tim Lennan}, {\it Rob Menke}, and {\it Andy Swanson},
developed {\it NetHack\/} 3.1 for the Macintosh, porting it for MPW.
Building on their development, {\it Bart House} added a Think C port.

%.pg
\medskip
\nd {\it Timo Hakulinen\/} ported {\it NetHack\/} 3.1 to OS/2.
{\it Eric Smith\/} ported {\it NetHack\/} 3.1 to the Atari.
{\it Pat Rankin}, with help from {\it Joshua Delahunty},
was responsible for the VMS version of {\it NetHack\/} 3.1.
{\it Michael Allison} ported {\it NetHack\/} 3.1 to Windows NT.

%.pg
\medskip
\nd {\it Dean Luick}, with help from {\it David Cohrs}, developed
{\it NetHack\/} 3.1 for X11.
It drew the map as text rather than graphically but
included {\tt nh10.bdf}, an optionally used custom X11 font which has
tiny images in place of letters and punctuation, a precursor of tiles.
Those images don't extend to individual monster and object types, just
replacements for monster and object classes (so one custom image for all
``{\tt a}'' insects and another for all ``{\tt [}'' armor and so
forth, not separate images for beetles and ants or for cloaks and boots).

%.pg
\medskip
\nd {\it Warwick Allison\/} wrote a graphically displayed version
of {\it NetHack\/}
for the Atari where the tiny pictures were described as ``icons'' and
were distinct for specific types of monsters and objects rather than just
their classes.
He contributed them to the {\it NetHack Development Team\/} which
rechristened them ``tiles'', original usage which has subsequently been
picked up by various other games.
{\it NetHack's\/} tiles support was then implemented on other platforms
(initially MS-DOS but eventually Windows, Qt, and X11 too).

%.pg
\medskip
\nd The 3.2 {\it NetHack Development Team}, comprised of {\it Michael Allison}, {\it Ken
Arromdee}, {\it David Cohrs}, {\it Jessie Collet}, {\it Steve Creps}, {\it
Kevin Darcy}, {\it Timo Hakulinen}, {\it Steve Linhart}, {\it Dean Luick},
{\it Pat Rankin}, {\it Eric Smith}, {\it Mike Stephenson}, {\it Janet Walz},
and {\it Paul Winner}, released version 3.2.0 in April of 1996.

%.pg
\medskip
\nd Version 3.2 marked the tenth anniversary of the formation of the
development team.
In a testament to their dedication to the game, all thirteen members
of the original {\it NetHack Development Team} remained on the team at the
start of work on that release.
During the interval between the release of 3.1.3 and 3.2.0,
one of the founding members of the {\it NetHack Development Team},
{\it Dr. Izchak Miller}, was diagnosed with cancer and passed away.
That release of the game was
dedicated to him by the development and porting teams.

%.pg
Version 3.2 proved to be more stable than previous versions.
Many bugs were fixed, abuses eliminated, and game features tuned for
better game play.

%.pg
\medskip
During the lifespan of {\it NetHack\/} 3.1 and 3.2, several enthusiasts
of the game added
their own modifications to the game and made these ``variants'' publicly
available:

%.pg
\medskip
{\it Tom Proudfoot} and {\it Yuval Oren} created {\it NetHack++},
which was quickly renamed {\it NetHack$--$\/}
when some people incorrectly assumed that it was a conversion of the
{\it C\/} source code to {\it C++}.
Working independently, {\it Stephen White} wrote {\it NetHack Plus}.
{\it Tom Proudfoot} later merged {\it NetHack Plus}
and his own {\it NetHack$--$} to produce {\it SLASH}.
{\it Larry Stewart-Zerba} and {\it Warwick Allison} improved the spell
casting system with the Wizard Patch.
{\it Warwick Allison} also ported {\it NetHack\/} to use the Qt interface.

%.pg
\medskip
{\it Warren Cheung} combined {\it SLASH} with the Wizard Patch
to produce {\it Slash'EM\/}, and
with the help of {\it Kevin Hugo}, added more features.
Kevin later joined the {\it NetHack Development Team} and incorporated
the best of these ideas into {\it NetHack\/} 3.3.

%.pg
\medskip
The final update to 3.2 was the bug fix release 3.2.3, which was released
simultaneously with 3.3.0 in December 1999 just in time for the Year 2000.
Because of the newer version, 3.2.3 was released as a source code patch only,
without any ready-to-play distribution for systems that usually had such.

%.pg
(To anyone considering resurrecting an old version:  all versions before
3.2.3 had a {\it Y2K\/} bug.
The high scores file and the log file contained
dates which were formatted using a two-digit year, and 1999's year 99 was
followed by 2000's year 100.
That got written out successfully but it
unintentionally introduced an extra column in the file layout which prevented
score entries from being read back in correctly, interfering with insertion
of new high scores and with retrieval of old character names to use for
random ghost and statue names in the current game.)

%.pg
\medskip
The 3.3 {\it NetHack Development Team}, consisting of {\it Michael Allison}, {\it Ken Arromdee},
{\it David Cohrs}, {\it Jessie Collet}, {\it Steve Creps}, {\it Kevin Darcy},
{\it Timo Hakulinen}, {\it Kevin Hugo}, {\it Steve Linhart}, {\it Ken Lorber},
{\it Dean Luick}, {\it Pat Rankin}, {\it Eric Smith}, {\it Mike Stephenson},
{\it Janet Walz}, and {\it Paul Winner}, released 3.3.0 in
December 1999 and 3.3.1 in August of 2000.

%.pg
\medskip
Version 3.3 offered many firsts. It was the first version to separate race
and profession. The Elf class was removed in preference to an elf race,
and the races of dwarves, gnomes, and orcs made their first appearance in
the game alongside the familiar human race.  Monk and Ranger roles joined
Archeologists, Barbarians, Cavemen, Healers, Knights, Priests, Rogues, Samurai,
Tourists, Valkyries and of course, Wizards.  It was also the first version
to allow you to ride a steed, and was the first version to have a publicly
available web-site listing all the bugs that had been discovered.  Despite
that constantly growing bug list, 3.3 proved stable enough to last for
more than a year and a half.

%.pg
\medskip
The 3.4 {\it NetHack Development Team} initially consisted of
{\it Michael Allison}, {\it Ken Arromdee},
{\it David Cohrs}, {\it Jessie Collet}, {\it Kevin Hugo}, {\it Ken Lorber},
{\it Dean Luick}, {\it Pat Rankin}, {\it Mike Stephenson},
{\it Janet Walz}, and {\it Paul Winner}, with {\it  Warwick Allison} joining
just before the release of {\it NetHack\/} 3.4.0 in March 2002.

%.pg
\medskip
As with version 3.3, various people contributed to the game as a whole as
well as supporting ports on the different platforms that {\it NetHack\/}
runs on:

%.pg
\medskip
\nd{\it Pat Rankin} maintained 3.4 for VMS.

%.pg
\medskip
\nd {\it Michael Allison} maintained {\it NetHack\/} 3.4 for the MS-DOS
platform.
{\it Paul Winner} and {\it Yitzhak Sapir} provided encouragement.

%.pg
\medskip
\nd {\it Dean Luick}, {\it Mark Modrall}, and {\it Kevin Hugo} maintained and
enhanced the Macintosh port of 3.4.

%.pg
\medskip
\nd {\it Michael Allison}, {\it David Cohrs}, {\it Alex Kompel},
{\it Dion Nicolaas}, and
{\it Yitzhak Sapir} maintained and enhanced 3.4 for the Microsoft Windows
platform.
{\it Alex Kompel} contributed a new graphical interface for the Windows port.
{\it Alex Kompel} also contributed a Windows CE port for 3.4.1.

%.pg
\medskip
\nd {\it Ron Van Iwaarden} was the sole maintainer of {\it NetHack\/} for
OS/2 the past
several releases. Unfortunately Ron's last OS/2 machine stopped working in
early 2006. A great many thanks to Ron for keeping {\it NetHack\/} alive on
OS/2 all these years.

%.pg
\medskip
\nd {\it Janne Salmij\"{a}rvi} and {\it Teemu Suikki} maintained
and enhanced the Amiga port of 3.4 after {\it Janne Salmij\"{a}rvi} resurrected
it for 3.3.1.

%.pg
\medskip
\nd {\it Christian ``Marvin'' Bressler} maintained 3.4 for the Atari after he
resurrected it for 3.3.1.

%.pg
\medskip
The release of {\it NetHack\/} 3.4.3 in December 2003 marked the beginning of
a long release hiatus. 3.4.3 proved to be a remarkably stable version that
provided continued enjoyment by the community for more than a decade. The
{\it NetHack Development Team} slowly and quietly continued to work on the game behind the scenes
during the tenure of 3.4.3. It was during that same period that several new
variants emerged within the {\it NetHack\/} community. Notably sporkhack by
Derek S. Ray, {\it unnethack\/} by Patric Mueller, {\it nitrohack\/} and its
successors originally by Daniel Thaler and then by Alex Smith, and
{\it Dynahack\/} by Tung Nguyen.
Some of those variants continue to be
developed, maintained, and enjoyed by the community to this day.

%.pg
\medskip
In September 2014, an interim snapshot of the code under development was
released publicly by other parties.
Since that code was a work-in-progress
and had not gone through the process of debugging it as a suitable release,
it was decided that the version numbers present on that code snapshot would
be retired and never used in an official {\it NetHack\/} release.
An announcement was posted on the {\it NetHack Development Team}'s official
{\it nethack.org\/} website
to that effect, stating that there would never be a 3.4.4, 3.5, or 3.5.0
official release version.

%.pg
\medskip
In January 2015, preparation began for the release of NetHack 3.6.

%.pg
\medskip
At the beginning of development for what would eventually get released as
3.6.0, the {\it NetHack Development Team} consisted of {\it Warwick Allison},
{\it Michael Allison}, {\it Ken Arromdee},
{\it David Cohrs}, {\it Jessie Collet},
{\it Ken Lorber}, {\it Dean Luick}, {\it Pat Rankin},
{\it Mike Stephenson}, {\it Janet Walz}, and {\it Paul Winner}.
In early 2015, ahead of the release of 3.6.0, new members
{\it Sean Hunt}, {\it Pasi Kallinen}, and {\it Derek S. Ray}
joined the {\it NetHack\/} development team.

%.pg
\medskip
Near the end of the development of 3.6.0, one of the significant inspirations
for many of the humorous and fun features found in the game,
author Terry Pratchett, passed away. {\it NetHack\/} 3.6.0 introduced
a tribute to him.

%.pg
\medskip
3.6.0 was released in December 2015, and merged work done by the development
team since the release of 3.4.3 with some of the beloved community
patches.  Many bugs were fixed and some code was restructured.

%.pg
\medskip
The {\it NetHack Development Team}, as well as {\it Steve VanDevender} and
{\it Kevin Smolkowski}, ensured that {\it NetHack\/} 3.6 continued to
operate on various UNIX flavors and maintained the X11 interface.

%.pg
\medskip
{\it Ken Lorber}, {\it Haoyang Wang}, {\it Pat Rankin}, and {\it Dean Luick}
maintained the port of {\it NetHack\/} 3.6 for MacOS.

%.pg
\medskip
{\it Michael Allison}, {\it David Cohrs}, {\it Bart House},
{\it Pasi Kallinen}, {\it Alex Kompel}, {\it Dion Nicolaas},
{\it Derek S. Ray} and  {\it Yitzhak Sapir}
maintained the port of  {\it NetHack\/} 3.6 for Microsoft Windows.

%.pg
\medskip
{\it Pat Rankin} attempted to keep the VMS port running for NetHack 3.6,
hindered by limited access.  {\it Kevin Smolkowski} has updated and tested it
for the most recent version of OpenVMS (V8.4 as of this writing) on Alpha
and Integrity (aka Itanium aka IA64) but not VAX.

%.pg
\medskip
{\it Ray Chason}  resurrected the MS-DOS port for 3.6 and contributed the
necessary updates to the community at large.

%.pg
\medskip
In late April 2018, several hundred bug fixes for 3.6.0 and some new features
were assembled and released as NetHack 3.6.1.
The {\it NetHack Development Team} at the
time of release of 3.6.1 consisted of
{\it Warwick Allison}, {\it Michael Allison}, {\it Ken Arromdee},
{\it David Cohrs}, {\it Jessie Collet},
{\it Pasi Kallinen}, {\it Ken Lorber}, {\it Dean Luick},
{\it Patric Mueller}, {\it Pat Rankin}, {\it Derek S. Ray},
{\it Alex Smith}, {\it Mike Stephenson}, {\it Janet Walz}, and
{\it Paul Winner}.

%.pg
\medskip
In early May 2019, another 320 bug fixes along with some enhancements and
the adopted curses window port, were released as 3.6.2.

%.pg
\medskip
{\it Bart House}, who had contributed to the game as a porting team participant
for decades, joined the {\it NetHack Development Team} in late May 2019.

%.pg
\medskip
NetHack 3.6.3 was released on December 5, 2019 containing over 190 bug
fixes to NetHack 3.6.2.

%.pg
\medskip
NetHack 3.6.4 was released on December 18, 2019 containing a security fix and
a few bug fixes.

%.pg
\medskip
NetHack 3.6.5 was released on January 27, 2020 containing some security fixes
and a small number of bug fixes.

%.pg
\medskip
NetHack 3.6.6 was released on March 8, 2020 containing a security fix and
some bug fixes.

%.pg
\medskip
NetHack 3.6.7 was released on February 16, 2023 containing a security fix and
some bug fixes.

%.pg
\medskip
\nd The official {\it NetHack\/} web site is maintained by {\it Ken Lorber} at
{\catcode`\#=11
\special{html:<a href="https://www.nethack.org/">}}
https:{\tt /}{\tt /}www.nethack.org{\tt /}.
{\catcode`\#=11
\special{html:</a>}}

%.pg
%.hn 2

\subsection*{Special Thanks}
\nd On behalf of the {\it NetHack\/} community, thank you very much once
again to {\it M. Drew Streib} and {\it Pasi Kallinen} for providing a
public NetHack server at nethack.alt.org. Thanks to {\it Keith Simpson}
and {\it Andy Thomson} for hardfought.org. Thanks to all those
unnamed dungeoneers who invest their time and effort into annual
{\it NetHack\/} tournaments such as {\it Junethack},
{\it The November NetHack Tournament}, and in days past,
{\it devnull.net\/} (gone for now, but not forgotten).
\clearpage

%.hn 2
\subsection*{Dungeoneers}
%.pg
\nd From time to time, some depraved individual out there in netland sends a
particularly intriguing modification to help out with the game.  The
{\it NetHack Development Team} sometimes makes note of the names of the worst
of these miscreants in this, the list of Dungeoneers:
\medskip
%.sd
\begin{center}
\begin{tabular}{llll}
%TABLE_START
Adam Aronow & J. Ali Harlow & Mikko Juola\\
Alex Kompel & Janet Walz & Nathan Eady\\
Alex Smith & Janne Salmij\"{a}rvi & Norm Meluch\\
Andreas Dorn & Jean-Christophe Collet & Olaf Seibert\\
Andy Church & Jeff Bailey & Pasi Kallinen\\
Andy Swanson & Jochen Erwied & Pat Rankin\\
Andy Thomson & John Kallen & Patric Mueller\\
Ari Huttunen & John Rupley & Paul Winner\\
Bart House & John S. Bien & Pierre Martineau\\
Benson I. Margulies & Johnny Lee & Ralf Brown\\
Bill Dyer & Jon W\{tte & Ray Chason\\
Boudewijn Waijers & Jonathan Handler & Richard Addison\\
Bruce Cox & Joshua Delahunty & Richard Beigel\\
Bruce Holloway & Karl Garrison & Richard P. Hughey\\
Bruce Mewborne & Keizo Yamamoto & Rob Menke\\
Carl Schelin & Keith Simpson & Robin Bandy\\
Chris Russo & Ken Arnold & Robin Johnson\\
David Cohrs & Ken Arromdee & Roderick Schertler\\
David Damerell & Ken Lorber & Roland McGrath\\
David Gentzel & Ken Washikita & Ron Van Iwaarden\\
David Hairston & Kevin Darcy & Ronnen Miller\\
Dean Luick & Kevin Hugo & Ross Brown\\
Del Lamb & Kevin Sitze & Sascha Wostmann\\
Derek S. Ray & Kevin Smolkowski & Scott Bigham\\
Deron Meranda & Kevin Sweet & Scott R. Turner\\
Dion Nicolaas & Lars Huttar & Sean Hunt\\
Dylan O'Donnell & Leon Arnott & Stephen Spackman\\
Eric Backus & M. Drew Streib & Stefan Thielscher\\
Eric Hendrickson & Malcolm Ryan & Stephen White\\
Eric R. Smith & Mark Gooderum & Steve Creps\\
Eric S. Raymond & Mark Modrall & Steve Linhart\\
Erik Andersen & Marvin Bressler & Steve VanDevender\\
Fredrik Ljungdahl & Matthew Day & Teemu Suikki\\
Frederick Roeber & Merlyn LeRoy & Tim Lennan\\
Gil Neiger & Michael Allison & Timo Hakulinen\\
Greg Laskin & Michael Feir & Tom Almy\\
Greg Olson & Michael Hamel & Tom West\\
Gregg Wonderly & Michael Sokolov & Warren Cheung\\
Hao-yang Wang & Mike Engber & Warwick Allison\\
Helge Hafting & Mike Gallop & Yitzhak Sapir\\
Irina Rempt-Drijfhout & Mike Passaretti\\
Izchak Miller & Mike Stephenson
%TABLE_END  Do not delete this line.
\end{tabular}
\end{center}
%.ed
\clearpage
%\vfill
%\begin{flushleft}
%\small
%Microsoft and MS-DOS are registered trademarks of Microsoft Corporation.\\
%%%Don't need next line if a UNIX macro automatically inserts footnotes.
%UNIX is a registered trademark of AT\&T.\\
%Lattice is a trademark of Lattice, Inc.\\
%Atari and 1040ST are trademarks of Atari, Inc.\\
%Amiga is a trademark of Commodore-Amiga, Inc.\\
%%.sm
%Brand and product names are trademarks or registered trademarks
%of their respective holders.
%\end{flushleft}

\end{document}
